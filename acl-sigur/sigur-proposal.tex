\documentclass[11pt,a4paper]{article}
\usepackage[margin=70pt,head=50pt,headsep=30pt,foot=50pt]{geometry}
\usepackage{fontspec}
\usepackage{xltxtra}
\usepackage{xspace}
\usepackage[english]{babel}
\usepackage{calc}
\usepackage[table]{xcolor}
\usepackage{verbatim}
\usepackage[normalem]{ulem}
\usepackage{multicol}
\usepackage[small,bf]{caption}
\usepackage{multirow}
\usepackage{natbib}
\usepackage{tikz-dependency}
\usepackage{setspace}
\usepackage{fancyhdr}
\usepackage{expex} % another possibility
\usepackage{url}


\title{Proposal: Creation of SIGUR (ACL SIG on Uralic languages)}
\date{}

\begin{document}
\maketitle

Dear ACL Executive,\\
~\\
We submit this proposal to form a new ACL SIG on
computational linguistics of Uralic languages. The
aim of this SIG is to promote interest in
computational approaches to research on Uralic
linguistics. 

\section*{Past events}

The main proposers of the SIG have had experience organising a number of workshops 
on language technology for lesser-resourced and marginalised languages, such as 
the SALTMIL series of workshops, which have attracted submissions related to 
Uralic languages.

Building on this, we organised a workshop on computational linguistics for the 
Uralic languages in Tromsø,
Norway\footnote{\url{http://gtweb.uit.no/iwclul2015/}}
in January this year. The prospect of forming an ACL
SIG was discussed among the participants during the closing session and all
agreed that it would be a good time to found a SIG
in ACL.

We have issued a call for papers in Northern European Journal of Language 
Technology (NEJLT) to publish a special issue on Uralic language
technology containing revised papers from our workshop and also an open 
call for new submissions.

Prior to the workshop conference series like NODALIDA have been central publication channel for the work on Uralic computational linguistics within the Nordic countries. However, NODALIDA mainly concerns researchers in / about Nordic, while much of Uralistics research is conducted in central Europe and Russia as well. One of the points of forming the SIG is to extend the co-operation and interaction to all sites interested in Uralistics.
For non-computational linguistics there's long tradition of conferences in Uralistics, such as the International Congress of Finno-Ugric studies (CIFU). In recent years they have included computational methods as special workshops.  Hungary has an annual conference on computational linguistics in Hungary including Hungarian works on Uralic languages.\footnote{\url{http://rgai.inf.u-szeged.hu/mszny2015/}}

\section*{Future plans}

For the next year we are organising the second edition of the 
International Workshop on Computational Linguistics for the Uralic
Languages  in co-operation with
the Hungarian Academy of science and the University of Szeged.\footnote{\url{http://rgai.inf.u-szeged.hu/iwclul2016/}} The workshop
will be held in Szeged in January 2016. The workshop would be an ideal place to hold the election of the formed SIG board.

In preparation of the second conference we also discussed of matters of unification of systems and frameworks for computational grammars including universal dependency issues. We hope that the SIG will further help such continuing cooperations.

A large number of current candidate members will participate CIFU XII International Congress of Finno-Ugric studies organised this year in Oulu on September\footnote{\url{http://www.oulu.fi/suomenkieli/fuxii/englanti/symposiumit}} we will arrange a meeting and hopefully gather more interested parties from linguistic circles. 

\section*{Rationale}

The existence of a special interest group specific to
Uralistics in computational linguistics is well-founded: on the
linguistic and computational side the Uralic languages share the
morphological and linguistic complexity that is characteristic to the whole
language group. The methodology for tackling the linguistic and
social / geo-political problems within the Uralic group is common and
the purpose of the SIG is to help researchers to share their work and
solutions. Research on Uralistics is scattered along sites in Northern
Europe, Russia, Hungary, Germany and Central Europe, the purpose of the
SIG is to promote co-operation among the sites and platforms for 
common work.

Currently some work is duplicated with computational grammars and resources,
our aim is to collect and unify the resources to avoid duplication of efforts.
At the moment we have an infrastructure for rule-based free/libre open-source computational linguistics descriptions and tools in the
University of Tromsø. The founding of the SIG would hopefully
increase the interaction with various sites on the sharing of the
resources. We aim to collect a resource database and related web-site for all resources related to computational linguistics of uralistics as well as cooperating on unification such as best common practices in software engineering, tagging schemata and so froth.

As specified in constitution, we propose following acting officers for the intermediate term between approval of application and founding meeting, all in good standing with ACL:

\begin{itemize}
\item Chair: Tommi A. Pirinen
\item Secretary, Liaison Officer: Francis M. Tyers
\end{itemize}

An election will be held as specified in the constitution at latest by the end of second International Workshop for Computational Linguistics for the Uralic languages in Szeged, Hungary in January 2016.

\section*{Expressions of interest}

The following list contains 32 people who have expressed an interest in 
participating in the SIG:

\begin{itemize}
  \item Francis M. Tyers (UiT Norgga árktalaš universitehta)
  \item Tommi A. Pirinen (Ollscoil Chathair Bhaile Átha Cliath)
  \item Trond Trosterud (UiT Norgga árktalaš universitehta)
  \item Marja-Liisa Olthuis (UiT Norgga árktalaš universitehta)
  \item Rogier Blokland (Uppsala universitet)
  \item Niko Partanen (Albert-Ludwigs-Universität Freiburg)
  \item Wouter Van Hemel (Helsingin yliopisto)
  \item Jussi Ylikoski (UiT Norgga árktalaš universitehta)
  \item Lene Antonsen (UiT Norgga árktalaš universitehta)
  \item Anis Moubarik (Helsingin yliopisto)
  \item Aarne Ranta (Chalmers tekniska högskola)
  \item Kaili Müürisep (Tartu ülikooli)
  \item Tiina Puolakainen (Tartu ülikooli)
  \item Péter Koczka (Magyar Tudományos Akadémia)
  \item Eszter Simon (Magyar Tudományos Akadémia)
  \item Marina Fedina (Komi republican academy of state service and administration)
  \item Ivett Benyeda (Magyar Tudományos Akadémia)
  \item Zsófia Schön (Ludwig-Maximilians-Universität München)
  \item Inari Listenmaa (Chalmers tekniska högskola)
  \item Jaak Pruulmann-Vengerfeldt (Tartu ülikooli)
  \item Heiki-Jaan Kaalep (Tartu ülikooli)
  \item Miikka Silfverberg (Helsingin yliopisto)
  \item Kimmo Koskenniemi (Helsingin yliopisto)
  \item Ciprian Gerstenberger (UiT Norgga árktalaš universitehta)
  \item Jeremy Bradley (University of Vienna)
  \item Stig-Arne Grönroos (Aalto yliopisto)
  \item Niklas Laxström (Helsingin yliopisto)
  \item Linda Wiechetek (UiT Norgga árktalaš universitehta)
  \item Sjur Moshagen (Sámediggi)
  \item Jack Rueter (Helsingin yliopisto)
  \item Veronika Vincze (Szegedi Tudományegyetem)
  \item Jonathan North Washington (University of Indiana)
\end{itemize}


\end{document}
