\documentclass[11pt,a4paper]{article}
\usepackage[margin=70pt,head=50pt,headsep=30pt,foot=50pt]{geometry}
\usepackage{fontspec}
\usepackage{xltxtra}
\usepackage{xspace}
\usepackage[english]{babel}
\usepackage{calc}
\usepackage[table]{xcolor}
\usepackage{verbatim}
\usepackage[normalem]{ulem}
\usepackage{multicol}
\usepackage[small,bf]{caption}
\usepackage{multirow}
\usepackage{natbib}
\usepackage{tikz-dependency}
\usepackage{setspace}
\usepackage{fancyhdr}
\usepackage{expex} % another possibility
\usepackage{url}


\title{Constitution of ACL Special Interest Group on Uralic languages}
\date{}

\begin{document}
\maketitle

\section*{I. STATEMENT OF PURPOSE}

The purpose of the Association for Computational Linguistics (ACL) Special Interest Group on Uralic languages (the SIG, SIGUR) shall be to promote interest in Uralic languages; to provide members of the ACL with a special interest in Uralic languages with a means of exchanging news of recent research developments and other matters of interest to Uralic languages; and to sponsor meetings and workshops in Uralic languages that appear to be timely and worthwile, operating within the framework of the ACL's general guidelines for SIGs.

The reason for existence of the special interest group for specific to Uralic languages in computational linguistics is well-founded: on the linguistic and computational side the Uralic languages share the morphological and linguistic complexity that is characteristic to whole language group. The methodology for tackling the linguistic and social / geo-political problems within the Uralic group is common and the purpose of the SIG is to help researchers to share their work and solutions. Research on Uralic languages is scattered along sites in Northern Europe, Russia, Hungary, Germany and Central Europe, the purpose of the SIG is to promote co-operation among the sites and platforms for the common work.

\section*{II. ELECTED OFFICERS}

The elected officers of the SIG shall consist of a President (or Chair), a Secretary and up to eight additional members, whose duties are determined in the officers' assembly meeting to be held within a month after the elected officers start their terms. The President and Secretary shall be members in good standing of the ACL. The elected officers form the SIG board.

The term of all elected officers of the SIG shall be 2 years.

The duties of the President (or Chair) shall be:
\begin{itemize}
\item    To have primary executive authority over actions and activities of the SIG.
\item    To prepare a written report on the activities of the SIG for the Executive Committee of the ACL, for presentation to the ACL at its Annual Business Meeting.
\end{itemize}
The duties of the Secretary shall be:

\begin{itemize}
\item    To maintain a membership roster of the SIG.
\item    To be responsible for any moneys awarded to the SIG by the ACL; to collect and manage any dues that may be required by the organisation; and to present a written annual report on SIG finances to the Executive Committee of the ACL.
\item    To act as a Liaison Representative, who shall be primarily responsible for communication with members of the SIG, answering inquiries about the SIG, and communication with the Executive Committee of the ACL.
\end{itemize}

Further roles of board members are determined by the board assembly meeting, and detailed in the SIGUR website. The board assembly may decide to elect members whose role overlaps Secretary's, such as Treasurer or Liaison Representative, in which case they carry the main responsibility of that given task of Secretary with Secretary acting as vice member for the role.

\section*{III. ELECTION OF OFFICERS}

All officers of the SIG shall be elected by a vote of the membership. The vote shall take place by the end of November prior to the expiration of the terms of the officers to be elected, with at least 4 weeks notice of nominations to SIG members. The candidates are nominated by SIG members, must be members of the SIG and must accept the nomination. Elected officers must organise an officers' assembly meting within a month of the beginning of their term to decide roles of the officers, however, the role of president and secretary is voted for in the November elections.

The normal term of a board begins on the 1st of January and runs for 2 years, however, the board may decide to hold elections for whole board or additional members sooner, as required. If a new board is elected, its term will end by the 1st of January in its second year. The elections may be held over the internet as defined by terms of section VI.

\section*{IV. DUES}

The ACL Special Interest Group does not collect dues from its members. In case it is seen necessary, the SIG board meeting may decide to instate such dues, as specified in section V. Any such dues will be subject to the approval of the Executive Committee of the ACL.

\section*{V. REFERENDA and BOARD MEETINGS}

Elected officers have the authority to decide on the actions of the special interest group, including organisation of events, use of moneys held by the group, or changes to the constitution and rules of the group in meetings where quorum is present, and that have been announced to the SIG with an agenda posted to the group no less than four weeks prior to a meeting to be held. An announcement can be made and the meetings can be held in a suitable internet-mediated environment as defined in section VI.

Amendments to the constitution can be initiated by any SIG member with a formal request to the board and must be put into a vote by the SIG board in its next meeting, however no longer than six months later. Any amendments to the constitution must receive 2/3 majority vote. Any other business placed into vote requires a simple majority.

\section*{VI. MEETING and VOTING PROTOCOLS}

In addition to a face-to-face meetings, elections, votes and board meetings can be organised via an applicable internet-mediated forum, as long as it provides sufficient means to verify authenticity of voters. Possible media are: Mailing lists, group chats in fora such as IRC, Google Hangouts, etc. Means to verify authenticity of participants include PGP signed electronic mail on mailing list posts, invite only or keyword-based group chats initiated using encrypted mail to the participants, or encrypted messages in given group chat network. The use of PGP as authenticity determination requires a valid PGP key belonging to the web of trust of at least one other board member.

\section*{VII. MEMBERSHIP}

SIGUR member register is organised by the Secretary of the SIG board. The SIG can be joined by filling a form at official SIG events, by a web form on the SIG website or by written formal request to Secretary. Membership can be ended by a formal request to the SIG board. The SIG board may end a membership of a member acting against the interest of the SIG in a board meeting via a vote.

%--Organisers of SIGUR / de facto board, March 9th 2015

\end{document}
