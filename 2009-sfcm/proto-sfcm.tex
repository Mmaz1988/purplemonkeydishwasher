\documentclass{llncs}
\usepackage[T1]{fontenc}
\usepackage{url}

\sloppy

\begin{document}

\title{Strategies and efficiency of morpheme sub-lexicon compilation with 
HFST-lexc}

\author{Krister Lind\'{e}n \and Tommi Pirinen}

\institute{University Of Helsinki, Helsingin yliopisto 00000-FI, Finland}

\maketitle

\begin{abstract}
This paper presents various algorithms and practices
used for compilation of morpheme
sublexicons to final morphological analyser transducer or \emph{lexical
transducer}. The algorithms' performance is tested and compared against
traditional lexical transducer building tools by Xerox and against
morphology built with SFST toolkit.
\end{abstract}

\section{Introduction}

A lexicon compiler is a program which reads set of morphemes and concatenative
rules and creates a finite state transducer which maps legal combinations of
morphemes. The baseline formula for combining described here was written to
emulate ubiquitous lexc by Xerox, which has traditionally been used for same
task. In Xerox lexc, the named lexicons are sets of strings and regular
expressions, each associated with target lexicon. For example one might
have lexicon named \texttt{nouns} with entries, such as \texttt{cat}
and \texttt{dog}, both associated with target lexicon \texttt{number}.
Number lexicon might contain two entries, defining empty string for
singular \ldots

The problem of compiling lexical transducer\cite{} from lexicons and
entries consists at least three optimisation problems that are discussed
in this paper. First is compiling simple string entries to finite state format,
that is \emph{tokenisation} and \emph{compilation} problem. Second optimisation
problem discussed is that of combining entries of single lexicon to a
transducer, that is efficient \emph{union} entry transducers. Third problem
is that of dealing with the combinations of lexicon entries and lexicons,
i.e. \emph{morphotax} implementation. Fourth algorithm that is relevant to
creation of lexical transducer, is problem of combining phonological rules
with resulting transducer, it is dealt more in depth in papers \cite{}.

\section{Compilation strategies and formulae}

\ldots

\subsection{Lexicon entry compilation}

An entry in lexicon is represented in lexc formalism as line of code consisting
of lexical data, a continuation class and semicolon. The problem of entry
compilation in lexc format is segmentation of ambiguous strings, e.g. example
\ldots This segmentation can efficiently be performed either by using
transducer lookup with special tokenising transducer
or segmenting from a string.

\subsection{Lexicon building}

Lexicon building concerns with creating a disjunction of the entries of the
lexicon. Disjunction per se is quite straight forward process, so only
optimisation strategies presented here are ordering of the disjunctions so
that disjuncted parts are equal rather than disjuncting n entries to n+1th
entry. More effective strategy is to build lexicon as prefix tree of the
entries.

\subsection{Lexicon combinatorics}

Two strategies of combining finite state lexicons together against specified
morphotactics can be thought of. A trivial approach would be to simply draw
an epsilon arc from entry with continuation class of $X$ to the beginning of
lexicon named $X$. Here we present alternative formula, that uses only 
basic finite state algebra with lexicon transducers to achieve the same goal.


\subsubsection{Combining lexicons using standard finite state transducer algebra.}

Here we assume all standard finite state operations and definitions, that is operators
$\cup$ for union, $\cap$ for intersection, $\circ$ for composition, $-$ for
difference, and juxtaposition for concatenation, and symbols
$\epsilon$ for zero-length string. We define
$\Sigma, \Gamma, \Delta$ for subsets of symbol alphabet. For
subsets of pair alphabet, $\Pi$ represents subset of
product $\Sigma \times \Sigma$.
Let $\Sigma = \{a, b, \ldots\}$ for set of symbols used
in natural language.
Let $\Gamma$ be set of auxiliary symbols used in formulae such that
$\Gamma \cap \Sigma = \emptyset$. Further we define
$\Gamma_{J^{2}_{x}} \subset \Gamma$ as \emph{joiners} we use to delimit
lexicon entries and define finite state rules of morphotax. Single joiners
referring to lexicon name or continuation class $X$ we write as $J_{x}$.
Let $\Delta = \Sigma \cup \Gamma$ and further 
$\Delta_x = \Delta - J_x$, that is, $\Delta$, with all auxiliary symbols
referring to lexicon or continuation class $x$ are removed.

Let us define a single entry of lexicon, that is, a line of code in
lexc file. We refer this to as \emph{morpheme} per principle of making lexicons
consist of morphologically motivated items:

\begin{equation}
M = \Sigma^{\star} \Gamma_{J^{2}}
\end{equation}

A morpheme, when building a lexicon for finite state combinatoric formula is
an entry compiled to finite state form as defined in \ref{}, appended with
joiner named after the continuation class of entry.

A lexicon then is simply disjunction of morphemes $M$ prepended with joiner
referring to lexicon name:

\begin{equation}
L = \Gamma_{J^{2}} \bigcup_{n = 0}^{k}(M_{n}), \mathrm{where} n = 
\mathrm{number of morphemes}
\end{equation}

Then we can create legal combination of morphemes by composing over language
allowing any string where joiners appear as pairs:

\begin{equation}
F_J^2 = (\bigcup_{J_x \in J^2}(\Sigma^{\star} J_{x} J_{x} \Sigma^{\star}))^{\star} 
\end{equation}

\emph{Note.} $\Sigma$ on both sides of joiners is not strictly necessary, of
course.

To also take into account special beginning and ending lexicons, let
$J_{Root} \in J^{2}$ be starting lexicon, that can be part of the paired joiners
and $J_{\#} \notin J^{2}$ be ending lexicon that may not be otherwise used as
a joiner, we extend filter as:

\begin{equation}
F_J^{<,2,>} = J_{Root} \Sigma^{\star} F_J^2 \Sigma^{\star} J_{\#}
\end{equation}

The final lexicon can be created by simple composition:

\begin{equation}
L^{\star} \circ F_J^{<,2,>}
\end{equation}

\emph{Note.} Intersection may also be used, but this requires the obvious
extension of defining $\Delta$ and $\Sigma$ over actual pairs.

\subsection{Note about combining lexicons and rules}

refernce to intersect compose article and to hfst-intersect compose article?

\section{Performance tests and evaluations}

For performance testing we present three different morphologies, which employ
varying complexities on both morphotactic and phonologic levels. The
French full form morphological lexicon is readily inflected list of all
forms of all words, that is a single lexicon of 500000 entries\cite{morphalou}.
Finnish lexicon
is reformulation of morphology defined \cite{pirinen2008}, implemented using
SFST \cite{schmidt2007}.
For comparison also the compilation times of SFST version in same test
environment are provided. Third is Northern S\'{a}mi morphology from \ldots.



\section{Further research}

(((
These formulae make it possible to extend combinatorics over long distance
dependencies using e.g. filter of flag diacritics\cite{} or e.g.
directed balanced brackets over the filter as other auxiliary character,

It may be not too complicated to extend morpheme combinatorics
with cover filters for languages that use e.g. interdigitation morphotactics.
)))

\bibliographystyle{splncs}
\bibliography{hfst-all}

\end{document}
