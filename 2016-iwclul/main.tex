\documentclass[b5paper]{article}

%\usepackage{polyglossia}
\usepackage{fontspec}
%\usepackage{xunicode}
%\usepackage{xltxtra}
\usepackage{url}
\usepackage{hyperref}
\usepackage{expex}

\usepackage{todonotes}

\makeatletter
\def\blfootnote{\gdef\@thefnmark{}\@footnotetext}
\makeatother

\setmainfont[Mapping=tex-text]{Linux Libertine O}

\newif\ifcamready
\camreadyfalse

\title{The state of the art in machine translation for Finnish\blfootnote{
    This work is licensed under a Creative Commons Attribution–NoDerivatives
    4.0 International Licence.  Licence details:
    \url{http://creativecommons.org/licenses/by-nd/4.0/}}
}

\author{Tommi A. Pirinen\\
Ollscoil Chathair Bhaile Átha Cliath\\
ADAPT Centre---School of Computing\\
\url{tommi.pirinen@computing.dcu.ie}
\and Francis M. Tyers \\
HSL-fakultetet \\
UiT Norgga árktalaš universitehta \and 
more authors}

\date{\today}


\begin{document}

\maketitle

\begin{abstract}
Finnish machine translation is very advanced these days.
\end{abstract}

\section{Introduction}
\label{sec:introduction}

\todo[inline]{NB that it's only 10 pages}
Finnish is a very language to translate automatically.

The morphological complexity is higher than any IE language really.

The different approaches to solving morphological complexity and its mismatch between languages.

We use following tools: moses, omorfi, morfessor, flatcat, apertium, GF.
We use the following techniques: PBSMT, morph segmentation, RBMT, GF stuff, system comb.
Languages: English, Swedish, North Sámi.
Experiments both ways. 

This article is organised as follows:

% something about translation directions

%% into Finnish / out of Finnish

% related stuffs like translating between finnishes :D

% something table/matrix about pairs and systems

\section{Resources}

Finnish is a moderately well-resourced language~\cite{whitepaper}.
In this section we describe the corpora, the \textit{natural language processing} (NLP) systems and the MT toolkits available for free and open source machine translation work for Finnish.

\subsection{Parallel corpora}

We use: WMT 2015 data (incl. europarl, newsfoo), fiwac and fienwac, 

\subsection{Machine translation systems}

\begin{table}

\begin{tabular}{lccc}
\hline
\textbf{Language} & \textbf{to Finnish} & \textbf{from Finnish} & \\
English &  ..          &              & \\
Swedish &            &              & \\
Croatian & & & \\
\hline
\end{tabular}
\end{table}


\section{Methods}
\label{sec:methods}

\subsection{Statistical MT methods}

\subsubsection{Moses}

\subsubsection{Commercial Online Translators}

We use online translators from commercial companies as a reference point to state-of-the-art in industry:
Google translate,
Bing,
Sunda,
Yandex.

\subsection{Rule-based MT methods}

\subsubsection{Apertium}

\subsubsection{Grammatical Framework}

\subsection{Language Modeling and Pre-processing}

\subsubsection{Morfessor}

\subsubsection{Omorfi}

\subsection{System combination}


\section{Experimental setup}
\label{sec:experimental-setup}

\section{Evaluation}
\label{sec:evaluation}

\section{Discussion}
\label{sec:discussion}

\section{Conclusion}
\label{sec:conclusion}

\bibliographystyle{unsrt}
\bibliography{iwclul2016}


\end{document}