%%%%%%%%%%%%%%%%%%%%%%%%%%%%%%%%%%%%%%%%%
% Beamer Presentation
% LaTeX Template
% Version 1.0 (10/11/12)
%
% This template has been downloaded from:
% http://www.LaTeXTemplates.com
%
% License:
% CC BY-NC-SA 3.0 (http://creativecommons.org/licenses/by-nc-sa/3.0/)
%
%%%%%%%%%%%%%%%%%%%%%%%%%%%%%%%%%%%%%%%%%

%----------------------------------------------------------------------------------------
%	PACKAGES AND THEMES
%----------------------------------------------------------------------------------------

\documentclass{beamer}
\usepackage{tikz}
\usepackage{graphicx} % Allows including images
\usepackage{booktabs} % Allows the use of \toprule, \midrule and \bottomrule in tables
\usepackage{xcolor}
\usepackage{rotating}
\usepackage{tikz-dependency}
\usepackage{gb4e}

\definecolor{cerulean}{rgb}{0.0, 0.48, 0.65}

\mode<presentation> {

\usetheme{Szeged}
\usecolortheme{whale}

%\useoutertheme{split}
%\useinnertheme{circles}
%\usebackgroundtemplate{
%\begin{sideways}
%\begin{minipage}{0.7\paperwidth}
%\vspace{5pt}
%magyar
%\tiny \flushleft \textcolor{blue}{SZEGEDI TUDOM\'ANYEGYETEM} \\
%\textcolor{cerulean}{Nyelvtechnológiai Csoport}
%angol
%\tiny \flushleft \textcolor{blue}{UNIVERSITY OF SZEGED} \\
%\textcolor{cerulean}{Natural Language Processing Group}
%\end{minipage}
%\end{sideways}
%\begin{minipage}[b][0.3\paperheight][b]{0.95\paperwidth}
%\flushright
%\tikz\node[opacity=1.0]{\includegraphics[width=0.1\paperwidth,height=0.1\paperheight]{szte_logo_trans.png}};
%\hfill
%\end{minipage}
%
%}

}


%----------------------------------------------------------------------------------------
%	TITLE PAGE
%----------------------------------------------------------------------------------------

\title[UD for Finno-Ugric Languages]{\textbf{Universal Dependencies for Finno-Ugric Languages}} % The short title appears at the bottom of every slide, the full title is only on the title page

\author[]{Veronika Vincze\\
Filip Ginter\\
Tommi Pirinen\\
Francis Tyers} % Your name
\institute% Your institution as it will appear on the bottom of every slide, may be shorthand to save space
[]{
University of Szeged\\ 
University of Turku\\
Dublin City University\\
University of Troms\o\\% Your institution for the title page
%\medskip
\textit{vinczev@inf.u-szeged.hu, figint@utu.fi\\
tommi.pirinen@computing.dcu.ie, francis.tyers@uit.no \\
% Your email address
}}
\date{20 January 2016} % Date, can be changed to a custom date

%\titlegraphic{{\hspace{80mm}}{\vspace{70mm}}{\includegraphics[width=20mm,height=20mm]{szte_cimer}}}

%\titlegraphic{\includegraphics[width=2cm]{szte_cimer.png}\hspace*{4.75cm}~%
   %\includegraphics[width=2cm]{logo_rgai.png}}
	
\begin{document}

\begin{frame}
\titlepage % Print the title page as the first slide
\end{frame}

\begin{frame}
\frametitle{Overview}
\begin{itemize}
\item introduction \& motivation
\item Finnish and Hungarian landscape before UD
\item the Universal Dependencies project
\item principles of Universal Morphology
\item specific morphological features and values for FU languages
\begin{itemize}
	\item possessive markers
	\item object-verb agreement
\end{itemize}
\item principles of Universal Dependencies
\item specific dependency labels and annotation practice for FU languages
\begin{itemize}
	\item empty copulas
	\item multiword named entities
	\item extended labels
\end{itemize}
\item building UD treebanks
\item how to contribute
\end{itemize}
\end{frame}


\begin{frame}
\frametitle{Introduction}
\textbf{Universal Dependencies:}
\begin{itemize}
\item international project (unfunded!)
\item 33 languages (v1.2)
\item goal: to develop a ``universal'', i.e.~a language-independent morphological and syntactic representation
\item multilingual morphological and syntactic parsing
\item studies on linguistic typology and contrastive linguistics
\end{itemize}
\end{frame}

\begin{frame}
\frametitle{Motivation}
\begin{itemize}
\item part-of-speech tagging and syntactic parsing have been popular research areas
\item several shared tasks have been recently organized that aimed at the morphological and syntactic parsing of several languages \cite{seddah-EtAl:2013:SPMRL,seddah-kubler-tsarfaty:2014:SPMRL-SANCL}
\item comparison of results achieved for different languages is not straightforward
\item language-specific morphological tagsets
\item language-specific syntactic labels
\item language-specific annotation principles
\end{itemize}
\end{frame}

\begin{frame}
\frametitle{The Finnish landscape before UD}
\begin{itemize}
\item a number of different commercial analysers, including TWOL-fi, fin-fdg,
    textmorfo, unitex Finnish (1980s -- mid 2000s) and some open source alternatives
    thereafter: omorfi, suomi-malaga, Grammatical Framework
\item similarly, old corpora made with textmorfo, no manual checking, new ones
    are partially hand-annotated: FinnTreeBanks 1, 2, 3,
    Turku Dependency Treebank etc... most derive from omorfi tags, often with
    trivial replacements(?)
\end{itemize}
\end{frame}

\begin{frame}
\frametitle{The Hungarian landscape before UD}
\begin{itemize}
\item 3 morphological tagsets: HUMOR, MSD, KR
\item 1 manually annotated corpus: the Szeged Treebank (with MSD codes)
\item several morphological analyzers and POS-taggers (Humor, hunmorph, hunpos, magyarlanc, purePOS)
\item \textbf{state-of-the-art results are hard to compare} due to differences in the tagsets used 
\item 1 dependency annotation scheme
\item 1 dependency parser (magyarlanc)
\end{itemize}
\end{frame}

\begin{frame}
\frametitle{Standardized Morphology and Syntax}
Standardized tagsets have been constantly developed:
\begin{itemize}
\item MSD morphological coding system for Eastern European languages \cite{erjavec}
\item Interset as an interlingua for several morphological coding systems \cite{ZEMAN08.66}
\item multilingual tagset for part-of-speech (POS) tagging and parsing \cite{rambow:2006:LREC} \item POS tags based on data from the CoNLL-2007 Shared Task \cite{NivreT207} \cite{McDonaldCTE07}
\item 12 POS tags for 22 languages \cite{petrov-das-mcdonald:2012:LREC}
\item Stanford dependencies \cite{stanford:dep}
\item Universal Dependencies and Morphology
\end{itemize} 
\end{frame}

\begin{frame}
\frametitle{Languages in UD}
\begin{itemize}
\item 10 languages in January 2015
\item now 33 languages
\item several language families and typological classes
\item Uralic languages:
\begin{itemize}
\item Finnish
\item Hungarian
\item Estonian
\end{itemize}
\end{itemize}
\end{frame}

\begin{frame}
\frametitle{UD as tagsets and principles}
\begin{itemize}
\item a fixed set of POS tags
\item feature--value pairs: features are fixed, values can be language specific
\item lexical features are characteristics of the lemmas 
\item inflectional features are characteristics of the word forms
\item layered features
\item a fixed set of dependency labels (language specific extension possible)
\item universal annotation principles (constant development)
\end{itemize}
\end{frame}

\begin{frame}
\frametitle{POS tags}
\begin{table}
\small
\begin{tabular}{llll}
\hline\noalign{\smallskip}
POS & Description & POS & Description \\
\noalign{\smallskip}\hline\noalign{\smallskip}
ADJ & adjective & PART & particle\\
ADV & adverb & PRON & nominal pronoun\\
ADP & adposition & PROPN & proper noun\\
AUX & auxiliary & PUNCT & punctuation\\
CONJ & coordinating conjunction & SCONJ & subordinating conjunction\\
DET & determiner & SYM & symbol\\
INTJ & interjection & VERB & verb\\
NOUN & noun & X & other\\
NUM & number & & \\
\noalign{\smallskip}\hline
\end{tabular}
\end{table}
\end{frame}

\begin{frame}
\frametitle{Agreement with number and person of the possessor on the noun}
\gll h\'azaim\'enak\\
house-1SGPOSS-PL-POSSD.SG-DAT\\
\trans to that of my houses\\
Analysis:\\
NOUN \\ Case=Dat$\vert$Number=Plur$\vert$Number[psed]=Sing$\vert$\\Number[psor]=Sing$\vert$Person[psor]=1\\

\textbf{Layered features} (Number[psed], Number[psor], Person[psor])
\end{frame}

\begin{frame}
\frametitle{Object-verb agreement}
\gll L\'atom Pistit .\\
see-1SGOBJ Steve-ACC .\\
\trans I can see Steve.
\gll L\'atok egy gyereket az udvaron .\\
see-1SGSUBJ a kid-ACC the yard-SUP .\\
\trans I can see a kid in the yard.
\gll L\'atlak .\\
see-1SGOBJ2 .\\
\trans I can see you.\\
\textbf{Definiteness marked on the verb}\\
\textbf{New feature value} for 2nd person objects
\end{frame}

\begin{frame}
\frametitle{Universal Dependencies}
\begin{table}
%TODO: dep labels
%\caption{Dependency labels.}
\tiny
\label{tab:dep}       
\begin{tabular}{llll}
\hline\noalign{\smallskip}
Label & Description & Label & Description \\
\noalign{\smallskip}\hline\noalign{\smallskip}
acl & relative clause & expl & expletive\\
advcl & adverbial clause & foreign & foreign words\\
advmod & adverbial modifier & goeswith & typo\\
amod & adjectival modifier & iobj & indirect object\\
appos & apposition & list & list\\
aux & auxiliary & mark & subordinating conjunction\\
auxpass & passive auxiliary & mwe & multiword expression\\
case & adposition & name & multiword named entity\\
cc & coordinating conjunction & neg & negation word\\
ccomp & clausal complement & nmod & nominal modifier\\
compound & compound & nsubj & subject\\
conj & coordinated word  & nsubjpass & passive subject\\
cop & copula & nummod & numeric modifier\\
csubj & clausal subject & parataxis & parataxis\\
csubjpass & clausal passive subject &  punct & punct\\
dep & dependency & remnant & argument in ellipted sentences\\
det & determiner &  reparandum & repairs\\
discourse & discourse element & root & root\\
dislocated & dislocated element & vocative & vocative \\
dobj & object & xcomp & open clausal complement\\
\noalign{\smallskip}\hline
\end{tabular}
\end{table}
\end{frame}

\begin{frame}
\frametitle{Empty copulas}
\begin{columns}
\column{.5\textwidth}
\textbf{Present tense, indicative mood, first person singular:
}\label{sg1}
\gll \'En tan\'ar vagyok .\\
I teacher be-1SG .\\
\trans I am a teacher.\\

\textbf{Present tense, indicative mood, third person singular:
}
\gll \H{O} tan\'ar .\\
he teacher .\\
\trans He is a teacher.\\

\column{.5\textwidth}
\textbf{Past tense, indicative mood, third person singular:
}
\gll \H{O} tan\'ar volt .\\
he teacher be-PAST-3SG.\\
\trans He was a teacher.\\

\textbf{Present tense, imperative mood, third person singular:
}
\label{imp}
\gll \H{O} legyen tan\'ar !\\
he be-IMP-3SG teacher !\\
\trans He should be a teacher .
\end{columns}
\end{frame}


\begin{frame}
\frametitle{Function head vs. content head}

\begin{columns}
\column{0.5\textwidth}
Original analysis in the Szeged Dependency Treebank:

\begin{figure}%[ht!]
	\centering
\begin{dependency}[theme = simple]
   \begin{deptext}[column sep=0.9em]
      E \& gondolat \& sem \& VAN \& \'{u}j \& . \\
   \end{deptext}
   \deproot{4}{ROOT}
   \depedge{2}{1}{DET}
   \depedge{4}{2}{SUBJ}
	 \depedge{4}{3}{NEG}
	 \depedge{4}{5}{PRED}
	 \depedge{4}{6}{PUNCT}
\end{dependency}
	%\caption{A function head analysis in the Szeged Dependency Treebank.}
		\label{kopuladep}
\end{figure}

\column{0.5\textwidth}
UD analysis:

\begin{figure}%[ht!]
	\centering
\begin{dependency}[theme = simple]
   \begin{deptext}[column sep=0.9em]
      E \& gondolat \& sem \& \'{u}j \& .\\
   \end{deptext}
   \deproot{4}{root}
   \depedge{2}{1}{det}
   \depedge{4}{2}{nsubj}
	 \depedge{4}{3}{neg}
	 \depedge{4}{5}{punct}
\end{dependency}
	%\caption{A content head analysis in the Hungarian UD treebank.}
	\label{kopulaud}
\end{figure}
\end{columns}

\textit{E gondolat sem \'uj.} (this thought not new) ``This thought is not novel at all.''\\
\end{frame}

\begin{frame}
\frametitle{Multiword named entities}
\begin{itemize}
\item the head of multiword units is the first element by convention
\item the last element might be inflected in FU languages:\\
\emph{Joakim Nivre} -- \emph{Joakim Nivr\'enek} ``for Joakim Nivre''
\item language-specific solutions
\end{itemize}

\begin{dependency}[theme = simple]
   \begin{deptext}[column sep=1em]
      Joakim \& Nivre \\
   \end{deptext}
   \deproot{2}{root}
   \depedge{2}{1}{name}
\end{dependency}
\end{frame}

\begin{frame}
\frametitle{Language-specific extensions}
\begin{itemize}
\item dependency labels extended
\item language-specific differentiation of subtypes of a relation (e.g.~\texttt{ccomp:dobj}, \texttt{ccomp:iobj})
\item in Hungarian, light verb constructions are also marked in this way
\item \emph{A bizotts\'ag \textbf{d\"ont\'est hozott} az \"ul\'esen.} ``The committee made a decision at the meeting.''

\end{itemize}

\begin{dependency}[theme = simple]
   \begin{deptext}[column sep=1em]
      A \& bizotts\'ag \& d\"ont\'est \& hozott \& az \& \"ul\'esen \& . \\
   \end{deptext}
   \deproot{4}{root}
   \depedge{4}{2}{nsubj}
   \depedge{2}{1}{det}
   \depedge{4}{7}{punct}
   \depedge{4}{3}{dobj:lvc}
   \depedge{4}{6}{nmod:obl}
   \depedge{6}{5}{det}
\end{dependency}

\end{frame}

\begin{frame}
\frametitle{Building UD treebanks}
\begin{itemize}
\item from scratch (languages with no treebanks, FU minority languages)
\item automatic conversion of existing treebanks (Finnish)
\item conversion of existing treebanks + manual correction (Hungarian)
\item various sizes (1K-90K sentences)
\item universal dependency annotation is necessary
\item universal morphology is optional (but welcome)
\item documentation on GitHub
\item files on LINDAT
\end{itemize}
\end{frame}

\begin{frame}
\frametitle{Finno-Ugric UD treebanks}
\begin{table}
\centering
\footnotesize
	\begin{tabular}{lccc}
	& \textbf{Token} & \textbf{Lemma/feature/secondary dep} & \textbf{Creation}\\\hline
	Finnish & 181K & LFD & automatic + manual\\
	Finnish-FTB & 159K & LF & automatic\\
	Hungarian & 26K & LF & (automatic +) manual\\
	Estonian & 9K & LF & automatic\\
	\end{tabular}
%	\caption{FU}
	\label{tab:FU}
\end{table}
\end{frame}

\begin{frame}
\frametitle{Technical issues}
\begin{itemize}
\item file format: CoNLL norms
\item one token - one line
\item grammatical information on tokens in columns
\end{itemize}

\tiny
1	1947.	1947.	ADJ	\_	Case=Nom$\vert$Number=Sing$\vert$Number[psed]=None$\vert$Number[psor]=None$\vert$NumType=Ord$\vert$Person[psor]=None	2	nmod:att	\_	\_ \\
2	december	december	NOUN	\_	Case=Nom$\vert$Number=Sing$\vert$Number[psed]=None$\vert$Number[psor]=None$\vert$Person[psor]=None	3	nmod:att	\_	\_ \\
3	6-án	6.	NOUN	\_	Case=Sup$\vert$Number=Sing$\vert$Number[psed]=None$\vert$Number[psor]=Sing$\vert$Person[psor]=3	6	nmod:obl	\_	\_ \\
4	Tito	Tito	PROPN	\_	Case=Nom$\vert$Number=Sing$\vert$Number[psed]=None$\vert$Number[psor]=None$\vert$Person[psor]=None	6	nsubj	\_	\_ \\
5	Budapestre	Budapest	PROPN	\_	Case=Sub$\vert$Number=Sing$\vert$Number[psed]=None$\vert$Number[psor]=None$\vert$Person[psor]=None	6	nmod:obl	\_	\_ \\
6	l\'atogatott	l\'atogat	VERB	\_	Definite=Ind$\vert$Mood=Ind$\vert$Number=Sing$\vert$Person=3$\vert$Tense=Past$\vert$VerbForm=Fin$\vert$Voice=Act	0	root	\_	\_ \\
7	.	.	PUNCT	\_	\_	6	punct	\_	\_ \\

\end{frame}

\begin{frame}
\frametitle{A Finnish example}
\footnotesize
1	Saavuin	saapua	VERB	V	Mood=Ind$\vert$Number=Sing$\vert$Person=1$\vert$Tense=Past$\vert$VerbForm=Fin$\vert$Voice=Act	0	root	\_	\_ \\
2	siis	siis	ADV	Adv	\_	1	advmod	\_	\_ \\
3	kaupunkiin	kaupunki	NOUN	N	Case=Ill$\vert$Number=Sing	1	nmod	\_	\_ \\
4	kameraryhm\"ani	kamera\#ryhm\"a	NOUN	N	Case=Gen$\vert$Number=Sing$\vert$Number[psor]=Sing$\vert$Person[psor]=1	1	nmod	\_	\_ \\
5	kanssa	kanssa	ADP	Adp	AdpType=Post	4	case	\_	\_ \\
6	jo	jo	ADV	Adv	\_	8	advmod	\_	\_ \\
7	ennen	ennen	ADP	Adp	AdpType=Prep	8	case	\_	\_ \\
8	yht\"atoista	yksi\#toista	NUM	Num	Case=Par$\vert$Number=Sing$\vert$NumType=Card	1	nmod	\_	\_ \\
9	.	.	PUNCT	Punct	\_	1	punct	\_	\_ \\
\end{frame}

\begin{frame}
\frametitle{Crowdsourcing at CICLING 2015}
\begin{itemize}
\item Joakim Nivre's presentation on UD
\item experiment with 20 sentences containing various syntactic phenomena
\item sentences were translated to several languages
\item participants had to provide UD annotation for those
\item \url{http://weaver.nlplab.org/ud/}
\end{itemize}
\end{frame}

\begin{frame}
\frametitle{How to contribute}
Main steps:
\begin{itemize}
\item adapt the tagsets and principles to your language
\item prepare a treebank with UD annotation
\item prepare documentation
\item upload data and documentation on GitHub
\item validate your data
\end{itemize}
For more information, contact Filip and Sampo.
\end{frame}

\begin{frame}
\frametitle{Summary}
\begin{itemize}
\item main principles behind Universal Morphology and Dependencies presented
\item specific language phenomena relevant in Finno-Ugric languages discussed
\item how to contribute to UD
\end{itemize}

Check this out:
\url{http://universaldependencies.github.io/docs/}
\end{frame}

\begin{frame}[allowframebreaks]
\tiny
\bibliographystyle{plain}
\bibliography{eacl}
\end{frame}

\end{document} 
