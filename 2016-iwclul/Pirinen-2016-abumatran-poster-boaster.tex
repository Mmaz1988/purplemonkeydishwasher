\documentclass{beamer}


\usepackage{amssymb}
\usepackage{amsmath}
\usepackage{amsfonts}

\usepackage{fontspec}
\usepackage{polyglossia}

\usepackage{graphicx}
\usepackage{color}
\usepackage{url}
\usepackage{textpos}
\usepackage{xspace}
\usepackage{array}

\mode<presentation>
{
  \usetheme{Abumatranpres}
}


\graphicspath{{./fig/}}


\title{Statistically Translating to Finnish Morph Segments\\
\scriptsize{in Szeged IWCLUL workshop}}
\author{Tommi A Pirinen \scriptsize \guilsinglleft{}tommi.pirinen@computing.dcu.ie\guilsinglright{}}
\institute{Ollscoil Chathair Bhaile Átha Cliath, ADAPT Centre\\
EU Marie Curie Abu-MaTran project}
\date{\today}

\begin{document}

\selectlanguage{english}

\maketitle

\begin{frame}
    \frametitle{What is it?}
    \begin{itemize}
        \item An experiment as a side project of using morphology in statistical
            machine translation
        \item A follow-up on annual WMT shared task competition
        \item Contains collective SMT work from a group of people
        \item A particularly extensive error analysis including humans looking
            at the translations and thinking about stuff
    \end{itemize}
\end{frame}


\begin{frame}
    \frametitle{What's the results, recommendations, future world?}
    \begin{itemize}
        \item A well-polished rule-based morphology may be better than
            unsupervised statistical one
        \item systems often get better when you combine more of them
        \item quality of to-Finnish translations is quite atrocious
        \item rule-based linguistic analysis of quality with $F_1$ measure
            was inconclusive?
        \item ...stay tuned for WMT 2016!
    \end{itemize}
\end{frame}

\begin{frame}
    \frametitle{Poster looks like this:}
    \begin{minipage}[0.2\textheight]{\textwidth}
\begin{columns}[T]
\begin{column}{0.8\textwidth}
    You can get all datas (forthcoming) and scripts \url{https://github.com/flammie/autostuff-moses-smt/}.
\end{column}
\begin{column}{0.2\textwidth}
    \includegraphics[height=0.8\textheight]{poster}
\end{column}
\end{columns}
\end{minipage}
\end{frame}

\end{document}
% vim: set spell:
