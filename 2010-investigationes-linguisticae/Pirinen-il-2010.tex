\documentclass[a4paper,runningheads]{llncs}

\usepackage{hyperref}
\usepackage{amssymb}
\usepackage{fontenc}
\setcounter{tocdepth}{3}
\usepackage{url}
\urldef{\mailsa}\path|{tommi.pirinen, krister.linden}@helsinki.fi|

\newcommand{\keywords}[1]{\par\addvspace\baselineskip
\noindent\keywordname\enspace\ignorespaces#1}


\usepackage{algorithm}
\usepackage{algorithmic}[2]
\usepackage{listings}

\lstset{stepnumber=2,numbers=left,extendedchars=true,escapeinside={|-}{-|},gobble=4}

\begin{document}

\mainmatter

\title{Creating and Weighting Hunspell Dictionaries as Finite-State Automata*
    }

\titlerunning{Weighted Hunspell Finite-State Automata}

\author{Tommi A Pirinen \and Krister Lind\'{e}n}

\institute{University of Helsinki, Department of Modern Languages\\
    Unionkatu 40, FI-00014 University of Helsinki, Finland\\
    \mailsa\\
    \url{http://www.helsinki.fi/modernlanguages}}


%\toctitle{Creating and Weighting Hunspell Dictionaries as Finite-State Automata}
%\tocauthor{Something something?}
\maketitle

\begin{abstract}
  There are numerous formats for writing spell-checkers for
  open-source systems and there are many descriptions for languages
  written in these formats. In this paper we
  demonstrate a method for converting these spell-checking lexicons
  into finite-state automata, and present a simple way to apply unigram
  corpus training over the spell-checking suggestion mechanisms using
  weighted finite-state tecnology.\footnote{This is author's pre-print version with hyperref turned on among other
    changes, it may differ from the printed version found in Investigationes
Linguistic\ae{}.}
\end{abstract}


\section{Introduction}
\label{sec:introduction}

Currently there is a wide range of different free open-source
solutions for spell-checking by computer. The most popular of the spelling
dictionaries are the various instances of *spell software, i.e.
ispell\footnote{\url{http://www.lasr.cs.ucla.edu/geoff/ispell.html}},
aspell\footnote{\url{http://aspell.net}}, myspell and
hunspell\footnote{\url{http://hunspell.sf.net}} and other *spell
derivatives. The hunspell dictionaries
provided with the OpenOffice.org suite cover 98 languages.

The program-based spell-checking methods have their limitations
because they are based on specific program code that is extensible
only by coding new features into the system and getting all users to
upgrade. E.g. hunspell has limitations on what affix morphemes you can
attach to word roots with the consequence that not all languages with
rich inflectional morphologies can be conveniently implemented in
hunspell. This has already resulted in multiple new pieces of software
for a few languages with implementations to work around the
limitations, e.g.  emberek (Turkish), hspell (Hebrew), uspell
(Yiddish) and voikko (Finnish). What we propose is to use a generic
framework of finite-state automata for these tasks. With finite-state
automata it is possible to implement the spell-checking functionality
as a one-tape weighted automaton containing the language model and a
two-tape weighted automaton containing the error model.

We also extend the hunspell spell checking system by using simple corpus-based
unigram probability training \cite{conf/lrec/Pirinen2010}. With this probability
trained lexicon it is possible to fine-tune the spelling error suggestions 

With this model, extensions to
context-based n-gram models for real-word spelling error problems
\cite{DBLP:conf/cicling/Wilcox-OHearnHB08} are also possible.

We also provide a method for integrating the finite-state spell-checking
and hyphenation into applications using an open-source spell-checking
library voikko\footnote{\url{http://voikko.sf.net}}, which provides a
connection to typical open-source software, such as Mozilla Firefox,
OpenOffice.org and the Gnome desktop via enchant.

\section{Definitions}
\label{sec:definitions}

In this article we use weighted two-tape finite-state automata---or
weighted finite-state transducers---for all processing. We use the
following symbol conventions to denote the parts of a weighted
finite-state automaton: a transducer $T = (\Sigma, \Gamma, Q, q_0,
Q_f, \delta, \rho)$ with a semi-ring $(S, \oplus, \otimes,
\overline{0}, \overline{1})$ for weights. Here $\Sigma$ is a set with
the input tape alphabet, $\Gamma$ is a set with the output tape
alphabet, $Q$ a finite set of states in the transducer, $q_0 \in Q$ is
an initial state of the transducer, $Q_f \subset Q$ is a set of finite
states, $\delta: Q \times \Sigma \times \Gamma \times S \rightarrow Q$
is a transition relation, $\rho: Q_f \rightarrow S$ is a final weight
function. A successful path is a list of transitions from an initial
state to a final state with a weight different from $\overline{0}$
collected from the transition function and the final state function in
the semi-ring $S$ by the operation $\otimes$. We typically denote a
successful path as a concatenation of input symbols, a colon and a
concatenation of output symbols. The weight of the successful path is
indicated as a subscript in angle brackets,
$\textit{input:output}_{<w>}$. A path transducer is denoted by
subscripting a transducer with the path.  If the input and output
symbols are the same, the colon and the output part can be omitted.

The finite-state formulation we use in this article is based on Xerox
formalisms for finite-state methods in natural language processing
\cite{beesley/2003}, in practice lexc is a formalism for writing right linear
grammars using morpheme sets called lexicons. Each morpheme in a lexc grammar
can define their right follower lexicon, creating a finite-state network called
a \emph{lexical transducer}. In formulae, we denote a lexc style lexicon named $X$ as
$Lex_X$ and use the shorthand notation $Lex_X \cup \textit{input:output Y}$ to
denote the addition of a lexc string or morpheme, \texttt{input:output Y ;} to
the \texttt{LEXICON X}.  In the same framework, the twolc formalism is used to
describe context restrictions for symbols and their realizations in the form of
parallel rules as defined in the appendix of \cite{beesley/2003}. We use
$Twol_Z$ to denote the rule set $Z$ and use the shorthand notation $Twol_Z \cap
\textit{a:b} \leftrightarrow \textit{l e f t} \_ \textit{r i g h t}$ to denote
the addition of a rule string \texttt{a:b <=> l e f t \_ r i g h t ;} to the
rule set $Z$, effectively saying that \textit{a:b} only applies in
the specified context.

A spell-checking dictionary is essentially a single-tape finite-state
automaton or a language model $T_L$, where the alphabet $\Sigma_L =
\Gamma_L$ are characters of a natural language. The successful paths
define the correctly spelled word-forms of the language
\cite{conf/lrec/Pirinen2010}. 

For weighted spell-checking, we define the weights in lexicon as
as probability of the word in Wikipedia. For weight model of the automaton
we use the tropical semi-ring assigning each word-form the weight of
$-\log \frac{f_w}{CS}$, where $f_w$ is the frequency of the word and $CS$
the corpus size in number of word form tokens. For word-forms not appearing
in Wikipedia, we assign small probability by formula $-\log \frac{1}{CS+1}$.

A spelling correction model or an error model $T_E$ is a two-tape
automaton mapping the input text strings of the text to be
spell-checked into strings that may be in the language model. The
input alphabet $\Sigma_E$ is the alphabet of the text to be
spell-checked and the output alphabet is $\Gamma_E = \Sigma_L$. For
practical applications, the input alphabet needs to be extended by a
special any symbol with the semantics of a character not belonging to
the alphabet of the language model in order to account for input text
containing typos outside the target natural language alphabet. The
error model can be composed with the language model, $T_L \circ T_E$,
to obtain an error model that only produces strings of the target
language. For space efficiency, the composition may be carried out
during run-time using the input string to limit the search space. The
weights of an error model may be used as an estimate for the likelihood of the
combination of errors. The error model is applied as a filter between
the path automaton $T_s$ compiled from the erroneous string, $s \notin
T_L$, and the language model, $T_L$, using two compositions, $T_s
\circ T_E \circ T_L$. The resulting transducer consists of a
potentially infinite set of paths relating an incorrect string with
correct strings from $L$. The paths, $s:s^i_{<w_i>}$, are weighted by
the error model and language model using the semi-ring multiplication
operation, $\otimes$. If the error model and the language model
generate an infinite number of suggestions, the best suggestions may
be efficiently enumerated with some variant of the n-best-paths
algorithm \cite{mohri/2002}. For automatic spelling corrections, the
best path may be used. If either the error model or the language model
is known to generate only a finite set of results, the suggestion
generation algorithm may be further optimized.

\section{Material}
\label{sec:material}

In this article we present methods for converting the hunspell 
dictionaries and rule sets for use with open-source finite-state
writer's tools.  As concrete dictionaries we use the repositories of
free implementations of these dictionaries and rule sets found on the
internet, e.g. for the hunspell dictionary files found on the
OpenOffice.org spell-checking
site\footnote{\url{http://wiki.services.openoffice.org/wiki/Dictionaries}}.

In this section we describe the parts of the file formats we are
working with. All of the information of the hunspell format specifics
is derived from the
\texttt{hunspell(4)}\footnote{\url{http://manpages.ubuntu.com/manpages/dapper/man4/hunspell.4.html}}
man page, as that is the only normative documentation of hunspell we
have been able to locate. 

The corpora for spell-checking dictionaries' unigram training used, are
wikipedia's database backups\footnote{\url{http://download.wikimedia.org}}. The
wikipedia is available in majority of languages, consisting large amount of
language that is typically well-suited for training a spell-checking dictionary.


\subsection{Hunspell File Format}
\label{subsec:material-hunspell}

A hunspell spell-checking dictionary consists of two files: a
dictionary file and an affix file. The dictionary file contains only
root forms of words with information about morphological affix classes
to combine with the roots.  The affix file contains lists of affixes
along with their context restrictions and effects, but the affix file
also serves as a settings file for the dictionary, containing all
meta-data and settings as well.

The dictionary file starts with a number that is intended to be the
number of lines of root forms in the dictionary file, but in practice
many of the files have numbers different from the actual line count,
so it is safer to just treat it as a rough estimate. Following the
initial line is a list of strings containing the root forms of the
words in the morphology. Each word may be associated with an arbitrary
number of classes separated by a slash. The classes are encoded in one
of the three formats shown in the examples of
Figure~\ref{fig:hunspell-dic-examples}: a list of binary octets
specifying classes from 1--255 (minus octets for CR, LF etc.), as in
the Swedish example on lines
\ref{vrb:hunspelldicsvstart}--\ref{vrb:hunspelldicsvend}, a list of
binary words, specifying classes from 1--65,535 (again ignoring octets
with CR and LF) or a comma separated list of numbers written in digits
specifying classes 1--65,535 as in the North Sámi examples on lines
\ref{vrb:hunspelldicsestart}--\ref{vrb:hunspelldicseend}. We refer to
all of these as continuation classes encoded by their numeric decimal
values, e.g. 'abakus' on line \ref{vrb:hunspelldicsvstart} would have
continuation classes 72, 68 and 89 (the decimal values of the ASCII
code points for H, D and Y respectively). In the Hungarian example,
you can see the affix compression scheme, which refers to the line
numbers in the affix file containing the continuation class listings,
i.e. the part following the slash character in the previous two
examples. The lines of the Hungarian dictionary also contain some
extra numeric values separated by a tab which refer to the morphology
compression scheme that is also mentioned in the affix definition
file; this is used in the hunmorph morphological analyzer
functionality which is not implemented nor described in this paper.

\begin{figure}[tbp]
  \centering
  \begin{lstlisting}
    # Swedish
    abakus/HDY|-\label{vrb:hunspelldicsvstart}-|
    abalienation/AHDvY
    abalienera/MY|-\label{vrb:hunspelldicsvend}-|
    # Northern S|-\'{a}-|mi
    okta/1|-\label{vrb:hunspelldicsestart}-|
    guokte/1,3
    golbma/1,3|-\label{vrb:hunspelldicseend}-|
    # Hungarian
    |-\"{u}-|z|-\'{e}-|r/1  1|-\label{vrb:hunspelldichustart}-|
    |-\"{u}-|zlet|-\'{a}-|g/2       2
    |-\"{u}-|zletvezet|-\"{o}-|/3   1
    |-\"{u}-|zletszerz|-\"{o}-|/4   1|-\label{verb:hunspelldichuend}-|
  \end{lstlisting}
  \caption{Excerpts of Swedish, Northern S|-\'{a}-|mi and Hungarian dictionaries}
  \label{fig:hunspell-dic-examples}
\end{figure}

The second file in the hunspell dictionaries is the affix file,
containing all the settings for the dictionary, and all non-root
morphemes. The Figure~\ref{fig:hunspell-aff-examples} shows parts of
the Hungarian affix file that we use for describing different setting
types.  The settings are typically given on a single line composed of
the setting name in capitals, a space and the setting values, like the
NAME setting on line~\ref{vrb:hunspellaffname}. The hunspell files
have some values encoded in UTF-8, some in the ISO 8859 encoding, and
some using both binary and ASCII data at the same time. Note that in the examples in this article, we have
transcribed everything into UTF-8 format or the nearest relevant encoded
character with a displayable code point.

The settings we have used for building the spell-checking automata can
be roughly divided into the following four categories: meta-data, error
correction models, special continuation classes, and the actual
affixes. An excerpt of the parts that we use in the Hungarian affix file
is given in Figure~\ref{fig:hunspell-aff-examples}.

The meta-data section contains, e.g., the name of the dictionary on
line~\ref{vrb:hunspellaffname}, the character set encoding on
line~\ref{vrb:hunspellaffset}, and the type of parsing used for
continuation classes, which is omitted from the Hungarian lexicon
indicating 8-bit binary parsing.

The error model settings each contain a small part of the actual error
model, such as the characters to be used for edit distance, their weights,
confusion sets and phonetic confusion sets. The list of word
characters in order of popularity, as seen on
line~\ref{vrb:hunspellafftry} of
Figure~\ref{fig:hunspell-aff-examples}, is used for the edit distance
model. The keyboard layout, i.e. neighboring key sets, is specified
for the substitution error model on
line~\ref{vrb:hunspellaffkey}. Each set of the characters, separated by
vertical bars, is regarded as a possible slip-of-the-finger typing
error. The ordered confusion set of possible spelling error pairs is
given on
lines~\ref{vrb:hunspellaffrepstart}--\ref{vrb:hunspellaffrepend},
where each line is a pair of a `mistyped' and a `corrected' word
separated by whitespace.

The compounding model is defined by special continuation classes,
i.e. some of the continuation classes in the dictionary or affix file
may not lead to affixes, but are defined in the compounding section of
the settings in the affix file. In
Figure~\ref{fig:hunspell-aff-examples}, the compounding rules are
specified on
lines~\ref{vrb:hunspellaffcompoundstart}--\ref{vrb:hunspellaffcompoundend}. The
flags in these settings are the same as in the affix definitions, so
the words in class 118 (corresponding to lower case v) would be
eligible as compound initial words, the words with class 120 (lower
case x) occur at the end of a compound, and words with 117 only occur
within a compound. Similarly, special flags are given to word forms
needing affixes that are used only for spell checking but not for the
suggestion mechanism, etc.

The actual affixes are defined in three different parts of the file:
the compression scheme part on the
lines~\ref{vrb:hunspellaffafstart}--\ref{vrb:hunspellaffafend}, the
suffix definitions on the
lines~\ref{vrb:hunspellaffsfxstart}--\ref{vrb:hunspellaffsfxend}, and
the prefix definitions on the
lines~\ref{vrb:hunspellaffpfxstart}--\ref{vrb:hunspellaffpfxend}.

The compression scheme is a grouping of frequently co-occurring
continuation classes. This is done by having the first AF line list a
set of continuation classes which are referred to as the continuation
class 1 in the dictionary, the second line is referred to the
continuation class 2, and so forth. This means that for example
continuation class 1 in the Hungarian dictionary refers to the classes
on line~\ref{vrb:hunspellaffafone} starting from 86 (V) and ending
with 108 (l).

The prefix and suffix definitions use the same structure. The prefixes
define the left-hand side context and deletions of a dictionary entry
whereas the suffixes deal with the right-hand side. The first line of
an affix set contains the class name, a boolean value defining whether
the affix participates in the prefix-suffix combinatorics and the
count of the number of morphemes in the continuation class, e.g. the
line \ref{vrb:hunspellaffpfxstart} defines the prefix continuation
class attaching to morphemes of class 114 (r) and it combines with
other affixes as defined by the Y instead of N in the third field. The
following lines describe the prefix morphemes as triplets of removal,
addition and context descriptions, e.g., the line
\ref{vrb:hunspellaffsfxone} defines removal of 'ö', addition of
'\H{o}s' with continuation classes from AF line 1108, in case the
previous morpheme ends in 'ö'. The context description may also
contain bracketed expressions for character classes or a fullstop
indicating any character (i.e. a wild-card) as in the POSIX regular
expressions, e.g. the context description on line
\ref{vrb:hunspellaffsfxend} matches any Hungarian vowel except a, e or
ö, and the \ref{vrb:hunspellaffpfxend} matches any context.  The
deletion and addition parts may also consist of a sole `0' meaning a
zero-length string. As can be seen in the Hungarian example, the lines
may also contain an additional number at the end which is used for the
morphological analyzer functionalities.

\begin{figure}[tbp]
  \centering
  \begin{lstlisting}
    AF 1263 |-\label{vrb:hunspellaffafstart}-|
    AF V|-\"{E}-|-jxLn|-\'{O}-||-\'{e}-||-\`{e}-|3|-\"{A}-||-\"{a}-|TtYc,4l # 1 |-\label{vrb:hunspellaffafone}-|
    AF Um|-\"{O}-|yiYc|-\c{C}-| # 2
    AF |-\"{O}-|CWR|-\'{I}-|-j|-ð-||-\'{O}-||-\'{i}-|y|-\'{E}-||-\'{A}-||-\"{y}-|Yc2 # 3 |-\label{vrb:hunspellaffafend}-|

    NAME Magyar Ispell helyes|-\'{i}-|r|-\'{a}-|si sz|-\'{o}-|t|-\'{a}-|r     |-\label{vrb:hunspellaffname}-|
    LANG hu_HU     |-\label{vrb:hunspellafflang}-|
    SET UTF-8     |-\label{vrb:hunspellaffset}-|
    KEY |-\"{o}-||-\"{u}-||-\'{o}-||qwertzuiop|-\H{o}-||-\'{u}-|| # wrap
        asdfghjkl|-\'{e}-||-\'{a}-||-\H{u}-||-\'{i}-|yxcvbnm        |-\label{vrb:hunspellaffkey}-|
    TRY |-\'{i}-||-\'{o}-||-\'{u}-|taeslz|-\'{a}-|norhgki|-\'{e}-| # wrap
        dmy|-\H{o}-|pv|-\"{o}-|bucfj|-\"{u}-|yxwq-.|-\'{a}-|      |-\label{vrb:hunspellafftry}-|
    
    COMPOUNDBEGIN v     |-\label{vrb:hunspellaffcompoundstart}-|
    COMPOUNDEND x     
    ONLYINCOMPOUND |     |-\label{vrb:hunspellaffcompoundend}-|
    NEEDAFFIX u     
    
    REP 125        |-\label{vrb:hunspellaffrepstart}-|
    REP |-\'{i}-| i       
    REP i |-\'{i}-|       
    REP |-\'{o}-| o       
    REP oliere oli|-\'{e}-|re
    REP cc gysz       
    REP cs ts       
    REP cs ds       
    REP ccs ts       |-\label{vrb:hunspellaffrepend}-|
    # 116 more REP lines
    
    SFX ? Y 3 |-\label{vrb:hunspellaffsfxstart}-|
    SFX ? |-\"{o}-| |-\H{o}-|s/1108 |-\"{o}-| 20973 |-\label{vrb:hunspellaffsfxone}-|
    SFX ? 0 |-\"{o}-|s/1108 [^a|-\'{a}-|e|-\'{e}-|i|-\'{i}-|o|-\'{o}-||-\"{o}-||-\H{o}-|u|-\"{u}-||-\H{u}-|] 20973
    SFX ? 0 s/1108 [|-\'{a}-||-\'{e}-|i|-\'{i}-|o|-\'{o}-||-\'{u}-||-\H{o}-|u|-\'{u}-||-\"{u}-||-\H{u}-|-] 20973 |-\label{vrb:hunspellaffsfxend}-|
    
    PFX r Y 195 |-\label{vrb:hunspellaffpfxstart}-|
    PFX r 0 leg|-\'{u}-|jra/1262 . 22551
    PFX r 0 leg|-\'{u}-|jj|-\'{a}-|/1262 . 22552 |-\label{vrb:hunspellaffpfxend}-|
    # 193 more PFX r lines
  \end{lstlisting}
  \caption{Excerpts from Hungarian affix file}
  \label{fig:hunspell-aff-examples}
\end{figure}

\section{Methods}
\label{sec:methods}

This article presents methods for converting the existing spell-checking
dictionaries with error models, as well as hyphenators to finite-state
automata. As our toolkit we use the free open-source HFST
toolkit\footnote{\url{http://HFST.sf.net}}, which is a general purpose API for
finite-state automata, and a set of tools for using legacy data, such as Xerox
finite-state morphologies. For this reason this paper presents the algorithms
as formulae such that they can be readily implemented using finite-state
algebra and the basic HFST tools.

The lexc lexicon model is used by the tools for describing parts of the
morphotactics. It is a simple right-linear grammar for specifying finite-state
automata described in \cite{beesley/2003,conf/sfcm/Linden2009}. The twolc rule
formalism is used for defining context-based rules with two-level automata and
they are described in \cite{koskenniemi/1983,conf/sfcm/Linden2009}.

This section presents both a pseudo-code presentation for the conversion
algorithms, as well as excerpts of the final converted files from the material
given in Figures~\ref{fig:hunspell-dic-examples},
\ref{fig:hunspell-aff-examples} and \ref{fig:tex-hyph-example} of
Section~\ref{sec:material}.  The converter code is available in the HFST SVN
repository\footnote{\url{http://hfst.svn.sourceforge.net/viewvc/hfst/trunk/conversion-scripts/}}
, for those who wish to see the specifics of the implementation in
lex, yacc, c and python.

\subsection{Hunspell dictionary conversion}

The hunspell dictionaries are transformed into a finite-state transducer
language model by a finite-state formulation consisting of two parts: a lexicon
and one or more rule sets. The root and affix dictionaries are turned into
finite-state lexicons in the lexc formalism. The Lexc formalism models the part
of the morphotax concerning the root dictionary and the adjacent suffixes. The
rest is encoded by injecting special symbols, called flag diacritics, into the
morphemes restricting the morpheme co-occurrences by implicit rules that have
been outlined in \cite{beesley/1998}; the flag diacritics are denoted in lexc
by at-sign delimited substrings. The affix definitions in hunspell also define
deletions and context restrictions which are turned into explicit two-level
rules.

The pseudo-code for the conversion of hunspell files is provided in
Algorithm~\ref{algo:dic-aff-lex} and excerpts from the conversion of the examples
in Figures~\ref{fig:hunspell-dic-examples} and \ref{fig:hunspell-aff-examples}
can be found in Figure~\ref{fig:lexc-twolc-dictionary}. The dictionary file of
hunspell is almost identical to the lexc root lexicon, and the conversion is
straightforward. This is expressed on lines~\ref{a}--\ref{b} as simply going
through all entries and adding them to the root lexicon, as in
lines~\ref{c}---\ref{d} of the example result. The handling of
affixes is similar, with the exception of adding flag diacritics for
co-occurrence restrictions along with the morphemes. This is shown on
lines~\ref{g}---\ref{h} of the pseudo-code, and applying it will create the
lines~\ref{e}---\ref{f} of the Swedish example, which does not contain further
restrictions on suffixes.

To finalize the morpheme and compounding restrictions, the final lexicon in the
lexc description must be a lexicon checking that all prefixes with forward requirements
have their requiring flags turned off.

\begin{algorithm}[tbp]
  \caption{Extracting morphemes from hunspell dictionaries}
  \label{algo:dic-aff-lex}
  \begin{algorithmic}[2]
    \STATE $finalflags \leftarrow \epsilon$
    \FORALL{lines $morpheme/Conts$ in dic}
    \STATE $flags \leftarrow \epsilon$
    \FORALL{$cont$ in $Conts$ \label{a}}
    \STATE $flags \leftarrow flags + \textit{@C.cont@}$
    \STATE $Lex_{Conts} \leftarrow Lex_{Conts} \cup \textit{0:[<cont] cont}$
    \ENDFOR
    \STATE $Lex_{Root} \leftarrow Lex_{Root} \cup \textit{flags + morpheme Conts}$
    \ENDFOR \label{b}
    \FORALL{suffixes $lex, deletions, morpheme/Conts, context$ in aff \label{g}}
    \STATE $flags \leftarrow \epsilon$
    \FORALL{$cont$ in $Conts$}
    \STATE $flags \leftarrow flags + \textit{@C.cont@}$
    \STATE $Lex_{Conts} \leftarrow Lex_{Conts} \cup \textit{0 cont}$
    \ENDFOR
    \STATE $Lex_{lex} \leftarrow Lex_{lex} \cup \textit{flags} + [<lex] + \textit{morpheme Conts}$
    \FORALL{$del$ in $deletions$}
    \STATE $lc \leftarrow context + deletions\textrm{ before del}$
    \STATE $rc \leftarrow deletions \textrm{ after del} + [<lex] + morpheme$
    \STATE $Twol_{d} \leftarrow Twol_{d} \cap \textit{del:0} \Leftrightarrow \textit{lc \_ rc}$
    \ENDFOR
    \STATE $Twol_{m} \leftarrow Twol_{m} \cap [<lex]:0 \Leftrightarrow \textit{context \_ morpheme}$
    \ENDFOR
    \FORALL{prefixes $lex, deletions, morpheme/conts, context$ in aff}
    \STATE $flags \leftarrow \textit{@P.lex@}$
    \STATE $finalflags \leftarrow finalflags + \textit{@D.lex@}$
    \STATE $lex \rightarrow prefixes$
    \COMMENT{othewise as with suffixes, swapping left and right}
    \ENDFOR \label{h}
    \STATE $Lex_{end} \leftarrow Lex_{end} \cup \textit{finalflags \#}$
  \end{algorithmic}
\end{algorithm}

\begin{figure}[tbp]
  \centering
  \begin{lstlisting}
    LEXICON Root
        HUNSPELL_pfx ;
        HUNPELL_dic ;
    
    ! swedish lexc
    LEXICON HUNSPELL_dic |-\label{c}-|
    @C.H@@C.D@@C.Y@abakus HDY ; 
    @C.A@@C.H@@C.D@@C.v@@C.Y@abalienation 
        HUNSPELL_AHDvY ;
    @C.M@@C.Y@abalienera MY ; |-\label{d}-|
    
    LEXICON HDY 
    0:[<H]    H ;
    0:[<D]    D ;
    0:[<Y]    Y ;
    
    LEXICON H |-\label{e}-|
    er  HUNSPELL_end ;
    ers  HUNSPELL_end ;
    er  HUNSPELL_end ;
    ers  HUNSPELL_end ; |-\label{f}-|
    
    LEXICON HUNSPELL_end
    @D.H@@D.D@@D.Y@@D.A@@D.v@@D.m@ # ;

    ! swedish twolc file
    Rules
    "Suffix H allowed contexts"
    %[%<H%]: 0 <=> \ a _ e r ;
        \ a _ e r s ;
        a:0 _ e r ;
        a:0 _ e r s ;

    "a deletion contexts"
    a:0 <=> _ %[%<H%]:0 e r ;
        _ %[%<H%]: e r s ;
  \end{lstlisting}
  \caption{Converted dic and aff lexicons and rules governing the deletions}
  \label{fig:lexc-twolc-dictionary}
\end{figure}

\subsection{Hunspell Error Models}

The hunspell dictionary configuration file, i.e. the affix file, contains
several parts that need to be combined to achieve a similar error correction
model as in the hunspell lexicon.

The error model part defined in the KEY section allows for one slip of
the finger in any of the keyboard neighboring classes. This is implemented by
creating a simple homogeneously weighted crossproduct of each class, as given
on lines~\ref{vrb:pseudokeystart}--\ref{vrb:pseudokeyend} of
Algorithm~\ref{algo:try-key-rep}. For the first part of the example on
line~\ref{vrb:hunspellaffkey} of Figure~\ref{fig:hunspell-aff-examples}, this
results in the lexc lexicon on
lines~\ref{lexcerrorkeystart}--\ref{lexcerrorkeyend} in
Figure~\ref{fig:lexc-error-models}.

The error model part defined in the REP section is an arbitrarily long ordered
confusion set. This is implemented by simply encoding them as increasingly
weighted paths, as shown in lines
\ref{vrb:pseudorepstart}--\ref{vrb:pseudorepend} of the pseudo-code in
Algorithm~\ref{algo:try-key-rep}.

The TRY section such as the one on line~\ref{vrb:hunspellafftry} of
Figure~\ref{fig:hunspell-aff-examples}, defines characters to be tried as the
edit distance grows in descending order. For a more detailed formulation of a
weighted edit distance transducer, see e.g.  \cite{conf/lrec/Pirinen2010}).  We
created an edit distance model with the sum of the positions of the characters
in the TRY string as the weight, which is defined on
lines~\ref{vrb:pseudotrystart}--\ref{vrb:pseudotryend} of the pseudo-code in
Algorithm~\ref{algo:try-key-rep}. The initial part of the converted example
is displayed on lines~\ref{lexcerrortrystart}--\ref{lexcerrortryend} of
Figure~\ref{fig:lexc-error-models}.

Finally to attribute different likelihood to different parts of the error
models we use different weight magnitudes on different types of errors, and 
to allow only correctly written substrings, we restrict the result by the
root lexicon and morfotax lexicon, as given on
lines~\ref{lexcerrorrootstart}--\ref{lexcerrorretend} of
Figure~\ref{fig:lexc-error-models}.  With the weights on
lines~\ref{lexcerrorrootstart}--\ref{lexcerrorrootend}, we ensure that KEY
errors are always suggested before REP errors and REP errors before TRY errors.
Even though the error model allows only one error of any type, simulating the
original hunspell, the resulting transducer can be transformed into an error
model accepting multiple errors by a simple FST algebraic concatenative
n-closure, i.e. repetition.

\begin{algorithm}[tbp]
  \caption{Extracting patterns for hunspell error models}
  \label{algo:try-key-rep}
  \begin{algorithmic}[2]
    \FORALL{neighborsets $ns$ in KEY \label{vrb:pseudokeystart}}
    \FORALL{character $c$ in $ns$}
    \FORALL{character $d$ in $ns$ such that $c != d$}
    \STATE $Lex_{KEY} \leftarrow Lex_{KEY} \cup c:d_{<0>} \#  $
    \ENDFOR
    \ENDFOR
    \ENDFOR \label{vrb:pseudokeyend}
    \STATE $w \leftarrow 0$
    \FORALL{pairs $wrong, right$ in REP \label{vrb:pseudorepstart}}
    \STATE $w \leftarrow w + 1$
    \STATE $LEX_{REP} \leftarrow LEX_{REP} \cup wrong:right_{<w>} \#  $
    \ENDFOR \label{vrb:pseudorepend}
    \STATE $w \leftarrow 0$
    \FORALL{character $c$ in TRY \label{vrb:pseudotrystart}}
    \STATE $w \leftarrow w + 1$
    \STATE $Lex_{TRY} \leftarrow Lex_{TRY} \cup c:0_{<w>} \#  $
    \STATE $Lex_{TRY} \leftarrow Lex_{TRY} \cup 0:c_{<w>} \#  $
    \FORALL{character $d$ in TRY such that $c != d$}
    \STATE $Lex_{TRY} \leftarrow Lex_{TRY} \cup c:d_{<w>} \#  $
    \COMMENT{for swap: replace $\#$ with $cd$ and add $Lex_{cd} \cup d:c_{<0>} \#$}
    \ENDFOR
    \ENDFOR \label{vrb:pseudotryend}
  \end{algorithmic}
\end{algorithm}

\begin{figure}[tbp]
  \centering
  \begin{lstlisting}
    LEXICON HUNSPELL_error_root |-\label{lexcerrorrootstart}-|
    < ? > HUNSPELL_error_root ; 
    HUNSPELL_KEY "weight: 0" ;
    HUNSPELL_REP "weight: 100" ;
    HUNSPELL_TRY "weight: 1000" ; |-\label{lexcerrorrootend}-|
    
    LEXICON HUNSPELL_errret |-\label{lexcerrorretstart}-|
    < ? > HUNSPELL_errret ; 
    # ;  |-\label{lexcerrorretend}-|
    
    LEXICON HUNSPELL_KEY |-\label{lexcerrorkeystart}-|
    |-\"{o}-|:|-\"{u}-| HUNSPELL_errret "weight: 0" ; 
    |-\"{o}-|:|-\'{o}-| HUNSPELL_errret "weight: 0" ;
    |-\"{u}-|:|-\"{o}-| HUNSPELL_errret "weight: 0" ;
    |-\"{u}-|:|-\'{o}-| HUNSPELL_errret "weight: 0" ;
    |-\'{o}-|:|-\"{o}-| HUNSPELL_errret "weight: 0" ;
    |-\'{o}-|:|-\"{u}-| HUNSPELL_errret "weight: 0" ; 
    ! same for other parts |-\label{lexcerrorkeyend}-|
    
    LEXICON HUNSPELL_TRY |-\label{lexcerrortrystart}-|
    |-\'{i}-|:0 HUNSPELL_errret "weight: 1" ;
    0:|-\'{i}-| HUNSPELL_errret "weight: 1" ;
    |-\'{i}-|:|-\'{o}-| HUNSPELL_errret "weight: 2" ;
    |-\'{o}-|:|-\'{i}-| HUNSPELL_errret "weight: 2" ;
    |-\'{o}-|:0 HUNSPELL_errret "weight: 2" ;
    0:|-\'{o}-| HUNSPELL_errret "weight: 2" ;
    ! same for rest of the alphabet |-\label{lexcerrortryend}-|
    
    LEXICON HUNSPELL_REP |-\label{lexcerrorrepstart}-|
    |-\'{i}-|:i HUNSPELL_errret "weight: 1" ;
    i:|-\'{i}-| HUNSPELL_errret "weight: 2" ;
    |-\'{o}-|:o HUNSPELL_errret "weight: 3" ;       
    oliere:oli|-\`{e}-|re HUNSPELL_errret "weight: 4" ; 
    cc:gysz HUNSPELL_errret "weight: 5" ;       
    cs:ts HUNSPELL_errret "weight: 6" ;       
    cs:ds HUNSPELL_errret "weight: 7" ;       
    ccs:ts HUNSPELL_errret "weight: 8" ;  
    ! same for rest of REP pairs... |-\label{lexcerrorrepend}-|
    
  \end{lstlisting}
  \caption{Converted error models from aff file}
  \label{fig:lexc-error-models}
\end{figure}

\subsection{Weighting Finite-State Dictionaries with Wikipedia}

Finite-state automata can be weighted simply by using finite-state composition.
For corpus based weighting, the automata containing weighted language model
simply encodes a probability of a token appearing in the corpus \cite{conf/lrec/Pirinen2010}.
The weights are formulated as penalty values belonging to the weighted semiring
using formula of $-\log\frac{f}{CS}$ where $f$ is the frequency of token, and
$CS$ the size of corpus in tokens. For tokens not appearing in the corpus, a
maximum weight of $-\log\frac{1}{CS+1}$ is used to ensure they will be suggested
last by the error correction mechanism.

Since also the error model is weighted, the
weights need to be scaled so that combining them under semiring's addition
operation will produce reasonable results. In our experiment we have opted to
scale the weights of the error model so that the weight of making one error is
always greater than the back-off weight in the unigram weighting model. Using
this scaling ensures that the error model has precedence over the probability
data learned from the dictionary, which may only fine-tune the results in cases
where multiple choices are at the same error distance using the error model.

The tokens are extracted from the wikipedia using dictionary transducer and
tokenizing analysis algorithm\cite{garrido-alenda/2002}. This algorithm uses
the dictionary automaton to extract tokens that appear in the dictionary from
the wikipedia data. The rest of the tokens are formed from contiguous runs of
other dictionary characters which did not result in dictionary word-form. From
this set, the correct tokens are turned into weighted suffix tree automaton
using the $-\log\frac{f}{CS}$ formula for the weights, and this is unioned with
a version of original dictionary whose final weights have been set to the
maximum weight, $-\log\frac{1}{CS+1}$.

\section{Tests and Evaluation}

We have implemented the spell-checkers and their error models as finite-state
transducers using program code and scripts with a Makefile.  To test the code,
we have converted 42 hunspell dictionaries from various language families. They
consist of the dictionaries that were accessible from the aforementioned web
sites at the time of writing. The Table~\ref{table:hunspell-automata} gives an
overview of the sizes of the compiled automata. The size is given in binary
multiples of bytes as reported by \texttt{ls -hl}.  In the
Table~\ref{table:hunspell-automata}, we also give the number of roots in the
dictionary file and the affixes in affix file. These numbers should also help
with identifying the version of the dictionary, since there are multiple
different versions available in the downloads.

\begin{table}[tbp]
  \caption{Compiled Hunspell automata sizes}
  \label{table:hunspell-automata}
  \centering
  \begin{tabular}{cccc}
    \hline
    \textbf{Language} & \textbf{Dictionary} & \textbf{Roots} & \textbf{Affixes} \\
    \hline
    \hline
    Portugese (Brazil) & 14 MiB & 307,199 & 25,434 \\
    Polish & 14 MiB & 277,964 & 6,909 \\
    Czech & 12 MiB & 302,542 & 2,492 \\
    Hungarian & 9.7 MiB & 86,230 & 22,991 \\
    Northern S\'{a}mi & 8.1 MiB & 527,474 & 370,982 \\ 
    Slovak & 7.1 MiB & 175,465 & 2,223 \\
    Dutch & 6.7 MiB & 158,874 & 90\\
    Gascon & 5.1 MiB & 2,098,768 & 110 \\
    Afrikaans & 5.0 MiB & 125,473 & 48 \\
    Icelandic & 5.0 MiB & 222087 & 0 \\
    Greek & 4.3 MiB & 574,961 & 126 \\
    Italian & 3.8 MiB & 95,194 & 2,687 \\
    Gujarati & 3.7 MiB & 168,956 & 0 \\
    Lithuanian & 3.6 MiB & 95,944 & 4,024 \\
    English (Great Britain) & 3.5 MiB & 46,304 & 1,011 \\
    German & 3.3 MiB & 70,862 & 348 \\
    Croatian & 3.3 MiB & 215,917 & 64 \\
    Spanish & 3.2 MiB & 76,441 & 6,773 \\
    Catalan & 3.2 MiB & 94,868 & 996 \\
    Slovenian & 2.9 MiB & 246,857 & 484 \\
    Faeroese & 2.8 MiB & 108,632 & 0 \\
    French & 2.8 MiB & 91,582 & 507 \\
    Swedish & 2.5 MiB & 64,475 & 330 \\
    English (U.S.) & 2.5 MiB & 62,135 & 41 \\
    Estonian & 2.4 MiB & 282,174 & 9,242\\
    Portugese (Portugal) & 2 MiB & 40.811 & 913 \\
    Irish & 1.8 MiB & 91,106 & 240 \\
    Friulian & 1.7 MiB & 36,321 & 664 \\
    Nepalese & 1.7 MiB & 39,925 & 502 \\
    Thai & 1.7 MiB & 38,870 & 0 \\
    Esperanto & 1.5 MiB & 19,343 & 2,338 \\
    Hebrew & 1.4 MiB & 329237 & 0 \\
    Bengali & 1.3 MiB & 110,751 & 0 \\
    Frisian& 1.2 MiB & 24,973 & 73 \\
    Interlingua & 1.1 MiB & 26850 & 54 \\
    Persian & 791 KiB & 332,555 & 0 \\
    Indonesian & 765 KiB & 23,419 & 17 \\
    Azerbaijani & 489 KiB & 19,132 & 0 \\
    Hindi & 484 KiB & 15,991 & 0 \\
    Amharic & 333 KiB & 13,741 & 4 \\
    Chichewa & 209 KiB & 5,779 & 0 \\
    Kashubian & 191 KiB & 5,111 & 0 \\
    \hline
  \end{tabular}
\end{table}

To test the converted spell-checking dictionaries and error models, we pickedn
10 dictionaries of varying size and features. For spelling material we
created sets of 1000 spelling errors automatically, by introducing spelling
errors to the tokens of Wikipedia data. The errors have been made by a python
script implementing the edit distance type of errors to the words with 
likelihood of 1/33 per character. The words which didn't receive any automatic
mispellings were not included in the test set, but words were spelling errors
introduced led to another word form of the language were retained. 

The table~\ref{table:error-analysis} summarizes the spelling suggestions made
by original hunspell algorithms, and our finite-state automata. Four variants
of finite-state automata combinations were tested; one allowing for two
hunspell errors without any weighting, one with wikipedia frequencies in
dictionary and one with weighted error model allowing up to four of converted
hunspell errors.  The hunspell results were obtained by \texttt{hunspell -1 -d
\$LL < misspelings}, and automata were applied using experimental HFST tool
\texttt{hfst-ospell error-model dictionary}. In the table, the column C is for
correct spelling results, that are foyund in the dictionary---in this case,
false positives.  The colums 1, 2---4 and L show numbers of correct results
showing as first, other top four, or lower suggestions. The column M contains
misses, where correct suggestion was not given at all.

\begin{table}[tbp]
  \caption{Suggestion algorithm results}
  \label{table:error-analysis}
  \centering
  \begin{scriptsize}
  \begin{tabular}{c ccccc ccccc ccccc ccccc}
    \hline
    \textbf{Language} & \multicolumn{5}{c}{\bf Hunspell} & \multicolumn{5}{c}{\bf FST} & \multicolumn{5}{c}{\bf FST + Unigrams} & \multicolumn{5}{c}{\bf FST + 4 errors}  \\
    & C & 1 & 2---4 & L & M & C & 1 & 2---4 & L & M & C & 1 & 2---4 & L & M & C & 1 & 2---4 & L & M  \\
    \hline
    \hline
    English (American) & 46 & 768 & 116 & 20 & 50 & 20 & 515 & 203 & 121 & 141 & 20 & 575 & 158 & 106 & 141 &&&& \\
    Occitan & &&&& & 0 & 233 & 20 & 2 & 25 & 0 & 236 & 19 & 0 & 25 & 0 & 239 & 20 & 2 & 19 \\
    Kurdish & &&&& & 1 & 238 & 27 & 3 & 141 & 1 & 237 & 27 & 4 & 141 & 1 & 238 & 30 & 4 & 137 \\
    Interlingua & &&&& & 7 & 570 & 122 & 27 & 274 & 7 & 790 & 107 & 15 & 81 & \\
    Polish & &&&& &&&& &&&& \\
    German & &&&& & 9 & 636 & 77 & 22 & 256 &&& &&&& \\
    Hungarian & &&&& & 6 & 424 & 30 & 8 & 482 &&& &&&& \\
    French & \\
    Slovak & \\
    Icelandic & \\
    \hline
  \end{tabular}
  \end{scriptsize}
\end{table}

The time requirements of each system was also briefly teste using standard
Unix \verb|time(1)| tool to measure the time of correcting the 1000
mispelled strings used for testing precision and recall of the systems
previously. The times were measured on application server provided by
centre of scientific computation Finland with 8 quad-core processors 
AMD 8360 and 512 GiB of RAM memory available 
\footnote{\url{http://www.csc.fi/english/pages/hippu\_guide/introduction/overview/index\_html/?searchterm=hippu}}.

The tests reveal that the weighting of dictionary and error model give no
significant hit to the performance of spell checking and correction, whereas
extending the search space by doubling the error model will sharply decrease the
running time.


\begin{table}[tbp]
  \caption{Suggestion algorithm speed}
  \label{table:error-analysis}
  \centering
  \begin{tabular}{ccccc}
    \hline
    \textbf{Language} & \textbf{Hunspell} & \textbf{FST} & \textbf{FST + Unigrams} & \textbf{FST + 4 errors} \\
    \hline
    \hline
    English (American) & 56.5 s & 15.5 s & 16.1 s &  \\
    Occitan & & 0.1 s & 0.1 s & 86.3 s \\
    Kurdish & & 1.1 s & 1.1 s & 103.2 s \\
    Interlingua & & 2.9 s & 4.9 s & 196.3 s \\
    Polish & & & & \\\
    German & & & & \\
    Hungarian & & & & \\
    French & \\
    Slovak & \\
    Icelandic & \\
    \hline
  \end{tabular}
\end{table}

\section{Conclusion}

We have demonstrated a method and created the software to convert legacy
spell-checker and hyphenation data to a more general framework of
finite-state automata and used it in a real-life application.

\section*{Acknowledgment}

The authors would like to thank Miikka Silfverberg, Sam Hardwick and Brian
Croom and others in the HFST research team for contributions to research tools
and useful discussions.

\bibliographystyle{splncs}
\bibliography{il-2010}

% that's all folks
\end{document}
