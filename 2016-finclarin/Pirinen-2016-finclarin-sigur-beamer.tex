\documentclass{beamer}

\usepackage{fontspec}
\usepackage{polyglossia}

\usepackage{graphicx}
\usepackage{color}
\usepackage{url}

\mode<presentation>
{
    \usetheme{HZSK}
}

\makeatletter
\newcommand\listofframes{\@starttoc{lbf}}
\makeatother

\addtobeamertemplate{frametitle}{}{%
  \addcontentsline{lbf}{section}{\protect\makebox[2em][l]{%
    \protect\usebeamercolor[fg]{structure}\insertframenumber\hfill}%
  \insertframetitle\par}%
}


\title{ACL SIG for Uralic Languages SIGUR,
\scriptsize{in Fin-CLARIN, Helsinki, 2016.\\
\url{http://acl-sigur.github.io}}}
\author{Tommi A Pirinen \scriptsize \guilsinglleft
tommi.antero.pirinen@uni-hamburg.de \guilsinglright }
\institute{Universität Hamburg}
\date{\today}

\begin{document}

\selectlanguage{english}

\maketitle

\section{ACL SIGUR}

\begin{frame}
    \frametitle{ACL}
    Association for computational linguistics:
    \begin{itemize}
        \item \url{http://aclweb.org}
        \item Members are academics, professionals etc. interested in CL
        \item Responsible for many high-impact conferences, publications
            etc. in CL
    \end{itemize}
\end{frame}

\begin{frame}
    \frametitle{SIG}
    Special Interest Group:
    \begin{itemize}
        \item A (loosely organised) group of people interested in a
            sub-topic of Computational linguistics (e.g. a language
            group, a theory or sub-field)
        \item organises events, publications, usually also resources,
            standards and best common practices
        \item
    \end{itemize}
\end{frame}

\begin{frame}
    \frametitle{UR}
    as in SIGUR, Special Interest Group for Uralic Ranguages
    \url{http://gtweb.uit.no/sigur}
    \begin{itemize}
        \item Newly founded ACL SIG
        \item organising yearly workshops, journal publications
        \item first group effort: harmonising and expanding Uralic UD
        \item second group effort: listing existing resources and co-ordinating
        \item third effort: Best common practices for resources and stuff
    \end{itemize}
\end{frame}

\begin{frame}
    \frametitle{The catalogising and co-ordinating effort}
    \begin{itemize}
        \item Because: there has been some duplicated work and there are
            undiscovered resources everywhere
        \item Existing resource catalogues in CLARIN for
            Finland\footnote{\url{http://kielipankki.fi}},
            Estonia\footnote{\url{http://keeleressursid.ee}}, and
            Hungary(?)
        \item Probably missing many resources outside these countries and
            national languages
        \item we concentrate on lightweight gathering and linking of free and
            open resource, no metadata harvesting and catalogising in databases
            and web services
    \end{itemize}
\end{frame}

\begin{frame}
    \frametitle{Uralic Universal Dependencies: Example of non-co-operation
    problematics}
    Languages that are quite known to be related and even mutually intelligible
    should probably share the same analyses for diachronically and synchronically
    same morphs and structures
    \begin{itemize}
        \item For longer presentation on the Uralic tagging issues,
            see Tyers \& Pirinen (2016) at
            \url{http://www.computing.dcu.ie/~tpirinen/Pirinen-2016-iwclul-rbmt-representations.pdf}
        \item E.g. The past participle in Finnish is
            \texttt{VerbForm=Part|PartForm=Pres}, Estonian
            \texttt{VerbForm=Part|Tense=Pres} and Hungarian \texttt{VerbForm=PartPres}
        \item Read and write UD documentations for intermediate level between
            Universal and language specific \url{http://universaldependencies.org/urj/overview/introduction.html}
        \item Write a lot of bugs \url{https://github.com/UniversalDependencies/docs/labels/Uralic}
        \item Work with authors of current UD treebanks (help / bribe / coerce)
            to harmonise and, work with new languages from beginning on as
            guide
        \item ...
    \end{itemize}
\end{frame}

\begin{frame}
    \frametitle{Best common practices (getting things done)}
    Idea brought up in 2016 meeting:
    \begin{itemize}
        \item We have experience of getting language support for Uralic languges
            in digital world:
        \item keyboards, mobile keyboards, spell checking, corpora,
        \item how-to: build new (to computational) Uralic languages to same level
    \end{itemize}
\end{frame}

\end{document}
