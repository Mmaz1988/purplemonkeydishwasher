\documentclass{beamer}


\usepackage{amssymb}
\usepackage{amsmath}
\usepackage{amsfonts}

\usepackage{fontspec}
\usepackage{polyglossia}

\usepackage{graphicx}
\usepackage{color}
\usepackage{url}
\usepackage{textpos}
\usepackage{xspace}
\usepackage{array}


\graphicspath{{./fig/}}


\title{Why Linguistics in Statistical Machine Translation (SMT)?\\
\scriptsize{in Tarto, 2015}}
\author{Tommi A Pirinen \scriptsize \guilsinglleft{}tommi.pirinen@computing.dcu.ie\guilsinglright{}}
\institute{DCU, ADAPT--CNGL\\Abumatran}
\date{\today}

\begin{document}

\selectlanguage{english}

\maketitle

\section{Introduction}

\begin{frame}
    \frametitle{Statistical Machine Translation (SMT) in nutshell}
    \begin{itemize}
        \item feed parallel texts to computer
        \item computer learns / memorises probable translations
        \item translate new text
        \item compare to reference translation to get scores, using automatic
            evaluation metrics
        \item rarely: ask humans to either read the translations or fix them,
            and measure times for that, etc.
    \end{itemize}
\end{frame}

\begin{frame}
    \frametitle{Statistical Machine Translation ``research''}
    \begin{itemize}
        \item need to improve scores; either:
            \begin{itemize}
                \item add more data
                \item try new (statistical) algorithms blindly to see what sticks
            \end{itemize}
        \item instead of:
            \begin{enumerate}
                \item analyse what is wrong in the translations
                \item hypothesize what may fix the problem
                \item experiment
                \item profit
            \end{enumerate}
    \end{itemize}
\end{frame}

\section{State of the Art}

\begin{frame}
    \frametitle{State of the Art in Finnish-English SMT}
    Some table of results, variance, stuff
\end{frame}

\begin{frame}
    \frametitle{State of the Art in Finnish-English SMT cont'd}
    Table of results ref. approach
\end{frame}

\begin{frame}
    \frametitle{WMT 2015 results, linguistics-based approaches highlighted}
    statmt matrix w/ omorfi systems highlighted
\end{frame}

\section{Automatic metrics vs. linguistic intuition}

\begin{frame}
    \frametitle{Automatic metrics and linguistics}
    \begin{itemize}
        \item automatic metrics have been designed to correlate with human
            evaluation on e.g., fluency
        \item however, ...
        \item
    \end{itemize}
\end{frame}

\begin{frame}
    \frametitle{Examples}
    \begin{itemize}
        \item (to be taken with a grain of salt; automatic metrics work on
            masses of texts (and references) rather than single sentences)
        \item negation mismatch
        \item direct object case flaws
        \item etc.
    \end{itemize}
\end{frame}


\end{document}
% vim: set spell:
