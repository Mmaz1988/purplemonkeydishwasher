\documentclass{beamer}


\usepackage{amssymb}
\usepackage{amsmath}
\usepackage{amsfonts}

\usepackage{fontspec}
\usepackage{polyglossia}

\usepackage{graphicx}
\usepackage{color}
\usepackage{url}
\usepackage{textpos}
\usepackage{xspace}
\usepackage{array}

\mode<presentation>
{
  \usetheme{Abumatran}
}


\graphicspath{{./fig/}}


\title{Why Linguistics in Statistical Machine Translation (SMT)?\\
\scriptsize{in Tartu, 2015}}
\author{Tommi A Pirinen \scriptsize \guilsinglleft{}tommi.pirinen@computing.dcu.ie\guilsinglright{}}
\institute{Ollscoil Chathair Bhaile Átha Cliath, ADAPT Centre\\
EU Marie Curie Abu-MaTran project}
\date{\today}

\begin{document}

\selectlanguage{english}

\maketitle

\begin{frame}
    \frametitle{Contents}
    \begin{itemize}
        \item SMT, SMT research and using linguistics
        \item automatic measures of SMT quality,
            and linguistic
        \item linguistic approaches in SMT systems
        \item Finnish$\leftrightarrow$English SMTs
    \end{itemize}
\end{frame}


\begin{frame}
    \frametitle{SMT in a nutshell}
    \begin{itemize}
        \item feed parallel texts to computer
        \item computer learns / memorises probable translations
        \item translate new text
        \item compare to reference translation to get scores, using automatic
            evaluation metrics
        \item rarely: ask humans to either read the translations or fix them,
            and measure times for that. E.g., when
            working with translators, 
    \end{itemize}
\end{frame}

\begin{frame}
    \frametitle{SMT \emph{research} in another nutshell}
    \begin{itemize}
        \item need to improve scores, so either:
            \begin{itemize}
                \item add more data
                \item try new (statistical) algorithms blindly to see what sticks
                \item no improvement; give up.
            \end{itemize}
        \item instead of:
            \begin{enumerate}
                \item analyse what is wrong in the translations
                \item hypothesize what may fix the problem
                \item experiment
                \item realise that fix breaks something else
                \item salvage the scraps
            \end{enumerate}
    \end{itemize}
\end{frame}

\begin{frame}
    \frametitle{Why linguistics in SMT: motivational results from
    WMT 2015 shared tasks (competition) Finnish$\leftrightarrow$English}
    \begin{columns}
        \begin{column}{.40\textwidth}
    \tiny{
                   \begin{tabular}{|l|r|r|}
                        \hline
    \bf System & \bf BLEU & \bf TER \\
    \hline
    English to Finnish \\
    \bf \em abumatran-enfi-uncons-combo & \bf 15.5 & \bf 0.777 \\
    \em abumatran-enfi-uncons  & 14.9 & 0.803 \\
    UU-enfi-unconstrained  & 13.7 & 0.796 \\
    \hline
    \bf \em abumatran-enfi-combo  & \bf 12.7 & 0.804 \\
    CMU.fi-en  & 12.5 & \bf 0.798 \\
    \bf abumatran-enfi-hfstcomp & 11.6 & 0.830 \\
    \em AaltoMorphsRescored & 11.6 & 0.808 \\
    \hline
Finnish to English \\
    \hline
    uedin-syntax-fi-en & \bf 17.9 & 0.738 \\
    \bf \em abumatran-fien-combo  & 17.6 & \bf 0.727 \\
    UU-fien & 16.4 & 0.749 \\
    \em abumatran-fien & 15.9 & 0.764 \\
    \bf UIUC fi-en omorfi & 15.7 & 0.764 \\
    \bf abumatran-fien-hfstmorph & 15.3 & 0.782 \\
    \hline
\end{tabular}
    }
    \end{column}
    \begin{column}{.59\textwidth}
        \begin{itemize}
            \item N.B. BLEU bigger is better, TER lower is better
        \item The \textbf{bold-face} systems are based on linguistic
            morphology
        \item The \emph{emphasized} systems are based on unsupervised
            segmentation
        \item both is both, neither is unknown
        \item \ldots but what's the point? 
        \end{itemize}
    \end{column}
\end{columns}
\end{frame}

\begin{frame}
    \frametitle{More details on linguistics vs statistical segmentation}
    \begin{itemize}
        \item segmentation for SMT chops words into
            pieces
        \item linguistic morphology into stems and affixes
        \item statistical segmentation into e.g., most likely
            letter sequences
        \item in our tests, plain linguistic segmentation
            is better than statistical in majority
            of the cases English to Finnish
        \item Finnish to English, scores are unclearer, though manual evaluation shows that unknown
            words do get translated in more cases
    \end{itemize}
\end{frame}

\begin{frame}
    \frametitle{A bit more about the automatic metrics of translation quality}
    \begin{itemize}
        \item roughly speaking, in BLEU, reference translation is compared
            word by word to find matching words,
            longer sequences of words get more points
        \item works quite neatly with strict word order languages without
            much inflection
        \item TER measures roughly the moves, insertions and deletions of
            words
        \item in practice, \emph{most} statistical machine translation is research of
            optimising these values
        \item as an example, missing negation is not
            penalised much in these measurements!
    \end{itemize}
\end{frame}



\begin{frame}
    \frametitle{Why (not) automatic metrics}
    \begin{itemize}
        \item cheap
        \item to correlate with human
            evaluation on e.g., fluency, and on a large scale
        \item the reference may be high quality but machine translatability
            is low (added / removed information to take target audience into
            account, etc.)
        \item increase/decrease captures some changes
            but non-change in score doesn't mean no
            changes in translation (quality)
    \end{itemize}
\end{frame}

\begin{frame}
    \frametitle{Automatic metrics of linguistic systems: Finnish-English case}
    When people have tried linguistic approaches to improve SMT for Finnish
    \begin{itemize}
        \item morphological segmentation: most publications give no or very little improvement in BLEU
        \item morphological factors: no improvements that I've read of
        \item so: linguistics may not improve Finnish SMT
        \item however: looking at results, the methods fix what they should, usually
        \item can further linguistic analysis tell us how to keep
            the improvements while not messing up anything else?
    \end{itemize}
\end{frame}


\begin{frame}
    \frametitle{So, what should linguists measure besides the BLEUs and TERs}
    \begin{itemize}
        \item it depends.
        \item if you aim to fix agreement, measure agreement errors
        \item if you are fixing negation problems, measure negations, etc.
        \item see what the fix breaks:
        \item evaluate what you fixed (based on the starting point)
            and what you broke (based on the results); requires reading the
            translations and thinking (linguistically)
        \item there are generic linguistic metrics too
    \end{itemize}
\end{frame}

\begin{frame}
    \frametitle{Example of linguistic evaluation in SMT}
    From Clifton et al. (2011), error analysis:
    \begin{enumerate}
        \item  explicitly marked noun forms,
        \item noun-adjective case agreement,
        \item subject-verb person/number agreement
        \item transitive object case marking
        \item postpositions,
        \item possession.
    \end{enumerate}
    Lower BLEU score leads to higher F-score on
    most linguistic phenomena.
\end{frame}


\begin{frame}
    \frametitle{Inconclusion}
    Linguistics in SMT is good for:
    \begin{itemize}
        \item Winning WMT shared task
        \item Finding problems
        \item Solving some problems
        \item Error analysis
        \item Salvaging good things of improved
            systems despite decreased score
        \item Create more challenges than you solve
        \item ...
    \end{itemize}
\end{frame}

\begin{frame}
    \frametitle{Questions?}
    Thanks, aitäh.
\end{frame}

\end{document}
% vim: set spell:
