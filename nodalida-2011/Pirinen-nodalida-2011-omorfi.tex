\documentclass[11pt]{article}
\usepackage{eacl2009}
\usepackage{times}
\usepackage{url}
\usepackage{latexsym}
\usepackage{xltxtra}

%\setlength\titlebox{6.5cm}    % You can expand the title box if you
% really have to

\title{Modularisation of Finnish Finite-State Language Description---Towards
Wide Collaboration in Open Source Development of Morphological Analyser}

%\author{Tommi A Pirinen\\
%  University of Helsinki\\
%  Helsinki, Finland\\
%  {\tt tommi.pirinen@helsinki.fi}
%}

%\date{\today}

\begin{document}
\maketitle
\begin{abstract}
In this paper we present an open source implementation for Finnish
morphological parser. We shortly evaluate it against contemporary criticism
towards monolithic and unmaintainable finite-state language description.  We
use it to demonstrate way of writing finite-state language description that is
used for varying set of projects, that typically need morphological analyser,
such as POS tagging, morphological analysis, hyphenation, spell checking and
correction, rule-based machine translation and syntactic analysis.  The
language description is done using available open source methods for building
finite-state descriptions coupled with autotools-style build system, which is
de facto standard in open source projects.
\end{abstract}

\section{Introduction}

Writing maintainable language descriptions for finite-state systems has
traditionally been a laborious task. Even though finite-state technology has
been de facto standard for writing computational language descriptions for more
than two decades now~\cite{beesley/2003}, it has some recognised flaws and
problems both caused by shortcomings of actual implementations and
background technology~\cite{wintner/2008}. Commonly language description is
performed by a single linguist or language technologist. The descriptions
typically wind up being complex enough that modifying them requires a great
amount of studying and understanding before one is able to do the smallest of
modifications to the system. In the current times that all proper, healthy,
scientific projects should be open source and globally developed, this poses a
challenge for such project's internal structure. Another source of problem in
such collaboration is that background of contributors for language description
varies from computer scientists to linguists~\cite{maxwell/2008} to
computer-savvy native language speakers, all of whom should be able to
contribute to the project.  The solutions we propose for this is to embrace
proper modularisation in language descriptions to allow multiple specific
entry points to contributors.

In this paper we describe a new implementation of the Finnish language
description called omorfi\footnote{http://home.gna.org/omorfi}, made to support
large variety of NLP applications and different audiences.  While the
background theory for implementing finite-state description of Finnish was laid
out already in \newcite{koskenniemi/1983}, and morphophonological system
doesn't have significant changes, the actual system was rewritten from the
scratch. The rewriting was originally done by single linguist as usual, in a
master's thesis project~\cite{pirinen/2008}, but afterwards it has been
extended as full-fledged open source project and used in various contexts. This
extended development has made necessary to create a better modularised
framework to allow people of different level of knowledge and familiarity with
finite-state technology and Finnish to contribute on their prospective parts of
the description without causing problems for other applications of the
finite-state analysers.

The projects that have used and use omorfi as language description include
spell checking and correction~\cite{pirinen/2010/lrec}, lemmatizing for IR
applications (e.g.~\newcite{kurola/2010}), named entity recognition, rule-based
machine translation (e.g. Finnish---Northern
Sámi\footnote{\url{http://apertium.svn.sourceforge.net/viewvc/apertium/incubator/apertium-sme-fin/}})
and syntactic disambiguation and analysis. The demands for even the basic
morphology with all these different applications are very different with
regards to productivity; lexical coverage and accuracy as well as depth of
tagging, so it has became obvious that no one lexical automaton will work for
everyone.  For this reason the modularisation has to provide easily
configurable options and modifiability for all end-points.

The modularisation of omorfi was based primarily on technological development,
which is the reason why it still maintains some traditional finite-state
morphology distinctions, such as strict separation off morphological
combinatorics (i.e. lexc code) and morphophonological phenomena (i.e.
twolc code). The same is similarly true for new code base, such as one for
training weighted finite-state models; weighting was split to its one
module.

One of the key points in modular structure here is that we ensure that modifying
will not typically break already working parts, so contributors adding new words
or moving hyphens will not cause problems in other parts of description
as much as possible.

\section{Modularisation of Finite-State Language Description}
\label{sec:modules}

The modularisation scheme we ended up with in finite-state description of
Finnish has grown organically around rather standard description of
finite-state morphology not significantly different from the work of
\newcite{koskenniemi/1983}. The further developments of modularisation more or
less have followed from development of finite-state technology along years
from initial implementation of omorfi at publication of \newcite{pirinen/2008}
to its current open source community developed version.

In figure~\ref{fig:modules} we give a hierarchical list of abstract
modules implemented in omorfi at the time of writing. As mentioned, the
classical modules of morphotactic combinatorics (i.e. Xerox compatible lexc
language description) and morphophonology (i.e. Xerox compatible twolc
description) is still present. The morphotactic combinatorics has already been
split to sub modules for two reasons. First is primarily practical fact
that code base for morphotactic combinatorics for words of Finnish is huge.
Second and perhaps the more important distinction is the fact that central
and integral part of the life force of morphophonological description of
the all languages is to keep up with constant influx of new lexical items
to the language; neologisms, proper nouns and other coinages.  From further
new modules, orthographical variations was implemented to create detached
support for certain obvious variations of Finnish written data, e.g. the
typewriter and SF7-ASCII era digraphs like sh and zh in stead of š and ž
respectively. The hyphenation and syllabification of Finnish language is also
one obvious service for morphological dictionary to provide; for Finnish the
compound boundaries cannot typically be discriminated without a
dictionary~\cite{linden/2009/nodalida}.  One of the features that has become
rather obvious along years for morphological parsers is the fact that all
computational linguistic applications must require their own very special
version of morphological analysis, so in omorfi we've chosen to avoid lock-ins
for any specific type of \emph{tag sets} or morphological analyses so to say,
and instead go for one version of analysis to contain certain superset of all
needed forms and implement finite-state descriptions (i.e. simple rewriting
rules) to provide all of the different analysis styles. The statistical models
is one of recent developments of finite-state technology, and there's a lot
to offer for language models here so the whole family of weighted finite-state
training and models is also implemented in omorfi as a separate module here,
which for most intents and purposes does work independently of any other part
of the language description.

\begin{figure}[h]
\begin{centering}
\caption{Hierarchy of modules in modern finite-state implementation Finnish languege
\label{fig:modules}}
\end{centering}
\begin{scriptsize}
\begin{itemize}
\item morphotactics \begin{itemize}
    \item lexical data
    \item stem morphophonology
    \item inflection, derivation and compounding
    \item lexical-syntactic data
\end{itemize}
\item morphophonology
\item analysis models
\item orthographical variations
\item hyphenation and syllabification
\item error models
\item statistical models
\item lexicographic filtering
\end{itemize}
\end{scriptsize}
\end{figure}

\section{Implementation}
\label{sec:implementation}

Here we briefly discuss implementation of modules, mainly to discuss
about practices that help the cooperation. Naturally full discussion of the
modules is found in the documentation of the
system\footnote{\url{http://home.gna.org/omorfi/}}. The system
implementation is harnessing the autotools framework and unix style tools of
HFST to incrementally build the finite-state automata using finite-state
algebra, such as composition to extend them, originally noted even in
\newcite{beesley/2003}. The crucial thing for this modular approach is that it
can be applied incrementally, each module can be replaced or disabled entirely
at needs of end application, and with autotools framework all this
can be performed by simple command-line switch to \verb|./configure|.

\subsection{Lexical Data---Lexicon and Features of Lexical Items}

The initial part of morphotactics deals with lexicographical data. This is
the part where most modification and cooperation can be used, the lexical items
in language change all the time in introduction of new word forms, and the
expertise needed to extend the lexicon does not require significant expertise
beyond understanding of the language. For this case we provide different entry
formats for new lexical data; csv, XML and so on. The minimal data to enter for
new word form is morphological part of speech, defined in Finnish grammar
as distinction between verb, nominal or non-inflecting particle, additionally a
paradigm classification is typically needed for working inflection and
derivation. While this is facilitated as much as possible, it is still seen as
problematic by part of contributors, so further research for easy lexicon
management is still required.

The other practical example as to why easy modification of lexical data is
crucial is that for example for rule-based machine translation---in our case
apertium---it is useful to establish mapping between lexical units of source
language and target language. For this reason easy access to lexical units
is required for developers of rule-based machine translation units.

Another extension scheme needed is for projects working syntactic parsers based
on morphological language description---an example in Finnish
cases is yet another forthcoming constraint grammar\cite{karlsson/1990} based
analyser. In this case extensions can be also based on purely lexical data,
such as argument structures and ordering for verbs and adpositions. The method
to implement this is currently at preprocessing phase, however it could be
arguably a separate phase, since lexical data can be trivially composed
afterwards.

\subsection{Traditional Morphotactics and Phonology---The Lexc and Twolc Model}

The various lexical data sources are joined back to traditional lexc format,
which is combined with word stem variation definitions and inflectional data to
produce lexical automaton. This is compose-intersected with morphophonology
descriptions to produce the analyser already; as these parts rarely need
changes beyond bug-fixes and are unlikely to benefit from open source
cooperation beyond initial linguist work, they are still in same form as
traditional finite-state morphology by \newcite{beesley/2003}, even if it was
deemed monolithic and fragile for such collaboration.

\subsection{Analysis Formats and Sets}

Another thing that is quickly obvious for interoperability is that all projects
using morphological analyser, for whatever purpose, require their own analysis
format. Instead of converging to standard we have temporarily solved this by
making our analyser to contain superset of required features at all times, and
providing rules to rewrite the tagsets. The rulesets can be compiled to
finite-state networks and composed like usual. Typical rules are of course
relatively simple contextless rewriting, for example the annotation for
singular nominative is \verb|+sg+nom|, \verb| SG NOM| or \verb|<sg><nom>|, for
different applications, so a simple composition of \verb|NUM=SG:+sg| rule and
\verb|CASE=NOM:nom| is enough for providing the singular nominatives to that
analysis style.  Ideally of course this would be solved by using more suitable
abstract data type for the annotations than character string
\cite{wintner/2008}, ideally derived from standardized set of features, such as
ISOCat as is also suggested by \cite{maxwell/2008}.


\subsection{Orthographical variations}

When dealing with data from various sources, such as old literature or
\emph{spoken} standard language found in instant messaging and perhaps
transcribed spoken corpora, there are certain variations on spelling rules to
take care of. These has also been implemented as independent rule set compiled
to composable finite-state automaton. Incidentally both mapping of typewriter
digraphs sh and zh to š and ž correspondingly and omission of final component
of i-final diphthong of spoken language are both definable as rule working on
morphological analyser as an independent unit.

\subsection{Hyphenation and syllabification}

Hyphenation is in practice also one of the applications of the language, It has
been defined as a rule set over half-build morphological analyser, since it can
neatly abuse build-time information of the analyser. such as word and morpheme
boundaries. The syllables could also conversely be used by other parts of the
description if needed, e.g. the morphophonological alteration a:e depends more
on syllable count than traditional stem paradigms classes of Finnish

\subsection{Error models}

Error model is a crucial part of spell-checking system, for the correction
task. This is implemented as finite-state filter that can be 
applied with on-the-fly composition \cite{pirinen/2010/cla} to perform the
error correction for spell checking, or for example error-tolerant analysis.

\subsection{Statistical models}

Statistical models provide for disambiguating language models and spell-checking
tasks for example. The statistical models used are simple finite-state automata
or training sets combinable to the language description by use of 
composition~\cite{linden/2009/nodalida,linden/2009/fsmnlp}. 

\subsection{Filtering the Analyser}

The models needed for different task may need widely different dictionaries and
allowed word-forms, and not always the statistical models are sufficient to
discriminate between good word forms. So we also provide filter rule sets, to
limit features, such as derivation and compounding, and lexical units, such as
archaic or dialectal words. For example for the spell-checker's error detection
lexicon or information retrieval task compounding and derivation can be largely
allowed, whereas in the spelling correction the suggestions should be relatively
conservative for plausible but non-existing compounds and derivations.

\section{Discussion and Future Work}

In this article we have showed that finite-state description can be implemented
in modularised manner enabling wide cooperation in the open source context for
people with varying background. Especially

Furthermore we have demonstrated the ease of
proper abstraction in finite-state language description using easily available
open source tools while still providing open source community with the de facto
standard build system of autotoolset for wide distribution, packaging and
deployment.

What we did not address here is the easy way of coupling up-to-date
documentation with our modularised language description. The next step to
research is to see into integrating the notion of literate programming in this
framework. This topic has already been widely researched by
\newcite{maxwell/2008}, specifically in case of finite-state language
descriptions.

%\section*{Acknowledgements}

% We thank the HFST research group and our colleagues for fruitful discussions.

\bibliographystyle{acl}
\bibliography{nodalida2011}


\end{document}
