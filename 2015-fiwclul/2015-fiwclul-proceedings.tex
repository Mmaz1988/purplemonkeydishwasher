\documentclass[b5paper]{book}

\usepackage{polyglossia}
\usepackage{fontspec}
\usepackage{xunicode}
\usepackage{xltxtra}
\usepackage{url}
\usepackage{hyperref}
\usepackage{expex}

\usepackage{geometry}
\usepackage{pdfpages}

\usepackage{color}
\usepackage{fancyhdr}
\usepackage{lastpage}

\fancyhf{}
\chead{\tiny\color{blue} Proceedings of 1st International Workshop in Computational
    Linguistics for Uralic Languages (IWCLUL 2015); SOME URL [page \thepage{} of \pageref{LastPage}.}

\setmainfont[Mapping=tex-text]{Linux Libertine O}

\begin{document}
\frontmatter

\cleardoublepage
\thispagestyle{empty}
{\centering\setlength{\parindent}{0pt}{\setlength{\parskip}{0pt}
        {IWCLUL\par}
        \vspace{\fill}
        {\bfseries\Large First International Workshop on Computational
        Linguistics for Uralic Languages\par}
        \vspace{29mm}
        {\Large Proceedings of the Workshop\par}
        \vspace{\fill}
        {\Large January 16$^{th}$, 2015\\
        Tromsø, Norway}
\clearpage
}}

This work is licensed under a Creative Commons Attribution–NoDerivatives
4.0 International Licence.  Licence details:
\url{http://creativecommons.org/licenses/by-nd/4.0/}. Page numbering and
footers have been added by the editors.

WWW address

ISBN number

\clearpage

\pagestyle{fancy}

\chapter*{Preface}

The uralic languages are an interesting group of language from computational
linguistics perspective. They share large parts of morphophonological
complexity that is not present in the indo-european group which has
traditionally dominated the computational linguistics research. This can be
seen for example in number of word-forms-per-word, which in indo-european
languages is in range of 1's to 10's whereas for uralics languages it is
in range of 100's and 1000's. Furthermore, uralic language share a lot
of geo-political aspects: the national languages of the group---Finnish,
Estonian and Hungarian---are minor languages and only moderately resourced
in terms of computational linguistics resources while being stable and
not in threat of extinction, the recognised minority languages of western
European states---such as North Saami and Võro---are clearly in category
of lesser resourced and more threatened, whereas the majority of uralic
languages in the east Europe and Russia are close to extinction. Commonly
to all rapid development of more advanced computational linguistics methods
is required for continued vitality of the languages in everyday life, to
enable archiving and use of the languages with computers and other devices
such as mobile applications.

The research of computational linguistics and uralistics is been carried out
in a handful of Universities, research institutes and other sites by
relatively few researchers. Our intent for organising this conference is to
gather these researchers together in order to share ideas and resources, and
avoid duplicating efforts in gathering and enriching these scarce resources,
and hopefully to found an ongoing tradition of concentrated effort in
collecting and improving language resources and technologies for survival
of the uralic languages.

For the conference we received 14 high-quality submissions about topics
including computational lexicography, language documentation, optical character
recognisition, web-as-corpus and automatic and rule-based morphological
analysis methods.  These are all very important for preservation and
development of uralic languages. The coverage of languages in the submissions
is also very good: North Saami, Khanty, Mansi, Udmurt, Erzya, Moksha, Finnish
and Estonian. 

\noindent ---Tommi A Pirinen, Francis M. Tyers, Trond Trosterud,\\
Conference organisers,\\
2015, Tromsø


\chapter*{Organisers}

\begin{itemize}
    \item Tommi A. Pirinen, Ollscoil Chathair Bhaile Átha Cliath
    \item Francis M. Tyers, UiT Norgga árktalaš universitehta
    \item Trond Trosterud, UiT Norgga árktalaš universitehta
\end{itemize}

\chapter*{Programme committee}

\begin{itemize}
    \item Тимофей Архангельский, Национальный исследовательский университет "Высшая школа экономики"
    \item Lars Borin, Göteborgs universitet
    \item Марина Серафимовна Федина, Финн-йӧгра кывъяслы информатика отсӧг кузя регионкостса лаборатория
    \item Mark Fishel, Tartu ülikool
    \item Mikel L. Forcada, Universitat d'Alacant
    \item Mans Hulden, University of Colorado at Boulder
    \item Heiki-Jaan Kaalep, Tartu ülikool
    \item András Kornai, Budapesti Műszaki és Gazdaságtudományi Egyetem
    \item Krister Lindén, Helsingin yliopisto
    \item Tommi A. Pirinen, Ollscoil Chathair Bhaile Átha Cliath
    \item Gabór Prószéky, Pázmány Péter Katolikus Egyetem
    \item Aarne Ranta, Chalmers tekniska högskola
    \item Jack Rueter, Helsingin yliopisto
    \item Trond Trosterud, UiT Norgga árktalaš universitehta
    \item Francis M. Tyers, UiT Norgga árktalaš universitehta
    \item Sami Virpioja, Aalto-yliopisto
    \item Anssi Yli-Jyrä, Helsingin yliopisto
\end{itemize}

\tableofcontents

\mainmatter

\chapter{Invited speech}
\section{That stuff about twol and stuff}

Kimmo Koskenniemi: Few words about invited speech.

%\includepdfset{pagecommand{}}

\chapter{Tutorials}

\includepdf[scale=0.9,addtotoc={
1,section,2,Grammatical Framework Tutorial with a Focus on Fenno-Ugric Languages,sec:gf-tutorial},
clip,trim=0 40mm 0 0,pagecommand={\thispagestyle{fancy}},pages=-]{iwclul2015_submission_3.pdf}
\includepdf[scale=0.9,addtotoc={
1,section,2,Language Documentation meets Language Technology,sec:tutorial-2},
clip,trim=0 40mm 0 0,pagecommand={\thispagestyle{fancy}},pages=-]{iwclul2015_submission_13.pdf}

\chapter{Accepted Papers}

\includepdf[scale=0.9,addtotoc={
1,section,2,Low-Resource Active Learning of North Sámi Morphological Segmentation,sec:paper-1},
clip,trim=0 40mm 0 0,pagecommand={\thispagestyle{fancy}},pages=-]{iwclul2015_submission_4.pdf}
\includepdf[scale=0.9,addtotoc={
1,section,2,Compiling the Uralic Dataset for NorthEuraLex a Lexicostatistical Database of Northern Eurasia,sec:paper-2},
clip,trim=0 40mm 0 0,pagecommand={\thispagestyle{fancy}},pages=-]{iwclul2015_submission_5.pdf}
\includepdf[scale=0.9,addtotoc={
1,section,2,Can Morphological Analyzers Improve the Quality of Optical Character Recognition?,sec:paper-3},
clip,trim=0 40mm 0 0,pagecommand={\thispagestyle{fancy}},pages=-]{iwclul2015_submission_6.pdf}
\includepdf[scale=0.9,addtotoc={
1,section,1,Corpus.mari-language.com: A Rudimentary Corpus Searchable by Syntactic and Morphological Patterns,sec:paper-4},
clip,trim=0 40mm 0 0,pagecommand={\thispagestyle{fancy}},pages=-]{iwclul2015_submission_8.pdf}
\includepdf[scale=0.9,addtotoc={
1,section,1,Infinite Monkeys of Babel --- Crowdsourcing for the betterment of OCR language material,sec:paper-5},
clip,trim=0 40mm 0 0,pagecommand={\thispagestyle{fancy}},pages=-]{iwclul2015_submission_14.pdf}
\includepdf[scale=0.9,addtotoc={
1,section,1,Multilingual Semantic MediaWiki for Finno-Ugric dictionaries,sec:paper-6},
clip,trim=0 40mm 0 0,pagecommand={\thispagestyle{fancy}},pages=-]{iwclul2015_submission_2.pdf}
\includepdf[scale=0.9,addtotoc={
1,section,1,The Finno-Ugric Languages and The Internet project,sec:paper-7},
clip,trim=0 40mm 0 0,pagecommand={\thispagestyle{fancy}},pages=-]{iwclul2015_submission_1.pdf}
\includepdf[scale=0.9,addtotoc={
1,section,1,On the Road to a Dialect Dictionary of Khanty Postpositions,sec:paper-8},clip,trim=0 40mm 0 0,pagecommand={\thispagestyle{fancy}},pages=-]{iwclul2015_submission_7.pdf}
\includepdf[scale=0.9,addtotoc={
1,section,1,FinUgRevita: Developing Language Technology Tools for Udmurt and Mansi,sec:paper-9},clip,trim=0 40mm 0 0,pagecommand={\thispagestyle{fancy}},pages=-]{iwclul2015_submission_9.pdf}
\includepdf[scale=0.9,addtotoc={
1,section,1,Automatic creation of bilingual dictionaries for Finno-Ugric languages,sec:paper-10},
clip,trim=0 40mm 0 0,pagecommand={\thispagestyle{fancy}},pages=-]{iwclul2015_submission_11.pdf}

\end{document}
