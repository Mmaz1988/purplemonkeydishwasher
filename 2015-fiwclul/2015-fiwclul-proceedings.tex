\documentclass[b5paper]{book}

\usepackage{polyglossia}
\usepackage{fontspec}
\usepackage{xunicode}
\usepackage{xltxtra}
\usepackage{url}
\usepackage{hyperref}
\usepackage{expex}

\usepackage{geometry}
\usepackage{pdfpages}

\usepackage{color}
\usepackage{fancyhdr}
\usepackage{lastpage}

\fancyhf{}
\chead{\tiny\color{blue} Proceedings of 1st International Workshop in Computational
    Linguistics for Uralic Languages (IWCLUL 2015); 
    \guilsinglleft \url{http://dx.doi.org/10.7557/scs.2015.2}\guilsinglright{}}
    \cfoot{\tiny\color{blue}  \rightmark [page \thepage{} of \pageref{LastPage}]
    \guilsinglleft \url{http://dx.doi.org/10.7557/...}\guilsinglright{}}
\setmainfont[Mapping=tex-text]{Linux Libertine O}

\begin{document}
\frontmatter

\cleardoublepage
\thispagestyle{empty}
{\centering\setlength{\parindent}{0pt}{\setlength{\parskip}{0pt}
        {IWCLUL\par}
        \vspace{\fill}
        {\bfseries\Large First International Workshop on Computational
        Linguistics for Uralic Languages\par}
        \vspace{29mm}
        {\Large Proceedings of the Workshop\par}
        \vspace{\fill}
        {\Large January 16$^{th}$, 2015\\
        Tromsø, Norway}
\clearpage
}}

This work is licensed under a Creative Commons Attribution–NoDerivatives
4.0 International Licence.  Licence details:
\url{http://creativecommons.org/licenses/by-nd/4.0/}. Page numbering and
footers have been added by the editors.

WWW address: \url{http://dx.doi.org/10.7557/scs.2015.2}

ISBN number TBD

eISSN: 2387-3086

DOI: (whole proceedings) 10.7557/scs.2015.2

Editors contact: \url{iwclul-2015@googlegroups.com}

\clearpage

\pagestyle{fancy}

\chapter*{Preface}

The Uralic languages are an interesting group of languages from
computational-linguistic perspective. They share large parts of morphological
and morphophonological complexity that is not present in the Indo-European
group which has traditionally dominated computational-linguistic research. This
can be seen for example in number of word forms per word, which in
Indo-European languages is in range of ones or tens whereas for Uralic
languages it is in range of hundreds and thousands. Furthermore, Uralic
languages share a lot of geo-political aspects: the national languages of the
group---Finnish, Estonian and Hungarian---are small languages and only
moderately resourced in terms of computational-linguistic resources while being
stable and not in threat of extinction, the recognised minority languages of
western-European states---such as North Sámi and Võro---are clearly in category
of lesser resourced and more threatened, whereas the majority of Uralic
languages in the east of Europe and Russia are close to extinction. Common to
all rapid development of more advanced computational-linguistic methods is
required for continued vitality of the languages in everyday life, to enable
archiving and use of the languages with computers and other devices such as
mobile applications.

The research of computational linguistics and Uralistics is being carried out
in a handful of universities, research institutes and other sites by
relatively few researchers. Our intention with organising this conference is to
gather these researchers together in order to share ideas and resources, and
avoid duplicating efforts in gathering and enriching these scarce resources,
and hopefully to found an ongoing tradition of concentrated effort in
collecting and improving language resources and technologies for the survival
of the Uralic languages.

For the conference we received 14 high-quality submissions about topics
including computational lexicography, language documentation, optical character
recognition, web-as-corpus and automatic and rule-based morphological analysis
methods.  These are all very important for preservation and development of
Uralic languages. We also received a broad coverage of languages in the
submissions: North Sámi, Khanty, Mansi, Udmurt, Erzya, Moksha, Finnish and
Estonian.

The conference was held in University of Tromsø, Norway, on January 16th 2015,
and consisted of poster sessions, three talks,
two tutorials, and an invited speech,
The articles related to poster sessions and the talks are included in this
proceedings.

\noindent ---Tommi A Pirinen, Francis M. Tyers, Trond Trosterud,\\
Conference organisers,\\
2015, Tromsø


\chapter*{Organisers}

\begin{itemize}
    \item Tommi A. Pirinen, Ollscoil Chathair Bhaile Átha Cliath
    \item Francis M. Tyers, UiT Norgga árktalaš universitehta
    \item Trond Trosterud, UiT Norgga árktalaš universitehta
\end{itemize}

\chapter*{Programme committee}

\begin{itemize}
    \item Тимофей Архангельский, Национальный исследовательский университет "Высшая школа экономики"
    \item Lars Borin, Göteborgs universitet
    \item Марина Серафимовна Федина, Финн-йӧгра кывъяслы информатика отсӧг кузя регионкостса лаборатория
    \item Mark Fishel, Tartu ülikool
    \item Mikel L. Forcada, Universitat d'Alacant
    \item Mans Hulden, University of Colorado at Boulder
    \item Heiki-Jaan Kaalep, Tartu ülikool
    \item András Kornai, Budapesti Műszaki és Gazdaságtudományi Egyetem
    \item Krister Lindén, Helsingin yliopisto
    \item Tommi A. Pirinen, Ollscoil Chathair Bhaile Átha Cliath
    \item Gabór Prószéky, Pázmány Péter Katolikus Egyetem
    \item Aarne Ranta, Chalmers tekniska högskola
    \item Jack Rueter, Helsingin yliopisto
    \item Trond Trosterud, UiT Norgga árktalaš universitehta
    \item Francis M. Tyers, UiT Norgga árktalaš universitehta
    \item Sami Virpioja, Aalto-yliopisto
    \item Anssi Yli-Jyrä, Helsingin yliopisto
\end{itemize}

\tableofcontents

\mainmatter

\chapter{Invited speech}
\section{Direct comparison of language forms in two-level framework}


%\includepdfset{pagecommand{}}

\chapter{Tutorials}

\includepdf[scale=0.9,addtotoc={
1,section,2,Grammatical Framework Tutorial with a Focus on Fenno-Ugric Languages,sec:gf-tutorial},
clip,trim=0 40mm 0 40mm,pagecommand={\thispagestyle{fancy}},pages=-]{iwclul2015_submission_3crop.pdf}
\includepdf[scale=0.9,addtotoc={
1,section,2,Language Documentation meets Language Technology,sec:tutorial-2},
clip,trim=0 40mm 0 40mm,pagecommand={\thispagestyle{fancy}},pages=-]{iwclul2015_submission_13crop.pdf}

\chapter{Accepted Papers}

\includepdf[scale=0.9,addtotoc={
1,section,2,Low-Resource Active Learning of North Sámi Morphological Segmentation,sec:paper-1},
clip,trim=0 40mm 0 40mm,pagecommand={\thispagestyle{fancy}},pages=-]{iwclul2015_submission_4crop.pdf}
\includepdf[scale=0.9,addtotoc={
1,section,2,Compiling the Uralic Dataset for NorthEuraLex a Lexicostatistical Database of Northern Eurasia,sec:paper-2},
clip,trim=0 40mm 0 40mm,pagecommand={\thispagestyle{fancy}},pages=-]{iwclul2015_submission_5crop.pdf}
\includepdf[scale=0.9,addtotoc={
1,section,2,Can Morphological Analyzers Improve the Quality of Optical Character Recognition?,sec:paper-3},
clip,trim=0 40mm 0 40mm,pagecommand={\thispagestyle{fancy}},pages=-]{iwclul2015_submission_6crop.pdf}
\includepdf[scale=0.9,addtotoc={
1,section,1,Corpus.mari-language.com: A Rudimentary Corpus Searchable by Syntactic and Morphological Patterns,sec:paper-4},
clip,trim=0 40mm 0 40mm,pagecommand={\thispagestyle{fancy}},pages=-]{iwclul2015_submission_8crop.pdf}
\includepdf[scale=0.9,addtotoc={
1,section,1,Infinite Monkeys of Babel --- Crowdsourcing for the betterment of OCR language material,sec:paper-5},
clip,trim=0 40mm 0 40mm,pagecommand={\thispagestyle{fancy}},pages=-]{iwclul2015_submission_14crop.pdf}
\includepdf[scale=0.9,addtotoc={
1,section,1,Multilingual Semantic MediaWiki for Finno-Ugric dictionaries,sec:paper-6},
clip,trim=0 40mm 0 40mm,pagecommand={\thispagestyle{fancy}},pages=-]{iwclul2015_submission_2crop.pdf}
\includepdf[scale=0.9,addtotoc={
1,section,1,The Finno-Ugric Languages and The Internet project,sec:paper-7},
clip,trim=0 40mm 0 40mm,pagecommand={\thispagestyle{fancy}},pages=-]{iwclul2015_submission_1crop.pdf}
\includepdf[scale=0.9,addtotoc={
1,section,1,On the Road to a Dialect Dictionary of Khanty Postpositions,sec:paper-8},clip,trim=0 40mm 0 40mm,pagecommand={\thispagestyle{fancy}},pages=-]{iwclul2015_submission_7crop.pdf}
\includepdf[scale=0.9,addtotoc={
1,section,1,FinUgRevita: Developing Language Technology Tools for Udmurt and Mansi,sec:paper-9},clip,trim=0 40mm 0 40mm,pagecommand={\thispagestyle{fancy}},pages=-]{iwclul2015_submission_9crop.pdf}
\includepdf[scale=0.9,addtotoc={
1,section,1,Automatic creation of bilingual dictionaries for Finno-Ugric languages,sec:paper-10},
clip,trim=0 40mm 0 40mm,pagecommand={\thispagestyle{fancy}},pages=-]{iwclul2015_submission_11crop.pdf}

\end{document}
