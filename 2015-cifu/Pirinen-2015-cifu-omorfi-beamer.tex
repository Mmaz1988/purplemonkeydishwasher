\documentclass{beamer}


\usepackage{amssymb}
\usepackage{amsmath}
\usepackage{amsfonts}

\usepackage{fontspec}
\usepackage{polyglossia}

\usepackage{graphicx}
\usepackage{color}
\usepackage{url}
\usepackage{textpos}
\usepackage{xspace}
\usepackage{array}

\mode<presentation>
{
  \usetheme{Abumatranpres}
}


\graphicspath{{./fig/}}


\title{Omorfi experiences with crowds and sourcings\\
\scriptsize{CIFU XII Oulu, 2015}}
\author{Tommi A Pirinen \scriptsize \guilsinglleft{}tommi.pirinen@computing.dcu.ie\guilsinglright{}}
\institute{Ollscoil Chathair Bhaile Átha Cliath, ADAPT Centre\\
EU Marie Curie Abu-MaTran project}
\date{\today}

\begin{document}

\selectlanguage{english}

\maketitle

\begin{frame}
    \frametitle{Contents}
    \begin{itemize}
        \item omorfi backgrounds
        \item crowd sources
        \item current datasets
        \item issues
        \item (no MT here this time...)
    \end{itemize}
\end{frame}

\begin{frame}
    \frametitle{What's omorfi}
    \begin{itemize}
        \item omorfi (open morphology of Finnish) has grown to be a mature
            computational language description
        \item useful stuff with omorfi that I've been involved with past years
            \begin{enumerate}
                \item morphological analysis (2008)
                \item spell-checking and correction (2014)
                \item rule-based machine translation (2016?)
                \item statistical machine translation (2015)
                \item morphosyntactic disambiguation (2015)
            \end{enumerate}
        \item lot of it is not new in any way though, and others have done
            more innovative stuffs
    \end{itemize}
\end{frame}

\begin{frame}
    \frametitle{So why am I here}
    \begin{itemize}
        \item lot of omorfi work is about lexical data management, even
            building new apps is more so
        \item there's an endless amount of new data to gather, missing, etc.
        \item after all, most new apps and research questions start with
            reading and scraping word lists, examples and other data
        \item it'd be more nicely doable with crowds of people? After all,
            Finnish is well-resourced and all
    \end{itemize}
\end{frame}

\begin{frame}{What data do I need}
    \begin{itemize}
        \item get new words. Neither spell-checker nor machine
            translation can be good without `selfies' and sticks and whatnot
        \item semantics for lexemes: genders, animacies,
            locations, entities are very useful
        \item popularity: common word, rare, obscure
        \item style and usage: dialects, curse words, academic, computer,
            medicine
        \item people usually know a whole lot, we just need to write it down
            (it's not a lot of effort, I do it all the time)
    \end{itemize}
\end{frame}


\begin{frame}{Where do we get that}
    \begin{itemize}
        \item Published dictionaries (Nykysuomen sanalista was a kind of a beginning for omorfi)
        \item Academia, University of Helsinki research has produced lots of good lexical data
        \item Traditional crowd-sourcing suspects: Wiktionary is what I have
            tried so far, with some tests with OmegaWiki for translations,
        \item some less crowd-sourcy things: tatoeba for sentence translations,
            twitter for short messages, even Web as Corpora harvested,
            for examples and statistics. Known good words and sentences are
            a very good thing.
    \end{itemize}
\end{frame}

\begin{frame}{Issues with wiktionary}
    \begin{itemize}
        \item Data formatting can be very problematic (broken, unreadable)
        \item Data quality can be anything really (vandalised, non-native learner's mistakes)
        \item The information you are specifically wanting to get is usually missing
        \item ... and there's no way to get it into,``official article
            format''
        \item and the article format will change again tomorrow
        \item E.g. for morphological analyser new words would need correct
            paradigm numbers but wiktionary has usable ones in around 15~\%
    \end{itemize}
\end{frame}

\begin{frame}{Others}
    \begin{itemize}
        \item issues with wiktionary apply to others more or less
        \item Low traffic / usage rate (e.g. omegawiki)
        \item 
    \end{itemize}
\end{frame}

\begin{frame}{Where I would want to get / ideas}
    \begin{itemize}
        \item more casual contributors
        \item web pages specific to add lexical data
        \item games?
    \end{itemize}
\end{frame}

\begin{frame}{Thanks}
    Kiitos, aitäh, köszönöm etc.\\
    Questions, ideas?
\end{frame}

\end{document}
% vim: set spell:
