\documentclass[xetex]{beamer}

\usepackage{beamerthemesplit}

\usepackage{fontspec}

\usetheme{default}

\begin{document}

\title[Weighted Finnish Compounds with HFST-LexC]{Weighted Finite-State Morphological Analysis of Finnish Compounding with
\textsc{HFST-LexC}}
\author[Tommi Pirinen]{Tommi Pirinen, Krister Lindén\\ \texttt{tommi.pirinen@helsinki.fi}}
\date{2009-XX-XX}
\institute[University of Helsinki]{University of Helsinki\\ Department of General Linguistics}


\begin{frame}
\titlepage
\end{frame}

\begin{frame}
\frametitle{Finnish Compounding Analysis}
\begin{itemize}
\item Finnish nominal compounding allows arbitrary noun chains with final noun
inflected and other nouns in genitive or nominative form.
E.g. \emph{isä} (father), \emph{isänisä} (grandfather), \emph{isänisänisä}
(great grandfather) and so forth.
\item In lexicons, certain compounds can be lexicalized, i.e. treated as mainly non-compound unit. E.g. \emph{talonmies} (janitor) $\sim$
\emph{talon\#mies} (man of the house)
\item Productive compounding and lexicalization of compounds results in segmentational ambiguity, such as:
\begin{itemize}
\item \emph{paikassa} (in the place), \emph{pai\#kassa} (pie cash register)
\item \emph{isän\#isä} (grandfather), \emph{isä\#nisä} (father udder)
\item \emph{avaruus\#lentotukikohta} (space \# flight base), \emph{avaruuslento\#tukikohta} (space flight \# base)
\end{itemize}
\end{itemize}
\end{frame}

\begin{frame}
\frametitle{Disambiguation criteria}
\begin{itemize}
\item Most likely reading is the one with simplest structure
\begin{itemize}
\item We use number of word boundaries as metric of structural complexity
\item e.g. prefer \emph{paikassa} over \emph{pai\#kassa}
\end{itemize}
\item Most likely reading has most common words
\begin{itemize}
\item We use frequency of word forms in corpus as commonness of words in compound
\item e.g. usually prefer \emph{isän\#isä} over \emph{isä\#nisä}
\item e.g. preference between \emph{avaruuslento\#tukikohta} and 
\emph{avaruus\#lentotukikohta} may vary depending on corpora
\end{itemize}
\end{itemize}
\end{frame}

\begin{frame}
\frametitle{Learning Corpus Frequencies from Word Forms}
\begin{itemize}
\item<1-> Assume corpus size of $CS$ entries, including:
\begin{tabular}{l|r|l|r|l|r|}
talo & 1149 & talon & \alert<1>{1673} & talotta & 0 \\
mies & \alert<1>{6250} & miehen & 2998 & miehettä & \alert<1>{0} \\
\multicolumn{3}{l|}{talonmies} & \multicolumn{3}{r}{\alert<1>{68}} \\
\end{tabular}
\item<1-> We use probabilities of word tokens in corpora
to give weight to compound parts (converted with $-log()$ for tropical semiring):
\begin{itemize}
\item talonmies = $-log(\frac{\alert<1>{69}}{CS})$ {\uncover<2->{$ + 0$}}
\item talon\#mies = $-log(\frac{\alert<1>{1674}}{CS}) + -log(\frac{\alert<1>{6251}}{CS})$ {\uncover<2->{$ + \alert<2>{-log(\frac{1}{CS+1})}$}}
\item talon\#miehettä = $-log(\frac{\alert<1>{1674}}{CS}) + -log(\frac{\alert<1>{1}}{CS})$ {\uncover<2->{$ + \alert<2>{-log(\frac{1}{CS+1})}$}}

\end{itemize}
\item<2-> To account the structural penalty we add weight equal to (or greater than)
\alert<2>{$-log(\frac{1}{CS+1})$} per word boundary.
\end{itemize}
\end{frame}

\begin{frame}
\frametitle{Weighting Finnish Compounding in Lexc Lexicon Formalism}
\begin{tiny}
\begin{tabular}{l l l l}
LEXICON & \alert<1>{CompoundNonFinal} & &  \\
talo & \alert<1>{Compound} & "house, {\uncover<3->{\alert<3>{weight: -log(f('talo')/CS)}}}" & ; \\
talon & Compound & "house's, {\uncover<3->{\alert<3>{weight: -log(f('talon')/CS)}}}" & ; \\
& & & \\
LEXICON & \alert<1>{Compound} & & \\
\# & \alert<1>{CompoundNonFinal} & {\uncover<2->{\alert<2>{"weight: -log(1/(CS+1))"}}} & ; \\
\# & \alert<1>{CompoundFinal} & {\uncover<2->{\alert<2>{"weight: -log(1/(CS+1))"}}} & ; \\
& & & \\
LEXICON & \alert<1>{CompoundFinal} & & \\
talo+sg+nom:talo & \alert<1>{\#} & "house, {\uncover<4->{\alert<4>{weight: -log(f('talo+sg+nom')/CS)}}}" & ; \\
talo+sg+gen:talon & \# & "house's, {\uncover<4->{\alert<4>{weight: -log(f('talo+sg+gen')/CS)}}}" & ; \\
talo+sg+ill:taloon & \# & "in to the house, {\uncover<4->{\alert<4>{weight: -log(f('talo+sg+ill')/CS)}}}" & ; \\ 
\end{tabular}
\end{tiny}
\begin{itemize}
\item<1-> Any amount of compound initial forms, then word boundary, and final form with analyses
\item<2-> Assign maximum weight at word boundaries
\item<3-> Calculate surface form frequency weights for initial elements
\item<4-> Calculate analysis form weights for final elements
\end{itemize}
\end{frame}

\begin{frame}
\frametitle{Results}
\begin{itemize}
\item Comparing against results of another automatic disambiguating analyzer (i.e. not a gold standard)
\item Using only structural penalty for boundaries gives 99.96 \% precision and recall
\item Adding corpus probability penalties for compound parts or only corpus weights we found
virtually no disagreements with reference corpus, i.e. achieved 100 \% precision
and recall
\item Discarding the structural penalty and retaining only the corpus penalty the result stays at 100 \% precision and recall
\end{itemize}
\end{frame}

\begin{frame}
\frametitle{Related research}
\begin{itemize}
\item Similar research from other languages and methods:
\begin{itemize}
\item Anne Schiller (2005) \emph{German compound analysis with \emph{wfsc}}
\item Fred Karlsson (1998) \emph{Swetwol: A comprensive morphological analyzer for Swedish}
\item Jonas Sjörbergh and Viggo Kann (2004) \emph{Finding the correct interperatioon of Swedish compounds a statistical approach}
\end{itemize}
\end{itemize}
\end{frame}

\end{document}
