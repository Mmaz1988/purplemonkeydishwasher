\documentclass[a4paper,12pt]{article}

\usepackage{fontspec}
\usepackage{xunicode}
\usepackage{xltxtra}

\usepackage{url}
\usepackage{hyperref}

\usepackage[margin=3cm]{geometry}

\usepackage{natbib}

\linespread{1.3}

\setmainfont{Times New Roman}

\title{The Newest Morphological Analyser of Finnish Language}

\iffalse
\author{Tommi A Pirinen}
\fi

\begin{document}

\section{Introduction}

Computational morphological analysis is a central component for most of the
computational applications of linguistic analysis. The morphological analysis
for Finnish language was first described some 30 years ago~\cite{}. In this
article, I show a new system for computational analysis of Finnish language,
and go through some topics that recent development in the field of computational
linguistics have reflected to our analyser that differs from the existing
implementations. In computational linguistics, the beginning of 2000's has been
largely time for statistical language models and engineering. It is commonly
argued that statistical models are not as easily suitable for morphologically
complex languages like Finnish as they are for e.g. English. Our system is
based on the same assumption, and the core of the system is similar rule-based
system as described in earlier research of Finnish analysis. In this article
we show how we have integrated statistical features to traditional rule-based
morphological analyser of Finnish language that has been developed
earlier in University of Helsinki~\cite{pirinen2008}.

Another recent development in the computational language models is the concept
of \emph{maintainability} of these computational systems, e.g.
in~\cite{maxwell}. Specifically we will show how we use the power of
\emph{crowd-sourcing} to keep up with the new words, neologisms and other rare
words missing from dictionary. In particular we study the use of the popular
online dictionary Wiktionary as a source of additional lexical data. The
crowd-sourcing as well as few newer morphological phenomena in the language
have implications to lexicographical structure of the data as well, so we try
to describe in this article some of the newest findings based on the word data
we have added to lexical resources of our systems. While morphology of language
is quite resistant to change, variations such as quantitative consonant
gradation of the bleh stops have required new additions to existing
classifications. Furthermore we have re-analysed the generalisations of old
lexicographical classifications and noted that resulting system is more
favorable for computational systems as well as human classifiers to make
educated guesses when classifying unknown words---as well as verifying
classifications of existing lexical data.

The basic framework of the computational system we present here is largely
unchanged from the one introduced in~\cite{koskenniemi1983twolevel} and
followed-up on in~\cite{pirinen2008}, that is, finite-state automata. The main
technological difference is that we are now using weighted finite-state 
automata~\cite{openfst}, which practically means that we have capability to
express statistics or preference relations in our morphological dictionaries.
In this article we show how we have used existing methods and findings with
our Finnish morphological analyser.

The paper is organised as follows. blah blah blah

\section{Methods}

\section{Evaluations}

In this section we present the evaluation of our system as a full-fledged
morphological, and we also evaluate the lexicographical system we use for
the classification of new word-forms. To show how our new lexicographical
classification improves from the baseline of the contemporary dictionary
classification, we show a run of word-forms found in number of resources that
old dictionary deemed non-existing or rare.

\subsection{Dictionary Coverage}


\subsection{Classification Precision}


\subsection{Analysis of Word Forms Outside Dictionary Classification}


\section{Discussion}


\section{Conclusion}



% apalike with underscores???
\bibliographystyle{apalike}
\bibliography{omorfi2013}

\end{document}
% vim: set spell:
