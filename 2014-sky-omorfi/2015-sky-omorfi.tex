\documentclass[a4paper,12pt]{article}

\newif\iffinal
\finalfalse

\usepackage{fontspec}
\usepackage{xunicode}
\usepackage{xltxtra}
\usepackage{expex}
\usepackage{pdfpages}

\usepackage{url}
\usepackage{hyperref}

\usepackage[margin=3cm]{geometry}

\usepackage{natbib}
\setlength{\bibsep}{0pt}
\linespread{1.2}

\defaultfontfeatures{Mapping=tex-text}
\setmainfont{Times New Roman}

\usepackage{titlesec}

\titleformat*{\section}{\bfseries}
\titleformat*{\subsection}{\bfseries}


\title{Development and Use of Computational Morphology of Finnish in
the Open Source, Crowd-Sourced Era}
\iffinal
\author{Tommi A Pirinen \\
    Dublin City University\\
    ADAPT---School of Computing}
\fi
\date{}
\begin{document}

\maketitle
\begin{abstract}

    This squib describes a contemporary system for the computational modeling
    of the morphology of Finnish word-forms called omorfi. The purpose of this
    squib is to present new developments of the morphological analysis of
    Finnish to the linguistic audience, and to describe the key advances from
    the earlier analysers that might still be in use. The article describes
    omorfi as a full-fledged, stable system, as opposed to presenting a new
    technological feature in a limited experiment. We describe results from
    experimenting the system in large-scale linguistic analysis setting. We also
    present a selection of real-world use cases for the morphological analysers
    that have affected the development of omorfi.

    Our system is based fully on free and open source data and methods. This
    allows for some new approaches to the lexicographical data collection using
    crowd\-sour\-ced dictionary Wiktionary, as well as statistical modelling
    with e.g., the Wikipedia data.  One of the reasons of this article is to 
    describe the successful use of the community-driven development model in
    omorfi, including crowd-sourcing as well as independent expert
    contributions.

    We evaluated our analyser to give a rough idea of its usefulness and
    applications in linguistic work. To do this, we processed large text
    corpora with omorfi and verified that only 1~\% of the word-forms
    remains unknown to the system. Omorfi was also verified to match the
    FinnTreeBank 3.1 standard analyses at 90~\% faithfulness per token
    count.
\end{abstract}

\section{Introduction}

Computational morphological models are a central component for most of the
computational applications of linguistic analysis. Computational morphology
of the Finnish language was first described some 30 years
ago~\citep{koskenniemi1983twolevel}. The aim of this article is to present
omorfi\footnote{
\url{https://github.com/flammie/omorfi/}} as a matured scientific project involving contributions from
scientific community as well as crowd-sourced lexicographical additions, as
a full-fledged project for managing lexicographical database on one hand and
its natural language parser on the other hand. We will discuss our approach
to the task and examine the differences of our current approach to
traditional systems, and highlight some of the new development in the
parser, especially in the point of view of a system that is usable for
linguists and end-users. The new development that we discuss is mainly in
line with the recent trends in the field of computational linguistics. We
also present some of the current use-cases of the system.

This article is also a scientific documentation of the state-of-the-art
morphological analysis of Finnish. To that goal, we have made an attempt to
highlight the long-term design goals of the system instead of transitional
and volatile features of a fast-moving computer software that is developed
by a base of open-source and language enthusiasts.\footnote{for up-to-date
documentation of technical kind, the project web site is the place to go:
\url{https://github.com/flammie/omorfi/wiki}}.

The scientific advances within development of various features and facets of
omorfi has been documented in scientific publications in various fora. The
main advance to previous systems is the introduction of statistical language
parsing component (c.f.~\citet[for basic introduction]{manning1999foundations}). This has
some interesting implications to the analyses, especially as it is coupled
with the large coverage lexicon, e.g. while previously manual effort has
been spend to avoid productive compounding from creating non-sensical
readings, we now rely more on statistics to discount them. Omorfi's lexical
database has been partially made possible by another new development aspect,
that is, \textit{crowd-sourcing} lexicography via sites like
Wiktionary.\footnote{\url{http://fi.wiktionary.org}}

One notable practical distinction in our system is its licensing policy. Omorfi
analyser is a free and open source product.  In contemporary computational
linguistics, freeness of systems and data is rightly seen as a cornerstone of
properly conducted science, as it fulfills the requirement of repeatability by
not setting unnecessary fences for the repetition of the scientific results.
There is a large base of recent research supporting this, specifically for
Finnish the latest is by~\citet{koskenniemi2008build}. For computer-savvy end
users this means that the tools necessary to perform linguistic analysis with
omorfi can be downloaded to and used on any average PC. We have also one
installation in the data centers of CSC maintained by CSC
staff,\footnote{\url{http://www.csc.fi/english/research/sciences/linguistics}},
that is usable for researchers by application.

\section{Prior and Related Work}
\label{sec:prior-work}

Omorfi is based on solid tradition of writing rule-based computational
descriptions for Finnish. The theoretical framework was laid out by
\citet{koskenniemi1983twolevel}, and while the implementation is completely
unrelated and written from the scratch, omorfi was created in the context of
University of Helsinki, parallel to a project to update, open-source and
maintain similar software as the one used in
FITWOL~\citep{hfst2012}.\footnote{\url{http://hfst.sf.net}} Omorfi started  as
a master's thesis project~\citep{pirinen2008suomen} based on the newly released
open source word list from kotus at the time.\footnote{Nykysuomen sanalista,
\url{http://kaino.kotus.fi/sanat/nykysuomi}} Many of scientific advances made
by research groups in language technology department of University of Heĺsinki
have directly or indirectly affected omorfi. The research on sub-word
\(n\)-gram models~\citep{pirinen2009weighted,pirinen2009weighting} has been
transferred to omorfi compound disambiguation schemes. The methodology for
semi-automatic lexical data harvesting, e.g.
by~\citep{linden2008probabilistic}, has been largely influential on gathering
of the huge lexical database in omorfi. Finally, the work on coupling
statistical and rule-based approaches for
disambiguation~\citep{pirinen2015using}, based on a grammar and a parsing
approach by~\citet{karlsson1995constraint}, and is included in the recent
versions of omorfi.

There have been competing and complementary approaches to computational
parsing of Finnish. For example, in machine learning,
~\citet{durrett2013supervised}\footnote{We thank the anonymous reviewer for
bringing this recent research to our attention} show, that unsupervised
learning from Wiktionary data will create an analyser with recall in
prediction of inflected word-forms in the ballpark of 83---87~\%, However,
their goal was to learn to predict Wiktionary's example inflection table's
28 forms per noun and 53 forms per verb, and they only performed intrinsic
evaluation on held-out Wiktionary pages. Our approach to usage of Wiktionary
data is to collect the lexemes and their inflectional patterns already
confirmed and written down by human language users,\footnote{in our opinion,
trying to machine learn data that is already available and verified by
humans is not largely useful} and use hand-written rules to inflect, which
yields to virtually 100~\% recall (bar bugs in our code) for the full
paradigms. For this reason, it is hard to directly compare these two
approaches, at all. On the other hand, statistical language parsing systems
have been built on top of omorfi that go far and beyond the language parsing
capabilities of a morphological parser, such as the universal dependency
parser of Finnish̃~\cite{pyysalo2015universal}.

One source of development in related works is the applications, omorfi has been
used in many real-world scientific applications to handle Finnish language.
For example spell-checking~\citep{pirinen2014weighted}, language
generation~\citep{toivanen2012corpus}, machine
translation~\citep{clifton2011combining,rubino2015abumatran}, and statistical
language modelling~\citep{haverinen2013building,bohnet2013joint}. On top of
adding lexical data and statistical models, the vast array of applications has
necessitated for omorfi to take strong software engineering best common
practices in use, in order to keep different end-applications usable. This is
one of the key developments we wish to highlight in this article, the concept
of continuous development by cooperation of computer scientists, linguists and
common crowds via crowd-sourcing is as far as we know unique and
underdocumented for such a long-term free and open-source project as omorfi is.
Some of the prior notes to this mode of development with computational
linguistics has been documented by~\citet{maxwell2008joint}, and we have done
our best to adapt and extend it to large open source development setting
described in this article.




\section{Methods}
\label{sec:methods}

The implementation of our analyser follows the traditional works on finite-state
morphology by~\citet{beesley2003finite}. On top of that we have applied recent
extensions from the research of finite-state morphology, such as weighted
finite-state methods~\citep{openfst,hfst2012}. What this means in practice is
basic unigram probabilities of word-forms composed\footnote{using the
    finite-state algebra operation composition that is well-defined in terms of
likelihood-based automata} over the analyser from a corpus.  This brings the
traditional rule-based language analyser towards the statistical language
analysers that are widely popular in the handling of morphologically less
complex languages. A diagram of the combination is shown in 
figure~\ref{fig:combo}, the figure is much simplified version of real
implementation, just to show how few forms of select words interact in the system, the
statistical component also omits the existence of known compounds to simplify
the presentation. The flow of the system is following: from database we generate
a rule based analyser, which is created as a finite.state automaton using
standard \textit{Finite State Morphology}~\citep{beesley2003finite}. The
statistical data is counted from the corpora, and applied over the automaton
using the formula by~\cite{pirinen2009weighted}.

\begin{figure}[tb!]
    \includegraphics[scale=0.9,clip,trim=0 300 0 0]{combo-crop}
    \caption{Diagram of omorfi technology showing one word example for
        database, analyser and statistical training. In finite-state
        representation, the double circle marks the end state, and the arrow
        leading away from the figure is cropped out of the example. The
        letter automata format is combined to longer strings to compact
        the representation.
    \label{fig:combo}}
\end{figure}

The implementation of finite-state morphology in omorfi is based on
arrangement of stems, stem variations and suffix morphs, without
intermediate morphographemic processing. This relies on word classification
to include data about stem patterns and vowel harmony for example. The
classified dictionary words are stripped of their varying stem parts, and
then concatenated with the variations and then stems, followed by all
suffixes and optionally extended by compounding. This is done using standard
finite-state morphology approach. E.g.  in figure~\ref{fig:combo} we have a
dictionary words \textit{vesi} `water' and \textit{käsi} `hand' with stem
invariants \textit{ve-} and \textit{kä} resp., and stem variation in
\textit{-si} $\sim$ \textit{-de-} \ldots, and respectively suffixes
$\emptyset$ (nominative) $\sim$ \textit{-n} (genitive, `water's') $\sim$
\textit{-ssä} (inessive `in water')  $\sim$ \textit{-stä} (elative, ´from
water') \ldots and so forth. This simple concatenation forms altoghether
some thousands of word-forms per dictionary word, as well as returns back to
new words for compounding where applicable.

The baseline statistical methods for morphological models are applied over
the finite-state formulation within the same framework, as is shown in the
example in figure~\ref{fig:combo}.  The formulation we use is the schoolbook
unigram training (c.f.~\cite{manning1999foundations}): get the likelihood $P(w)$ for the
surface form $w$, by counting the amount of word-forms $f(w)$ in a corpus
and divide it by the number of word-forms in the whole corpus $CS$: $P(w) =
\frac{f(w)}{CS}$.  To get around the problems with the probability of $0$
for unseen word-forms, we use additive smoothing~\citep{chen1999empirical},
which estimates frequency of each type as $1$ larger than it is and the size
of corpus as number of types larger $P(\hat w) = \frac{f(w) + 1}{CS + TC}$,
where $TC$ is a type count. The acquired likelihoods are combined to the
finite-state morphological analyser producing a weighted finite-state
morphological analyser capable of producing both analyses and their
likelihoods as shown in the last frame of figure~\ref{fig:combo}.

The current version of omorfi at the time of writing provides an application
of surface weights like these as an option, given a suitable corpus is
provided during compilation. Similarly given a corpus of disambiguated
analyses, the same form of probabilities could be applied on the analyser
for the full analysis strings, or part of them in similar manner to create a
baseline disambiguating analyser. The inclusion of statistically weighted
analyses in omorfi versions may be dependent on user obtaining suitable
training corpus, by default and in the experiments of this article, omorfi
includes rough manually estimated likelihoods. With both methods, the
analysis can be disambiguated using likelihood-aware constraint
grammars~\cite{pirinen2015using}.

\section{Data}
\label{sec:data}

There are a few freely available open resources for lexicographical data of
Finnish. The first one  we used is based on a dictionary maintained by
(Research) institute of languages in Finland (RILF). The lexicographical data
of the dictionary has been available under free software licence LGPL since
2007 as \textit{Nykysuomen
sanalista.}\footnote{\url{http://kaino.kotus.fi/sanat/nykysuomi}}.  The second
source of lexical data we acquired from the internet is a free, open source
database named
\textit{Joukahainen.}\footnote{\url{http://joukahainen.puimula.org}}  For another
source of lexical data we use the popular crowd-sourced \textit{Wiktionary}
project.  Finally, we have used data that has been produced for various
projects at university of Helsinki, including other lexical projects, such as
FinnWordNet~\citep{linden2010finnwordnet}.\footnote{The primary sources for
lexical data used in these projects include Europarl corpus and English
Wordnet.  The overlap between this and our experimental evaluation data is
minimal.} 

The full lexical data used in the current version of our analyser consists of
396,673 classified lexemes, the classification is roughly summarised by word's
classes and their origin in table~\ref{table:lexical}. 

\begin{table}
    \begin{scriptsize}
  \centering
    \begin{tabular}{|l|r|r|r|r||r|}
        \hline
        \bf Database: & Kotus & Joukahainen & Wiktionary & UH & \bf Total \\
        \bf Class   & & & & & \\
        \hline
        Adjectives     & 10,537 & 652   & 59    & 6,368   & 17,616  \\
        Nouns          & 67,608 & 2,945 & 1,133 & 1,589   & 73,275  \\
        (Proper nouns) & 20     & 5,394 & 70    & 262,330 & 267,814 \\
        Verbs          & 9,685  & 476   & 12    & 499     & 10,672  \\
        Others         & 6620   & 5     & 12    & 25,536  & 32,173  \\
        \hline
        \bf Total      & 94,290 & 9,472 & 1,286 & 296,322 & 396,637 \\
        \hline
    \end{tabular}
  \caption{Lexical data used in the analyser.  The columns are the lexical
      sources used: Nykysuomen sanalista in column titled \textit{Kotus} (RILF),
      \textit{Joukahainen}, \textit{Wiktionary}, terms harvested in various
      research project in University of Helsinki in column titled \textit{UH},
      and a total sum in the last column. The rows represent coarse
      morphological classification, with proper nouns separated into their own
      row. On the \textit{others} row are the adverbs, adpositions and such words
      with defective paradigms. These other words in current dictionaries have
      one entry per inflectional ending.
  \label{table:lexical}}
  \end{scriptsize}
\end{table}


\section{Experimental Setup and Evaluation}
\label{sec:evaluation}

In this section we evaluate omorfi to give an impression of its usefulness
in various tasks and potential caveats when using for linguistic research.
For evaluation we use only freely available corpora. The size of the corpora
is detailed in table~\ref{table:corpora}. Following corpora are included:
ebooks of project Gutenberg\footnote{\url{http://gutenberg.org}}, the data
of Finnish Wikipedia\footnote{\url{http://fi.wikipedia.org}}, and the JRC
acquis corpus\footnote{\url{http://ipsc.jrc.ec.europa.eu/index.php?id=198}}.
For downloading and preprocessing these corpora we use freely available
scripts~\footnote{\url{https://github.com/flammie/bash-corpora}}. The
scripts retain most of the punctuation and white-space as-is. The resulting
token counts are given in table~\ref{table:corpora}. Some further tests were
made with fully tokenised and analysed
FinnTreeBank~\citep{voutilainen2012specifying} version 3.1.  The scripts
used for this evaluation are part of omorfi source code and are usable for
anyone.

\begin{table}
    \begin{scriptsize}
  \centering
    \begin{tabular}{|l|r|r|}
        \hline
        \bf Feature: & Tokens     & Types     \\
        \bf Corpus   &            &           \\
        \hline
        Gutenberg    & 36,743,872 & 1,590,642 \\
        Wikipedia    & 55,435,341 & 3,223,985 \\
        JRC Acquis   & 42,265,615 & 1,425,532 \\
        FTB 3.1      & 76,369,439 & 1,648,420 \\
        \hline
    \end{tabular}
  \caption{Corpora used for evaluations. Tokens are all strings extracted from
      corpus and types are unique strings, both include punctuation and some
      codified expressions like URLs, addresses etc.
  \label{table:corpora}}
  \end{scriptsize}
\end{table}

First we measure how big proportion of data in the material are
out-of-vocabulary items. This gives us naive coverage, formally defined as
$\mathrm{Coverage} = \frac{\mathrm{Analysed}}{\mathrm{Corpus size}}$.
The results are presented in table~\ref{table:coverage} for all the
corpora we have.

\begin{table}
    \begin{scriptsize}
    \centering
    \begin{tabular}{|l|r|r|r|r|}
        \hline
        Corpus:            & \bf Gutenberg & \bf Wiki & \bf JRC acquis & \bf FTB \\
        \hline
        Coverage (tokens): & 96.4~\%       & 91.7~\%  & 93.4~\%        & 96.5~\% \\
        Coverage (types):  & 72.2~\%       & 60.8~\%  & 59.4~\%        & 72.2~\% \\
        \hline
    \end{tabular}
    \caption{Naive coverages when analysing common corpora
    \label{table:coverage}}
  \end{scriptsize}
\end{table}

Faithfulness is measured as a proportion of equal analyses, formally $\mathrm{Faithfulness} =
\frac{\mathrm{Matched}}{\mathrm{Correct} + \mathrm{Missing}}$. In
table~\ref{table:quality} we show the results for FTB3.1 corpus and
analyses, first by proportion of all tokens in data then by unique tokens.

\begin{table}
    \begin{scriptsize}
    \centering
    \begin{tabular}{|l|r|r|}
        \hline
        \bf Corpus & \bf Faithfulness \\
        \hline
        FTB (tokens) & 90.1~\% \\
        FTB (types)  & 48.0~\% \\
        \hline
    \end{tabular}
    \caption{The amount of FTB3.1 analyses omorfi can analyse with exact
    match in results, from FTB3.1 reference corpus\label{table:quality}}
  \end{scriptsize}
\end{table}

The sizes and processing speeds for automata built from the data described in
table~\ref{table:lexical} using Debian packaged HFST software
version~3.8.2\footnote{\url{http://wiki.apertium.org/wiki/Prerequisites_for_Debian}}
on a Dell XPS 13 laptop are given in table~\ref{table:size-speed}. The speed
was averaged over three runs using 1 million first tokens from europarl.

\begin{table}
    \begin{scriptsize}
        \centering
        \begin{tabular}{|l|r|}
            \hline
            \bf Feature & \bf Value \\
            \hline
            Size & 52M \\
            Speed & 955 wps\\
            \hline
        \end{tabular}
        \caption{Size of omorfi analyser as measured by \texttt{ls -lh}, speed
        of analysis using hfst-lookup in words per second \label{table:size-speed}}
    \end{scriptsize}
\end{table}

This result is in line with previous research on speed of optimised finite-state
automata in natural language processing by~\citet{silfverberg2009hfst}.

\section{Discussion and Future Work}
\label{sec:discussion}

The techniques of statistical language parsing in omorfi are quite modest at
modern standards, while successful combination of statistical parsing and
rule-based disambiguation is state-of-the-art, it would be interesting to
see the effects of more representative corpora used with different methods
to parsing quality of omorfi.  In particular, it would be interesting to see
some development on an end-user application that necessitates a use of
high-quality disambiguated morphological analyses. We expect that
development towards universally recognised and comparable linguistic
resources by projects like UPOS and UD will be crucial to future development
of omorfi to the direction of state-of-the-art language processing.

One of the key components to recent success of omorfi that this article also
outlines is its adaptability and usefulness over various end uses.  While it
seems from the number of end users that it is in fact possible for
independent researchers to use and develop omorfi, it would be interesting
to see how more linguists and lexicographers using omorfi might benefit the
description as well as end applications.

\subsection{Error Analysis}

The coverage of the analyser is systematically around 98~\%, this is
virtually at the upper limits of reasonable results with the given corpora.
This can be noticed by analysing the errors or the out-of-vocabulary
word-forms left in the current corpora. For Wikipedia, we get word-forms
like: \textit{Lä}, \textit{of}, \textit{The}, \textit{amp}, \textit{HMS},
\textit{and}, \textit{jpg}, that is, mainly code notations, foreign words
and abbreviations. In the Gutenberg corpus, we get, among some missing
proper nouns, forms like: \textit{nämät}, \textit{kauvan}, \textit{sitte}
and other dialectal, spoken, archaic or otherwise non-standard forms. Of
course these are all plausible additions for morphological analyser as well.
This shows a well-known effect of Zipfian distribution of language data:
since rare word-forms and phenomena get exponentially rarer, the effect of
collecting and classifying further lexemes will be insignificantly small
(compare to~\cite{manning2011part}). For applications requiring higher, potentially
100~\% coverage, using guessing techniques, e.g. \citet{}, should be
investigated.

The measurements of faithfulness we used in the evaluation section are based on
a ``gold standard'' which is merely a result of another automatic analyser.
This is not necessarily a representative of correctly analysed Finnish and thus
we have not regarded getting maximal faithfulness as a priority. As an example
of mismatched analyses right now: top wrong word-forms \textit{oli} `was',
\textit{olivat} `were' are analysed as present tense in the gold standard, we
feel this is incorrect and have opted not to match that analyses. A large part
of mismatches comes from analysis of compounds and derivations, whether they
are lexicalised or still treated as combinations of the root words, this can be
fixed by developing the lexicon.  Also, if one wishes to obtain 100~\%
faithfulness of FTB 3.1, the easiest path is to compile all the analyses from
the corpus into secondary automaton and use that along omorfi, although it is
unclear what the use case would be, beyond reproducing the mistakes contained
within that corpus.

It is noteworthy that in an open source project with multiple active
developers, importance of the evaluations is somewhat diminished by the
relatively active development pace. That is, if there are some missing analyses
or low score is a problem, it will likely be improved or fixed in a short
time-frame. For example, the faithfulness measure has been varying in range of
80---93~\% during the preparations of this article.


\section{Conclusion}
\label{sec:conclusion}

In this squib we set out to present a new fully open source Finnish
morphological lexicon. We confirm that it is full-fledged and mature lexical
database that can be used as a baseline morphological analyser with large
coverage, suitable for linguistic research, as well as in external applications
such as spelling correction and rule-based machine translation. We have shown
some approaches that make available use of modern natural language processing
techniques like statistics in conjunction with analysers built from our data
and paved a way forward for researchers interested those topics. We also
provide some easy-to-access ways for linguists and researchers to use and
extend our database via publicly maintained servers and crowd-sourced web-based
services.

% apalike with underscores???
\bibliographystyle{apalike}
\bibliography{skyomorfi2015}
\iffinal
\newpage
\begin{small}
    \noindent Contact Information:\\
\\
    Tommi A Pirinen\\
    Ollscoil Chathair Bhaile Átha Cliath\\
    Baile Átha Cliath 9\\
    Éire\\
    e-mail: \url{Tommi.Pirinen@computing.dcu.ie}\\
\end{small}
\fi
\end{document}
% vim: set spell:
