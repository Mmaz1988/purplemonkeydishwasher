\documentclass{beamer}


\usepackage{amssymb}
\usepackage{amsmath}
\usepackage{amsfonts}

\usepackage{fontspec}
\usepackage{polyglossia}

\usepackage{graphicx}
\usepackage{color}
\usepackage{url}
\usepackage{textpos}
\usepackage{xspace}
\usepackage{array}

\mode<presentation>
{
  \usetheme{Abumatranpres}
}


\graphicspath{{./fig/}}


\title{From WFST to VISL-CG 3? Some
    experiments\\
\scriptsize{in Vilnius CG workshop}}
\author{Tommi A Pirinen \scriptsize \guilsinglleft{}tommi.pirinen@computing.dcu.ie\guilsinglright{}}
\institute{Ollscoil Chathair Bhaile Átha Cliath, ADAPT Centre\\
EU Marie Curie Abu-MaTran project}
\date{\today}

\begin{document}

\selectlanguage{english}

\maketitle

\begin{frame}
    \frametitle{Contents}
    \begin{itemize}
        \item weighted FSTs
        \item example use cases
        \item current experiments
        \item issues
    \end{itemize}
\end{frame}

\begin{frame}
    \frametitle{Why did I do it? A little background}
    \begin{itemize}
        \item omorfi (open morphology of Finnish) has grown to be a mature
            description
        \item Fred Karlsson agreed to release Finnish CG rules open source
        \item however, tag sets and analyses don't quite agree
        \item it appears that CG rules built for morphology with smaller
            lexicon and less ambiguity don't work as well with bigger lexicon
            (we have 100,000's new proper nouns and some neologisms, dialects
            and stuff)
    \end{itemize}
\end{frame}


\begin{frame}
    \frametitle{Weighted finite-state morphological analysis}
    \begin{itemize}
        \item traditional finite-state morphology from Beesley \& Karttunen
            with HFST for weight handling
        \item weights can be added surface : analysis pairs by three methods
        \begin{itemize}
            \item train analysis (or lemmas etc.)
            from gold corpora that happens to have tags
            that match yours... e.g., $P('N Sg Nom') > P('N Pl Ins PxSg1 Kin')$
            \item or approximate probabilities using linguistic skills
            \item train surface probabilities from plain text corpora, e.g. 
            $P('\mathrm{on}') > P('\mathrm{morfologia}')$ (doesn't matter in
            CG application)
        \end{itemize}
        \item and number of variations and combinations of above
    \end{itemize}
\end{frame}

\begin{frame}
    \frametitle{Example with toy weights}
    \emph{Löysin sinut}
    \begin{tabular}{llr}
        Löysin & löysä ADJECTIVE SUP SG NOM & 0,000000 \\
Löysin & löytää VERB ACT INDV PAST SG1 & 0,000000 \\
Löysin & löysä ADJECTIVE PL INS & 2,000000 \\
Löysin & löysä NOUN PL INS & 2,000000 \\
Löysin & Löysä NOUN PROPER PL INS & 2,000000 \\

sinut & sinut ADVERB & 0,000000 \\
sinut & sinä PRONOUN PERSONAL SG2 PL NOM & 1,000000 \\
sinut & sinä PRONOUN PERSONAL SG2 SG ACC & 0,000000 \\
    \end{tabular}
    ``I found you''
\end{frame}

\begin{frame}
    \frametitle{Translations depending on disam}
    most I've seen in apertium during recoding of the CG...
    \begin{itemize}
        \item I found you
        \item loosest you
        \item with looses you
        \item with Löysäs you
        \item I found even
        \item I found yous
        \item etc.
    \end{itemize}
\end{frame}

\begin{frame}
    \frametitle{Why weights in CG}
    \begin{itemize}
        \item Tie-break decisions in spots that the rules don't yet
        \item most pipelines (outside FSM tradition) operate on n-best
            lists with weights
        \item Binary disambiguation leads to worse error propagation
        \item removing readings that are needed further down the pipeline
            may not be a good idea
        \item they seem to fill in nicely when rewriting external cg for
            another system by hand
        \item statistics are a good source of information for rules too
    \end{itemize}
\end{frame}

\begin{frame}
    \frametitle{How weights in CG}
    \begin{itemize}
        \item CG has support for integer tags that has been already
            used for such (\texttt{<W=1>}) -> move FSM weights into
            that tag
        \item \texttt{SELECT (<W=MIN>)} is equivalent to baseline
            weighted finite-state algorithm
        \item turn disambiguations into weight adding by post-processing
            the traces,
        \item Problems? iterative cg process is not very ideal for weighting as is.
            the values used for additional weights need to be selected
            somehow (line number, keyword named rules?)
    \end{itemize}
\end{frame}

\begin{frame}
    \frametitle{Example converted to VISL-CG 3 with weights}
    \begin{tabular}{ll}
"<Löysin>" & \\
 "Löysä" & NOUN PROPER PL INS <W=200> \\
 "löysä" & <*> ADJECTIVE SUP SG NOM <W=0> \\
 "löytää" & <*> VERB ACT INDV PAST SG1 <W=0> \\
 "löysä" & <*> ADJECTIVE PL INS <W=200> \\
 "löysä" & <*> NOUN PL INS <W=200> \\
 "löysä" & <*> ADJECTIVE SUP SG NOM <W=0> \\
"<sinut>" & \\
 "sinut" & ADVERB <W=0> \\
 "sinä" & PRONOUN PERSONAL SG2 PL NOM <W=100> \\
 "sinä" & PRONOUN PERSONAL SG2 SG ACC <W=0> \\
\end{tabular}
\end{frame}

\begin{frame}
    \frametitle{Example of disambiguation trace}
    \small
    \begin{tabular}{ll}
        "<Löysin>" & \\
        "löytää" & VBLEX ACT INDV PAST SG1 <W=0> SELECT:2620:SURELY-vfin \\
        ;	"Löysä" & NOUN PROPER TOP PL INS <W=200> REMOVE:797 \\
        ;	"löysä" & ADJECTIVE POS PL INS <W=200> SELECT:2620:SURELY-vfin \\
        ;	"löysä" & ADJECTIVE SUP SG NOM <W=0> SELECT:2620:SURELY-vfin \\
        ;	"löysä" & NOUN PL INS <W=200> SELECT:2620:SURELY-vfin \\
        "<sinut>" & \\
"sinä" & PRONOUN PERSONAL SG2 SG ACC <W=0> \\
        "sinä" & PRONOUN PERSONAL SG2 PL NOM <W=100> \\
        ;	"sinut" & ADVERB <W=0> REMOVE:2631:RARELY-sinut \\
    \end{tabular}
\end{frame}

\begin{frame}
    \frametitle{Join traces back to weights and reorder }
    \begin{tabular}{llr}
löysin & löytää VERB ACT INDV PAST SG1 & 0,000000 \\
Löysin & Löysä NOUN PROPER PL INS & 2,079700 \\
löysin & löysä ADJECTIVE SUP SG NOM & 10,262000 \\
löysin & löysä ADJECTIVE PL INS & 12,262000 \\
löysin & löysä NOUN PL INS & 12,262000 \\

sinut & sinä PRONOUN PERSONAL SG2 SG ACC & 0,000000 \\
sinut & sinä PRONOUN PERSONAL SG2 PL NOM & 1,000000 \\
sinut & sinut ADVERB & 1,2631000 \\
    \end{tabular}
    $W=W_{\mathrm{original}} + W_{\mathrm{KEYWORD}} + W_{\mathrm{line number}}$
\end{frame}

\begin{frame}
    \frametitle{Questions, suggestions?}
    Thanks, Ačiū
\end{frame}

\end{document}
% vim: set spell:
