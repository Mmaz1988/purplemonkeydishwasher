%
% File nodalida2015.tex
%
% Contact beata.megyesi@lingfil.uu.se
%
% Based on the instruction file for EACL 2014
% which in turn was based on the instruction files for previous 
% ACL and EACL conferences.

\documentclass[11pt]{article}
\usepackage{nodalida2015}
%\usepackage{times}
\usepackage{mathptmx}
%\usepackage{txfonts}
\usepackage{url}
\usepackage{latexsym}
\special{papersize=210mm,297mm} % to avoid having to use "-t a4" with dvips 
%\setlength\titlebox{6.5cm}  % You can expand the title box if you really have to

\title{What can comparing baseline SMT and RBMT for non-related languages tell us about languages}

\newif\ifpublished
\publishedfalse

\ifpublished
\author{Tommi A Pirinen \\
  Ollscoil Chathair Bhaile Átha Cliath \\
  CNGL --- School of Computing \\
  Dublin City University, Dublin 9 \\
  {\tt tommi.pirinen@computing.dcu.ie} \\\And
  Second Author \\
  Affiliation / Address line 1 \\
  Affiliation / Address line 2 \\
  Affiliation / Address line 3 \\
  {\tt email@domain} \\}
\fi

\date{\today}

\begin{document}
\maketitle
\begin{abstract}
  In this paper we report an experiment on rapid building of machine
  translation systems for three language pairs that are not of the same
  family. It is well known that rule-based machine translation is best
  suited to closely related languages and statistical machine translation
  also mainly works well with reasonably closely related languages. In this
  paper we have selected three languages from different families none of
  which is English to show highlight how to build new language models statistical
  and rule-based paradigms rapidly and how looking at not English data is
  interesting. We have selected for our use case English from the
  indo-european family, Lithuanian from the Slavic family and Finnish from
  the Uralic family. 
\end{abstract}

\section{Introduction}




\bibliographystyle{acl}
\bibliography{nodalida2015}


\end{document}

% vim: set spell:
