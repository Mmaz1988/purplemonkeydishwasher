\documentclass[11pt]{article}
%\documentclass{flammie}
\usepackage{nodalida2015}
%\usepackage{times}
\usepackage{mathptmx}
%\usepackage{txfonts}
\usepackage{url}
\usepackage{latexsym}

\usepackage{fontspec}

\special{papersize=210mm,297mm} % to avoid having to use "-t a4" with dvips
%\setlength\titlebox{6.cm} % You can expand the title box if you really have to

\title{Using weighted finite state morphology with VISL CG-3---Some experiments
    with free open source Finnish resources}

\newif\ifpublished
\publishedfalse


\ifpublished
\author{Tommi A Pirinen\\
Ollscoil Chathair Bhaile Átha Cliath\\
CNGL---School of Computing\\
Dublin City University, Dublin 9\\
{\tt tommi.pirinen@computing.dcu.ie}
}
\fi

\date{\today}

\begin{document}

\maketitle
\begin{abstract}
    Traditionally, the coupling of finite state morphology and constraint
    grammar has been strictly  rule-based, making binary distinctions
    between allowed and disallowed readings, however, in the recent years much
    of the research in the finite state morphologies has adapted the
    contemporary paradigm of statistically weighted analysis. This is reflected
    in current versions of free and open source morphology of Finnish, omorfi, in
    the finite state morphology part. In this paper we examine two strategies
    of making use of the weights as a part of VISL CG-3 pipeline.
    We evaluate the results intrinsically on small sample of analyses we
    have disambiguated by hand ourselves, and extrinsically on the effect it
    has to the rule-based machine translation of that text using freely
    available open source translator, apertium-fin-eng. 
\end{abstract}

\section{Introduction}

In the recent years, use of statistical information in computational
linguistics has gained much interest, with systems like hunpos, moses etc.
being the main points of interest of most research in the field. In finite
state morphology as well as constraint grammars, extensions to handle
probabilities are recent and scarcely
documented~\cite{pirinen2009weighting,bick2009introducing}.  In this paper we
experiment with an existing weighted finite state morphology of
Finnish~\cite{pirinen2011modularisation}\footnote{\url{https://github.com/flammie/omorfi/}}
with VISL CG-3. For CG we have adapted Fred Karlsson's Finnish CG rules to
omorfi's tag set, however, the rules were written for completely different
analyser, which results in relatively low quality and high level of ambiguity
at the current level. We estimate that salvaging these rules for the current
version of analysis would require a substantial re-writing effort. In the
meanwhile, there are a lot of easy targets that correctly trained statistical
analyser can already deal with without extra effort. E.g., one large 
difference we assume between the analyser these CG rules were written for and
omorfi's are that omorfi contains a huge number of proper nouns, dialectal
and sub-standard forms, and rare language, animal etc. names, that are left
ambiguous. It is obvious for a human reader that these words are very
unlikely and given most corpora we expect them to be highly penalised as
well.

The main goal of this experiment is to create a functional
pipeline out of weighted finite-state analysis and current version of the
constraint grammar. There are obvious conflicts between the statistically
driven ranked hypotheses approach and the strictly deleting approach of the
current constraint grammar, which may limit usefulness of our current method
of combining these two information sources. 

\section{Methods}

The probabilities in omorfi are provided by the underlying system as a floating
point number based on the finite-state implementation of a tropical semiring.
This weight can be based on negative logarithms of probabilities of the
word-forms, lemmas, analyses etc., as well as linguist-defined arbitrary
values. The typical method to get probabilities for the analyses is to
count the probabilities as the proportion of the analyses among all analysed
tokens in the corpus. 

%however the calculation of probabilities of
%the analyses and lemmas would require a high quality freely available gold
%corpus, which we do not currently have. For this reason,
We use the default
setting which is based on approximated tag likelihoods. Since VISL CG 3 does
not support floating point numbers, our python-based analysis script, which
outputs CG-compatible format, multiplies the weights by a 100 before rounding
them down and turning into a tag of form \texttt{<W=\emph{weight}>}, where
\emph{weight} weight is the multiplied weight. This is sufficient for the
coarse weights that default analyser produces.

\section{Experimental Setup}

For analysis we use the python API to omorfi version 20150326, to turn the
analyses into the format understood by VISL CG 3. We use a version Fred
Karlsson's Finnish CG found in apertium's
repository.\footnote{\url{http://sourceforge.net/p/apertium/svn/HEAD/tree/nursery/apertium-fin-eng/apertium-fin-eng.fin-eng.rlx}},
with the tag set manually converted to match omorfi's,\footnote{URL hidden for
reviews} howeve, given the amount of ambiguous names of tags and sets and lists
in the grammar, there may be some conversion errors left.  The system is tested
with VISL CG 3 version 0.9.8.10063.

To test the functionality of the our combination of weighted
finite-state analyser and VISL CG-3, we analyse a short text that we have
manually disambiguated and measure the quality of analyses. The source of the
text is found in the apertium's SVN
repository.\footnote{\url{http://sourceforge.net/p/apertium/svn/HEAD/tree/nursery/apertium-fin-eng/texts/tarina.fin.text}}
For the purpose of this experiment, we have manually tokenised the text
before processing it with omorfi.
In addition of analysis we use the results of analyses in apertium's
Finnish-English machine translator, and measure the translation quality. This
way we can ensure that the gold annotation has not been selected to best fit
our results but is actually the semantically most fitting one.

\section{Evaluation}

We evaluated the resulting analysers by analysing the text of the story and
measuring the proportion of correct guesses from all analyses that were left.
For unweighted test the results show 73 \% precision while the weighted test
gives us 77 \%, giving a boost of 4 \% units.

\section{Conclusion}

We have implemented a VISL CG-3 output on top of existing weighted finite-state
analysis of Finnish language and tested that it works combined with VISL CG-3.
We have successfully included this combination as a part of apertium machine
translation pipeline. We note that weighted finite-state analysis can be
easily combined with VISL CG 3 and results in an increased accuracy.


\bibliographystyle{acl}
\bibliography{nodalida2015}

\end{document}
% vim: set spell:
