\documentclass{beamer}


\usepackage{amssymb}
\usepackage{amsmath}
\usepackage{amsfonts}

\usepackage{fontspec}
\usepackage{polyglossia}

\usepackage{graphicx}
\usepackage{color}
\usepackage{url}
\usepackage{textpos}
\usepackage{xspace}
\usepackage{array}


\graphicspath{{./fig/}}


\title{Combining WFST morphology and VISL CG-3\\
\scriptsize{in CG workshop, Vilnius, 2015}}
\author{Tommi A Pirinen \scriptsize \guilsinglleft{}tommi.pirinen@computing.dcu.ie\guilsinglright{}}
\institute{DCU, ADAPT--CNGL\\Abumatran}
\date{\today}

\begin{document}

\selectlanguage{english}

\maketitle

\section{Introduction}

\begin{frame}
Traditional disambiguation is error propagation.

Stuff.

Maybe example from apertium world.
\end{frame}

\begin{frame}
    \frametitle{WFST morphology}
    \begin{itemize}
        \item Finite State Morphology, e.g., using HFST lexc or apertium
        \item weights added to analyser ... 
            set in tropical semiring weight structure i.e., as penalties
            per path = analysis
        \item Statistical weights from analysed, disambiguated reference
            corpora as $-\log P(w)$, where $w$ is word-form
        \item Linguist-defined weights based on the morphology, e.g.,
            for each compound boundary or derivation add 123
    \end{itemize}
\end{frame}

\begin{frame}
    \frametitle{Weights in VISL CG-3}
    \begin{itemize}
        \item numeric tags
        \item SELECT <MIN>
        \item trace
        \item post-processing
    \end{itemize}
\end{frame}



\end{document}
% vim: set spell:
