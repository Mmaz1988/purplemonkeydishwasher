%
% File nodalida2015.tex
%
% Contact beata.megyesi@lingfil.uu.se
%
% Based on the instruction file for EACL 2014
% which in turn was based on the instruction files for previous 
% ACL and EACL conferences.

\documentclass[11pt]{article}
\usepackage{nodalida2015}
%\usepackage{times}
\usepackage{mathptmx}
%\usepackage{txfonts}
\usepackage{url}
\usepackage{latexsym}
\special{papersize=210mm,297mm} % to avoid having to use "-t a4" with dvips 
%\setlength\titlebox{6.5cm}  % You can expand the title box if you really have to

\usepackage{fontspec}
\usepackage{xltxtra}

\title{Omorfi---Free and open source morphological lexical database for Finnish}

\newif\ifpublished
\publishedfalse

\author{Tommi A Pirinen \\
  Ollscoil Chathair Bhaile Átha Cliath \\
  ADAPT Centre --- School of Computing \\
  Dublin City University, Dublin 9 \\
  {\tt tommi.pirinen@computing.dcu.ie}  \\}

\date{\today}

\begin{document}
\maketitle
\begin{abstract}
    This demonstration presents a freely available open source lexical database
    omorfi. Omorfi is a mature lexicographical database project, started out as
    a single-person single-purpose free open source morphological analyser
    project, omorfi has since grown to be used in variety of applications
    including spell-checking, statistical and rule-based machine translation,
    treebanking, joint syntactic and morphological parsing, poetry generation,
    information extraction. In this demonstration we hope to show both the
    variety of end-user facing applications as well as the tools and interfaces
    for computational linguists to make the best use of a developing product.
    We show an shallow database arrangement that has allowed a great variety
    of contributors from different projects to extend the lexical database
    while not breaking the continued use of existing end-applications.
    We hope to show both the best current practices for lexical data management
    and software engineering with regards to continuous external project
    integration of a constantly developing product. As case examples we show
    some of the integrations with following applications: Voikko spell-checking
    for Windows, Mac OS X, Linux and Android, statistical machine translation
    pipelines with moses, rule-based machine translation with apertium and
    traditional xerox style morphological analysis and generation. 
    morphological segmentation, as well
    as application programming interfaces for python and Java.
\end{abstract}

\section{Introduction}

Omorfi (open morfology for
Finnish)\footnote{\url{https://github.com/flammie/omorfi/}}
, as a project is centred around morphological
analysis of Finnish. Morphology is a core for many if
not most natural language processing systems,
especially in the case of such morphology-heavy
language as Finnish.  However, the detail and even
the formatting of the result of morphological
processing varies from application to application.
In order to produce all the different formats of
output and details of data, one needs to maintain a
large database of all the bits and pieces of lexical
information that is necessary to produce the wanted
readings, and that is what exactly what omorfi has
become over the years. What we have in the current
version of omorfi, is a lexical database of roots,
morphs and combinatorics, that can be \emph{weaved}
for use of different applications, documentations,
and automatic test suites by use of simple scripting.
The database is easy to maintain and update for
linguists and contributors while robust enough to
support a wide range of applications without progress
of each application specific data interfering with
each other.


\section{Database}

The word \emph{database}, in context of omorfi is currently used in a very
liberal sense, while we have structured our data in a manner that resembles
relational database to the extent that it could be converted into a one by
quite simple and fast process, we have opted to stick with basic
\emph{tab-separated-values} format for basically two reasons: firstly,
computational linguists and computer scientists are already well-versed to
handle this type of files with ease and efficiently on the command-line, they
integrate with the basics of unix taught to any computational linguist in the
past 20 years or so~\cite{church1994unix}, and the expected improvement in the use of real
relational databases comes from the complexity of dozens of tables and billions
of rows of data, whereas lexicon will likely not reach even million of
root-words any time soon. For easy editing the TSV files are importable and
exportable to all the commonly available free office products: e.g.,
LibreOffice\footnote{\url{http://libreoffice.org}} and OpenOffice.

\section{Application Data Format Generation}

The application of morphs and morphotactics is based
on \emph{finite state morphology} as documented by
e.g.,~\cite{beesley2003finite}. In order to be able
to compile our lexical database into a finite-state
automaton format for efficient processing, we
generate a \emph{lexc} file representation of the
data, and compile it using \emph{HFST}~\cite{hfst}
software that is available as a free and open source
system. The translation from tab-separated-values
into lexc is done by python scripts, an example of
formats is in listings~\ref{fig:tsv}
and~\ref{fig:lexc}.  Transformation is pretty
straight-forward and easy to maintain.

\begin{figure}[ht]
    \centering
    \begin{verbatim}
lemma   homonym new_para    origin
Aabel   1       N_STADION   unihu
...
talo    1       N_TALO      kotus
...
Öösti   1       N_TYYLI     unihu
    \end{verbatim}
    \caption{Lexical data in lexeme database,
        consisting of a \emph{unique key} of lemma
        and homonym number, a paradigm for inflection
        and source of origin, which is all the
        obligatory information for each word to be
        added in the database. The origin is only
        important for copyright purposes: here
        \emph{unihu} refers to a yet unnamed project
        in University of Helsinki and \emph{kotus}
    stands for Nykysuomen sanalista by kotus (RILF;
research institute of languages in Finland)
\label{fig:tsv}}
\end{figure}

\begin{figure*}[ht]
    \centering
    \begin{verbatim}
        LEXICON Nouns
        [WORD_ID=Aabel][POS=NOUN][PROPER=PROPER]:0Aabel N_STADION       ;
        ...
        [WORD_ID=talo][POS=NOUN]:0talo  N_TALO  ;
        ...
        [WORD_ID=Öösti][POS=NOUN][PROPER=PROPER]:0Ööst  N_TYYLI ;
    \end{verbatim}
    \caption{Lexical data in lexc-compatible format for compilation.
    \label{fig:lexc}}
\end{figure*}

The additional data can be added to each lexeme using
the lemma and homonym number as an identifier, these
data are stored in a separate TSV file that is joined
to the database in figure~\ref{fig:tsv} to produce a
master database. E.g., a named-entity recognition
project would add lines from \texttt{Aabel 1 first}
to \texttt{Öösti 1 geo} into a file of named entity
classes to add "first name" information to
\emph{Aabel} and "geographical place" information to
\emph{Öösti}. Then it can be accessed e.g., to
generate readings of \texttt{PROPER=FIRST} and
\texttt{PROPER=GEO} into the lexc code to be added to
a specific named-entity categorising automaton, or
the master database can be queried for this
information when the given lemma is seen in the
analysed text.


\section{Applications}

The applications we are demonstrating in this paper are:
statistical machine translation, rule-based machine translation,
spell-checking and correction, morphological segmentation, analysis and
generation.

For statistical machine translation, we are using
moses~\cite{moses}.\footnote{\url{http://www.statmt.org/moses/}} There are at
least two ways morphological processing can be used in moses
pipeline:\emph{segmentation}~\cite{dyer2008generalizing} and
\emph{factorisation}~\cite{koehn2007factored}.  In segmentation approach, the
word-forms are reduced to smaller units, such as morphs, or just words (i.e.
root morphs with sufixes intact but compounds broken), applying traditional
statistical machine translation methods to morphs is supposed to improve the
translation quality by decreasing the amount of unseen tokens and matching the
morphs to many non-Finnish word-forms more regularly (e.g., aligning suffixes
to preposition). In factored translation, results of morphological analysis are
stored in a vector, each component of which can be used at any point of
statistical machine translation: typical components of the vector are e.g.,
lemma, part-of-speech and full morphosyntactic description.

For rule-based machine translation setting we combine omorfi with
apertium~\cite{apertium}.\footnote{\url{http://apertium.sf.net}} In
apertium's shallow rule-based machine translation, omorfi is used for
morphological analysis, disambiguation and morphological generation.

For spell-checking we use voikko.\footnote{\url{http://voikko.sf.net}} In
spell-checking and correction omorfi is used to locate misspelled words
and to find most likely corrections given mispelling.~\cite{pirinen2014state}

The morphological analysis and segmentation are tasks that the above-mentioned
end-user programs depend on, but we also provide an API access for python and
Java as well as convenience bash scripts on top of direct access to the
automata.

Given this wide array of applications it is obvious how importance of lexical
data management has gotten to more central position as the project has
progressed: maximal coverage of word-forms for machine translation is not
invariably a good thing for spell-checker.

\section*{Acknowledgments}

The research leading to these results has received
funding from the European Union Seventh Framework
Programme FP7/2007-2013 under grant agreement
PIAP-GA-2012-324414 (Abu-MaTran).

Omorfi consists of a large body of work from numerous academic and open source contributors,
including: Inari Listenmaa, Francis M Tyers, Ryan
Johnson, and Juha Kuokkala.

\bibliographystyle{acl}
\bibliography{nodalida2015}


\end{document}
% vim: set spell:
