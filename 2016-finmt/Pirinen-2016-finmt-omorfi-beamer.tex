\documentclass{beamer}

\usepackage{fontspec}
\usepackage{polyglossia}

\usepackage{graphicx}
\usepackage{color}
\usepackage{url}

\mode<presentation>
{
    \usetheme{HZSK}
}


\title{Omorfi as a part of an MT pipeline
\scriptsize{in FinMT, Helsinki, 2016-09-12\\
\url{https://github.com/flammie/omorfi/}}}
\author{Tommi A Pirinen \scriptsize \guilsinglleft tommi.antero.pirinen@uni-hamburg.de \guilsinglright }
\institute{HZSK.de, CLARIN.de, etc.}
\date{\today}

\begin{document}

\selectlanguage{english}

\maketitle

\begin{frame}
    \frametitle{Introduction}
    \begin{itemize}
        \item Omorfi is a lexical database that can be compiled into different
            sorts of rule-based parsers and similar stuff
        \item can be useful for machine translation with: morphological
            analysis/generation, \{de,\}segmentation, tokenisation,
            lemmatisation
    \end{itemize}
\end{frame}


\begin{frame}
    \frametitle{Installation and use}
    \begin{itemize}
        \item For serious use: modern Linux distro, install recent autotools,
            gcc or clang, python3 etc. basics, then
            HFST\footnote{\url{http://hfst.github.io}} with python bindings
        \item \texttt{git clone git@github.com:flammie/omorfi \&\& cd omorfi
            \&\& ./autogen.sh \&\& make \&\& make install}
        \item Scripts starting with \texttt{omorfi-} should now just work
            (tabtab).
        \item e.g. \texttt{\$MOSES/scripts/tokenizer/tokenizer.perl <
            europarl-v8.fi.text | omorfi-factorise.py}
    \end{itemize}
\end{frame}

\begin{frame}
    \frametitle{Examples}
    Pretty pictures here
\end{frame}

\end{document}

