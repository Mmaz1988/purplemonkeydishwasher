\documentclass[a4paper]{article}
\usepackage{amssymb}
\usepackage{latexsym}
\usepackage{hyperref}
\usepackage{fontenc}

\begin{document}

% \mainmatter

\ifpdf
\pdfinfo{
  /Title (Weighting Finite-State Morphological Analyzers using HFST Tools)
  /Author (Krister Linden and Tommi Pirinen)
  /CreationDate (D:20090301123456)
  /Subject (Finite-State Morphology)
  /Keywords (FSM;FST;Morphology)
}
\fi

\title{Weighting Finite-State Morphological Analyzers
  using \textsc{HFST} Tools
  \footnote{This is author's draft; it may differ from published version slightly, especially since hyperref doesn't play nicely with llncs}}

\author{Krister Lindén \and Tommi Pirinen \\%}
%
%\institute{
  University of Helsinki\\
  Helsinki, Finland\\
  \{krister.linden,tommi.pirinen\}@helsinki.fi\\
}

\maketitle
\begin{abstract}
  In a language with very productive compounding and a rich
  inflectional system, e.g. Finnish, new words are to a large extent
  formed by compounding. In order to disambiguate between the possible
  compound segmentations, a probabilistic strategy has been found
  effective by Lindén and Pirinen \cite{linden09nodalida}. In this
  article, we present a method for implementing the probabilistic
  framework as a separate process which can be combined through
  composition with a lexical transducer to create a weighted
  morphological analyzer. To implement the analyzer, we use the
  \textsc{HFST-LexC} and related command line tools which are part of
  the open source \emph{Helsinki Finite-State Technology} package.
  Using Finnish as a test language, we show how to use the weighted
  finite-state lexicon for building a simple unigram tagger with 97~\%
  precision for Finnish words and word segments belonging to the
  vocabulary of the lexicon.
\end{abstract}


\section{Introduction}

In English the received wisdom is that traditional morphological
analysis is too complex for statistical taggers to deal with; a
simplified tagging scheme is needed. The disambiguation accuracy will
otherwise be too low even with an n-gram tagger because there is not
enough training material. However, currently training material for
morphological disambiguators is abundantly available. At the same
time, one could argue that the interest in tagging has disappeared,
because we can do more complex things such as syntactic dependency
analysis and get the morphological disambiguation as a side effect. As
a matter of curiosity, we will still pursue statistical tagging,
because there is also the initial result often attributed to Ken
Church that approximately 90 \% of the readings in English will be
correct if one simply gives each word its most frequent
morphosyntactic tag. We wish to derive a similar baseline for Finnish.

In addition, a morphologically complex language like Finnish is
different than English. In English there are hardly any inflectional
endings and applying traditional morphological analysis to English
necessarily creates massive ambiguity that can only be resolved by
context, whereas morphologically complex languages like Finnish in
each word most often carry the morphemes referred to by the
morphological tags. As the morphological tags have a physical
correspondence in the strings, it should be possible to use much less
context, or perhaps none at all, to disambiguate the traditional
morphological analysis of languages like Finnish. After all, the
reduced tag sets of English statistical taggers can be viewed as an
attempt to simplify the tag set to refer only to the visible surface
morphemes in a locally constrained context.

There are some initial encouraging results by Lindén and Pirinen
\cite{linden09nodalida} for disambiguating Finnish compounds using
unigram statistics for the parts in a productive compound process.
Unigram statistics for compounds is essentially the same as taking the
most likely morpheme segmentation and the most frequent reading of
each compound word. Similar results for disambiguating compounds using
a slightly different basis for estimating the probabilities have been
demonstrated for German by Schiller \cite{schiller2005} and by Marek
\cite{marek2006}. These results further encourage us to pursue the
topic of full morphological tagging for a complex language like
Finnish using only a lexicon and unigram statistics for the words and
their compound parts.

% Lindén and Pirinen \cite{linden09nodalida} demonstrated that a
% strategy of using as few parts as possible in a compound suggested by
% Karlsson \cite{karlsson1992} and corroborated by Sjöbergh et al.
% \cite{sjobergh2004} and Schiller \cite{schiller2005} can be framed in
% a probabilistic way and that probabilities can be estimated from
% non-compound word frequencies in a corpus. In fact, no segment
% counting seems to be necessary if the segment probabilities are
% used. A statistically motivated approach is sufficient. The rationale
% for the approach was that compounds are formed in order to distinguish
% between instances of frequently occurring phenomena, and consequently
% compounds are more often formed for more frequently discussed
% phenomena. To further simplify the estimation process, Lindén and
% Pirinen assumed that the frequencies of the word tokens directly
% affect the probability of the word forms used in the compound
% formation process. 

In \cite{linden09nodalida}, Lindén and Pirinen suggest a method which
essentially requires the building of a full form lexicon and an
estimate for each separate word form. This is not particularly
convenient, instead we introduce a simplified way to weight the
different parts of the lexicon with frequency data from a corpus by
using weighted finite-state transducer calculus. We use the open
source software tools of
\textsc{HFST}\footnote{\url{hfst.sourceforge.net}}, which contains
\textsc{HFST-LexC} similar to the Xerox LexC tool
\cite{beesley2004}. In addition to compiling LexC-style lexicons,
\textsc{HFST-LexC} has a mechanism for adding weights to compound
parts and morphological analyses. The \textsc{HFST} tools also contain
a set of command line tools that are convenient for creating the final
weighted morphological analyzer using transducer calculus.

We apply the weighted morphological analyzer to the task of
morphologically tagging Finnish text. As expected, it turns out that a
highly inflecting and compounding language with a free word order like
Finnish solves many of its linguistic ambiguities during word
formation. This pays back in the form of 97~\% tagger precision using
only a very simple unigram tagger in the form of a weighted
morphological lexicon for the words and word parts that are in the
lexicon. For words that contain unknown parts, the lexicalized
strategy is, however, rather toothless. For such words it seems, we
may, after all, need a traditional guesser and n-gram statistics for
morphological disambiguation.

The remainder of the article is structured as follows. In
Sections~\ref{Sect1}, we briefly present some aspects of Finnish
morphology that may be problematic for statistical tagging. In
Section~\ref{Sect3}, we introduce the probabilistic formulation of how
to weight lexical entries. In Section~\ref{Sect4}, we introduce the
test and training corpora. In Section~\ref{Sect5}, we evaluate the
weighted lexicon on tagging Finnish text. Finally, in
Sections~\ref{Sect7} and \ref{Sect8}, we discuss the results and draw
the conclusions.

\section{Finnish Morphology}
\label{Sect1}

We present some aspects of Finnish inflectional and compounding
morphology that may be problematic for statistical tagging in
Sections~\ref{Sect11} and \ref{Sect12}. For a more thorough
introduction to Finnish morphology, see Karlsson \cite{karlsson1999},
and for an implementation of computational morphology, see Koskenniemi
\cite{koskenniemi1983}. In Section~\ref{Sect12}, we present an outline
of how to implement the morphology in sublexicons which are useful for
weighting.
 
\subsection{Inflection in Finnish}
\label{Sect11}

In Finnish morphology, the inflection of typical nouns produces
several thousands of forms for the productive inflection. E.g. a noun
has more than 12 cases in singular and plural as well as possessive
suffixes and clitic particles resulting in more than 2000 forms for
every noun. 

Mostly the traditional linguistically motivated morphological analysis
of Finn\-ish is based on visible morphemes. However, for illustrational
purposes we will discuss two prototypical cases where the analysis
needs context. One such case is where a possessive suffix overrides
the case ending to create ambiguity: {\em taloni} 'my house/of my
house/my houses', i.e. either {\em talo} 'house' nominative singular,
{\em talon} 'of the house' genitive singular or {\em talot} 'houses'
nominative plural followed by a possessive suffix. This ambiguity is
systematic, so either the distinctions can be left out or one can
create a complex underspecified tag {\em +Sg+Nom/\-+Sg+Gen/\-+Pl+Nom} for
this case.

Another case, which is common in most languages, is the distinction
between nouns or adjectives and participles of verbs. This often
affects the choice of baseform for the word, i.e. the baseform of
'writing' is either a verb such as 'write' or a noun such as
'writing'. In Finnish, we have words like {\em taitava} 'skillful
Adjective' or 'know Verb Present Participle' and {\em kokenut}
'experienced Adjective' or 'experience Verb Past Participle'. Since
the two readings have different baseforms, it is not be possible to
defer the ambiguity to be resolved later by using underspecification.
In some cases, one of the forms is rare and can perhaps be ignored
with a minimal loss of information, but sometimes both occur regularly
and in overlapping contexts, in which case both forms should be
postulated and eventually disambiguated. However, sufficient
information for doing this reliably may not be available before some
degree of syntactic or semantic analysis.

In Sections~\ref{Sect5} and \ref{Sect7}, we will return to the
significance of these problems in Finnish and their impact on the
morphological disambiguation.

\subsection{Compounding in Finnish}
\label{Sect12}

Finnish compounding theoretically allows nominal compounds of
arbitrary length to be created from initial parts of certain noun
forms. The final part may be inflected in all possible forms.

Normal inflected Finnish noun compounds correspond to prepositional
phrases in English, e.g. {\em ostoskeskuksessa} 'in the shopping
center'. The morphological analysis in Finnish of the previous phrase
into {\em ostos\#keskus+N+Sg+Ine} corresponds in English to noun
chunking and case analysis into 'shopping center +N+Sg+Loc:In'.

In extreme cases, such as the compounds describing ancestors, nouns
are compounded from zero or more of \emph{isän} `father
\textsc{singular genitive}' and \emph{äidin} `mother \textsc{singular
  genitive}' and then one of the inflected forms of \emph{isä} or
\emph{äiti} creating forms such as \emph{äidinisälle} `to (maternal)
grandfather' or \emph{isänisänisänisä} `great great grandfather'.  As
for the potential ambiguity, Finnish also has the noun \emph{nisä}
`udder', which creates ambiguity for any paternal grandfather,
e.g. \emph{isän\#isän\#isän\#isä}, \emph{isän\#isä\#nisän\#isä},
\emph{isä\#nisä\#nisä\#\-nisä}, ...

Finnish compounding also includes forms of compounding where all parts
of the word are inflected in the same form, but this is limited to a
small fraction of adjective initial compounds and to the numbers if
they are spelled out with letters. In addition, some inflected verb
forms may appear as parts of compounds. These are much more rare than
nominal compounds \cite{iskwww} so they do not interfere with the
regular compounding.

\subsection{Finnish Computational Morphology}
\label{Sect13}

Pirinen \cite{pirinen2008} presented an open source implementation of
a finite state morphological analyzer for Finnish, which has been
reimplemented with the \textsc{HFST} tools and extended with data
collected and classified by Listenmaa \cite{listenmaa2009}. We use the
reimplemented and extended version as our unweighted lexicon.
Pirinen's analyzer has a fully productive noun compounding
mechanism. Fully productive noun compounding means that it allows
compounds of arbitrary length with any combination of nominative
singulars, genitive singulars, or genitive plurals in the initial part
and any inflected form of a noun as the final part. 

The morphotactic combination of morphemes is achieved by combining
sublexicons as defined in \cite{beesley2004}. We use the open source
software called \textsc{HFST-LexC} with a similar interface as the
Xerox LexC tool. The interested reader is referred to
\cite{beesley2004} for an exposition of the LexC syntax. The
\textsc{HFST-LexC} tool extends the syntax with support for adding
weights on the lexical entries. 

We note that the noun compounding can be decomposed into two
concatenatable lexicons separated by a word boundary marker, i.e. any
number of noun prefixes \emph{CompoundNonFinalNoun}$^*$ in
Figure~\ref{fig:unweighted1} separated by '\#' and from the inflected
noun forms \emph{CompoundFinalNoun} in
Figure~\ref{fig:unweighted2}. Similar decompositions can be achieved
for other parts of speech as needed. For a further discussion of the
structure of the lexicon, see \cite{linden09nodalida}.

\begin{figure}[h!]
  \centering
  \begin{scriptsize}
\begin{verbatim}
LEXICON Root
## CompoundNonFinalNoun ;
## #;

LEXICON Compound
#:0 CompoundNonFinalNoun;
#:0 #;

LEXICON CompoundNonFinalNoun
isä   Compound  "weight: 0, gloss: father" ;
isän  Compound  "weight: 0, gloss: father's" ;
äiti  Compound  "weight: 0, gloss: mother" ;
äidin Compound  "weight: 0, gloss: mother's" ;
\end{verbatim}
  \end{scriptsize}
  \caption{Unweighted fragment for
    \{\emph{CompoundNonFinalNoun}\}$^*$ i.e. \emph{noun
      prefixes}.}\label{fig:unweighted1}
\end{figure}

\begin{figure}[h!]
  \centering
  \begin{scriptsize}
\begin{verbatim}
LEXICON Root
CompoundFinalNoun ;

LEXICON CompoundFinalNoun
isä:isä+sg+nom     ## "weight: 0, gloss: father" ;
isän:isä+sg+gen    ## "weight: 0, gloss: father's" ;
isälle:isä+sg+all  ## "weight: 0, gloss: to the father" ;

LEXICON ##
## # ;
\end{verbatim}
  \end{scriptsize}
  \caption{Unweighted fragment for \emph{CompoundFinalNoun}, i.e.
    \emph{noun forms}.}\label{fig:unweighted2}
\end{figure}


\section{Methodology}  
\label{Sect3}

Assume that we want to know the probability of a morphological
analysis with a morpheme segmentation \emph{A} given the token
\emph{a}, i.e. $\mathrm{P}(A|a)$. According to Bayes rule, we get
Equation~\ref{eqn:cprob}.

\begin{equation}
  \label{eqn:cprob}
  \mathrm{P}(A|a) = \mathrm{P}(A,a)/\mathrm{P}(a) = \mathrm{P}(a|A) \mathrm{P}(A)/\mathrm{P}(a)
\end{equation}

We wish to retain only the most likely analysis and its segmentation
\emph{A}. As we know that $\mathrm{P}(a|A)$ is almost always 1, i.e. a
word form is known when its analysis is given. Additionally,
\emph{P(a)} is constant during the maximization, so the expression
simplifies to finding the most likely global analysis \emph{A} as
shown by Equation~\ref{eqn:maxprob}, i.e. we only need to estimate the
output language model.

\begin{equation}
  \label{eqn:maxprob}
  \arg\max_{A} \mathrm{P}(A|a) = \arg\max_{A} \mathrm{P}(a|A) \mathrm{P}(A)/\mathrm{P}(a) = \arg\max_{A} \mathrm{P}(A)
\end{equation}

In order to find the most likely segmentation of \emph{A}, we can make
the additional assumption that the probability \emph{P(A)} is
proportional to the product of the probabilities $\mathrm{P}(s_i)$ of
the segments of \emph{A}, where $A=s_1s_2...s_n$, defined by
Equation~\ref{eqn:prodprob}. This assumption based on a unigram
language model of compounding has been demonstrated by Lindén and
Pirinen \cite{linden09nodalida} to work well in practice.

\begin{equation}
  \label{eqn:prodprob}
  \mathrm{P}(A) \propto \prod_{s_i} \mathrm{P}(s_i)
\end{equation}

\subsection{Estimating probabilities}

The estimated probability of a token, \emph{a}, to occur in the corpus
is proportional to the count, \emph{c(a)}, divided by the corpus size,
\emph{cs}. The probability \emph{p(a)} of a token in the corpus is
defined by Equation~\ref{eqn:prob}. We also note that the corpus
estimate for \emph{p(a)} is in fact an estimate of the sum of the
probabilities of all the possible analyses and segmentations of
\emph{a} in the corpus.

\begin{equation}
  \label{eqn:prob}
  \mathrm{p}(a) = \mathrm{c}(a)/\mathrm{cs}
\end{equation}

Tokens \emph{x} known to the original lexicon but unseen in the corpus
need to be assigned a small probability mass different from 0, so they
get \emph{c(x) = 1}, i.e. we define the count of a token as its corpus
frequency plus 1 as in Equation~\ref{eqn:count}, also known as Laplace
smoothing.

\begin{equation}
  \label{eqn:count}
  \mathrm{c}(a) = 1 + \mathrm{frequency}(a)
\end{equation}

\subsection{Weighting the Lexicon}

In order to use the probabilities as weights in the lexicon, we
implement them in the tropical semiring, which means that we use the
negative log-probabilities as defined by Equation~\ref{eqn:logprob}.

\begin{equation}
  \label{eqn:logprob}
  \mathrm{w}(a) = -\mathrm{log}(p(a))
\end{equation}

In the tropical semiring, probability multiplication corresponds to
weight addition and probability addition corresponds to weight
maximization. In \textsc{HFST-LexC}, we use OpenFST \cite{openfst} as
the software library for weighted finite-state transducers.

\begin{figure*}[h!]
  \centering
  \begin{scriptsize}
\begin{verbatim}
LEXICON Root
## CompoundNonFinalNoun ;
## CompoundFinalNoun ;

LEXICON Compound
0:# CompoudNonFinalNoun;
0:# CompoudFinalNoun;

LEXICON CompoundNonFinalNoun
isä   Compound  "weight: -log(c(isä)/cs)" ;
isän  Compound  "weight: -log(c(isän)/cs)" ;
äiti  Compound  "weight: -log(c(äiti)/cs)" ;
äidin Compound  "weight: -log(c(äidin)/cs)" ;

LEXICON CompoundFinalNoun
isä+sg+nom  ##  "weight:-log(c(isä+sg+nom)/cs)" ;
isä+sg+gen  ##  "weight:-log(c(isä+sg+gen)/cs)" ;
isä+sg+all  ##  "weight:-log(c(isä+sg+all)/cs)" ;
isä+pl+ins  ##  "weight:-log(c(isä+sg+all)/cs)" ;

LEXICON ##
## # ;
\end{verbatim}
  \end{scriptsize}
  \caption{Structure weighting scheme using token penalties on the
    output language. Note that the functions in the comment field are
    placeholders for the actual weights.}\label{fig:weighted1}
\end{figure*}

For short, we call our unweighted compounding lexicon, \emph{Lex}, and
the decomposed noun compounding lexicon parts, i.e. the noun prefixes
\emph{CompoundNonFinalNoun}$^*$ in Figure~\ref{fig:unweighted1} and
the inflected noun forms \emph{CompoundFinalNoun} in
Figure~\ref{fig:unweighted2}, \emph{Pref} and \emph{Final},
respectively.

For an illustration of how the weighting scheme can be implemented in
the weighted output language model, $WLex$, of the noun compounding
lexicon, see Figure~\ref{fig:weighted1}. There is an obvious extension
of the weighting scheme to the output models of the decomposed
unweighted lexicons, \emph{Pref} and \emph{Final}. We call these
weighted output language models \emph{WPref} and \emph{WFinal},
respectively.

\subsection{Back Off Model}

The original lexicon, $Lex$, can be weighted by composing it with the
weighted output language, $WLex$, as in
Equation~\ref{eqn:knownweights}. However, there are a number of word
forms and compound segments in the lexicon, for which no estimate is
available in the corpus. We wish to assign a large weight to these
forms and segments, i.e. a weight \emph{M} which is greater than any
of the weights estimated from the corpus, e.g. $M =
log(1+\mathrm{cs})$. To calculate the missing words, we first use the
homomorphism $uw$ to map the $WPref$ to an unweighted automata, which we
subtract from $\Sigma^*$ and give the output model the final weight
$M$ using the homomorphism $mw$.

We create the following new sublexicons using automata difference and
composition with the original decomposed transducers in
Equations~\ref{eqn:unknownweights1} and \ref{eqn:unknownweights2}.

\begin{eqnarray}
  \label{eqn:knownweights} KnownAndSeenWords & = & Lex~o~WLex\\
  \label{eqn:unknownweights1} MaxUnseenPref & = & Pref~o~(mw(\Sigma^* - uw(WPref)))\\
  \label{eqn:unknownweights2} MaxUnseenFinal & = & Final~o~(mw(\Sigma^* - uw(WFinal)))
\end{eqnarray}

These sublexicons can be combined as specified in
Equation~\ref{eqn:unknownweights} to cover the whole of the original
lexicon.

\begin{eqnarray}
  \label{eqn:unknownweights}
  WeightedLexicon~=~KnownAndSeenWords~|~Pref~MaxUnseenFinal \nonumber\\
  |~MaxUnseenPref~Final~|~MaxUnseenPref~MaxUnseenFinal~~
\end{eqnarray}

The $WeightedLexicon$ will assign the lowest corpus weight to the most
likely reading and the highest corpus weight to the most unlikely
reading of the original lexical transducer.

\section{Data Sets}
\label{Sect4}

As training and test data, we use a compilation of three years,
1995-1997, of daily issues of Helsingin Sanomat, which is the most
wide-spread Finnish newspaper. We disambiguated the corpus using
Machinese for Finnish\footnote{Machinese is available from Connexor
  Ltd., www.connexor.com} which provided one reading in context for
each word using syntactic parsing. This provided us with a
mechanically derived standard and not a human controlled gold
standard.

\subsection{Training Data}

The training data actually spanned 2.5 years with 1995 and 1996 of
equal size and 1997 only half of this. This collection contained
approximately 2.4 million different words, i.e. types, corresponding
to approximately 70 million words of Finnish, i.e. tokens, divided
into 29 million tokens for 1995, 29 for 1996 and 11 for 1997. We used
the training data to count the non-compound tokens and their analyses.

\subsection{Test Data}

From the three years of training data we extracted running text from
comparable sections of the news paper data. We chose articles from the
section reporting on general news with normal running text (as a
contrast to e.g. the economy or sports section with significant
amounts of numbers and tables). The extracted test data sets contained
118~838, 134~837 and 193~733 tokens for 1995, 1996 and 1997,
respectively. We used the test data to verify the result of the
disambiguation.

\subsection{Baseline}

As a baseline method, we use the training data as such to create
statistical unigram taggers as outlined in Section~\ref{Sect3}. In
Table~\ref{tab:taggerdata}, we show the baseline result for the test
data samples with a given training data tagger, the number of tokens
with 1st correct reading, the number of tokens with some other correct
reading, the number of tokens with some readings but no correct and
the number of tokens with no reading.

\begin{table}[htb!]
  \centering
  \caption{Baseline of the tagger test data.
  }\label{tab:taggerdata}
  \begin{scriptsize}
    \begin{tabular}{c|c|c|c|c|c|c}
      \hline
      ~~Train~~ & ~~Test~~ & ~~$1^{st}$~~& ~~$n^{th}$~~ & ~~No~~ & ~~No~~ & ~~Comment~~\\
      ~~Year~~ & ~~Year~~ & ~~Correct (\%)~~ & ~~Correct (\%)~~ & ~~Correct (\%)~~ & ~~Analysis (\%)~~ \\
      \hline 
      1995 & 1995 & 96.3 & 3.7 & 0.0 & 0.0 &~~Max.~~\\
      1995 & 1996 & 92.2 & 3.3 & 0.3 & 4.1 & \\
      1995 & 1997 & 91.9 & 3.3 & 0.3 & 4.6 & \\
      \hline 
      1996 & 1995 & 91.9 & 3.4 & 0.4 & 4.5 & \\
      1996 & 1996 & 96.4 & 3.6 & 0.0 & 0.0 & ~~Max.~~\\
      1996 & 1997 & 92.4 & 3.2 & 0.3 & 4.1 & \\
      \hline 
      1997 & 1995 & 89.6 & 3.3 & 0.5 & 6.6 & \\
      1997 & 1996 & 90.1 & 3.2 & 0.4 & 6.2 & \\
      1997 & 1997 & 96.7 & 3.3 & 0.0 & 0.0 & ~~Max.~~\\
      \hline 
    \end{tabular}
  \end{scriptsize}
\end{table}


\section{Tests and  Results}
\label{Sect5}

We created two versions of the weighted lexicon for disambiguating
running text. One weights the lexicon using the current corpus and
tests the result using only the weighted lexicon data. The second test
adds the baseline tagger to the lexicon in order to ensure some
additional domain specific data for lack of a guesser.

\subsection{Lexicon-based Unigram Tagger}

We did our first tagging experiment using a full year of news paper
articles as training data for the lexicon and testing with the test
data from the other two years. The first correct results are
consistently at 97~\% of the words with some correct
analysis. However, the coverage is totally dependent on the fairly
restricted lexicon as shown in Table~\ref{tab:taggerresults1}. We also
include the results for testing and training on the same year as an
upper limit or reference.

\begin{table}[h!]
  \centering
  \caption{Lexicon-based unigram tagger results for Finnish.
  }\label{tab:taggerresults1}
  \begin{scriptsize}
    \begin{tabular}{c|c|c|c|c|c|c}
      \hline
      ~~Train~~ & ~~Test~~ & ~~$1^{st}$~~& ~~$n^{th}$~~ & ~~No~~ & ~~No~~ & ~~Comment~~\\
      ~~Year~~ & ~~Year~~ & ~~Correct (\%)~~ & ~~Correct (\%)~~ & ~~Correct (\%)~~ & ~~Analysis (\%)~~ \\
      \hline 
      1995 & 1995 & 68.2 & 1.2 & 12.0 & 18.5 & ~~Max.~~\\
      1995 & 1996 & 69.4 & 1.3 & 12.0 & 17.3 & \\
      1995 & 1997 & 69.4 & 1.4 & 11.7 & 17.5 & \\
      \hline 
      1996 & 1995 & 67.9 & 1.4 & 12.0 & 18.5 & \\
      1996 & 1996 & 69.7 & 1.0 & 12.0 & 17.3 & ~~Max.~~\\
      1996 & 1997 & 69.4 & 1.3 & 11.7 & 17.5 & \\
      \hline 
      1997 & 1995 & 67.9 & 1.6 & 12.0 & 18.5 & \\
      1997 & 1996 & 69.4 & 1.3 & 12.0 & 17.3 & \\
      1997 & 1997 & 69.6 & 1.3 & 11.7 & 17.5 & ~~Max.~~\\
      \hline 
    \end{tabular}
  \end{scriptsize}
\end{table}


\subsection{Extended Lexicon-based Unigram Tagger}

We did our second tagging experiment as the first with the addition of
using the full year of news paper data for extending the lexicon.
Again, we tested with the test data from the other two years. The
first correct results are consistently at 98~\% of the words with some
correct analysis and the coverage is now considerably better as shown
in Table~\ref{tab:taggerresults2}. We also include the results for
testing and training on the same year as an upper limit or reference.

\begin{table}[h!]
  \centering
  \caption{Extended lexicon-based unigram tagger results for Finnish.
  }\label{tab:taggerresults2}
  \begin{scriptsize}
    \begin{tabular}{c|c|c|c|c|c|c}
      \hline
      ~~Train~~ & ~~Test~~ & ~~$1^{st}$~~& ~~$n^{th}$~~ & ~~No~~ & ~~No~~ & ~~Comment~~\\
      ~~Year~~ & ~~Year~~ & ~~Correct (\%)~~ & ~~Correct (\%)~~ & ~~Correct (\%)~~ & ~~Analysis (\%)~~ \\
      \hline 
      1995 & 1995 & 95.9 & 4.1 & 0.0 & 0.0 & ~~Max.~~\\
      1995 & 1996 & 93.3 & 4.0 & 0.7 & 2.0 & \\
      1995 & 1997 & 93.1 & 4.0 & 0.6 & 2.3 & \\
      \hline 
      1996 & 1995 & 92.9 & 4.0 & 0.7 & 2.2 & \\
      1996 & 1996 & 96.1 & 3.9 & 0.0 & 0.0 & ~~Max.~~\\
      1996 & 1997 & 93.6 & 3.7 & 0.6 & 1.9 & \\
      \hline 
      1997 & 1995 & 91.6 & 4.1 & 1.0 & 3.2 & \\
      1997 & 1996 & 92.1 & 3.9 & 0.9 & 3.1 & \\
      1997 & 1997 & 96.3 & 3.7 & 0.0 & 0.0 & ~~Max.~~\\
      \hline 
    \end{tabular}
  \end{scriptsize}
\end{table}

\section{Discussion and Further Research}
\label{Sect7}

In this section we analyze the errors for which the correct tag
sequence was not first, for which there was no correct tag sequence
and for which there was no analysis at all. We present the most common
by tag sequences or tokens. Finally, we make a few
additional observations.

\subsection{Correct Tag not $1^{st}$ in Analysis}

The cases where correct tag is not the first are dominated by the already
known ambiguities where a token has multiple readings and both exist in corpus.
One big class of these are verbs like \emph{olla} or negation verb \emph{ei},
since in perfect tense's passive the auxiliary is still in present tense
active form (e.g. \emph{on kerrottu} `has been told' is 
\texttt{olla+pass+ind+pres kerrottu+pass+pcp2}
while most likely reading of \emph{on} `is' is \texttt{olla+act+ind+pres+sg3}).
The majority of variation between adjective readings and participles results
also in number of wrong choices in tag strings with A or V PCP2. 
Also for many tokens the variation between adverb and adposition is purely
syntactical and as such unigram tagger will fail in minority of cases.
Also for handful of verbs, the A infinitive form falls together with
3${}^{rd}$ person singular present tense (e.g. \emph{järjestää} `to arrange/(he) arranges') which causes fair amount of unigram tagger misreadings. The amount
of compounds in analyses where first is not correct ranges from 6 \% to 12 \%.

\begin{table}[h!]
  \centering
  \caption{Error analysis for cases where correct reading exists.
  }\label{tab:incorrect1}
  \begin{scriptsize}
    \begin{tabular}{c|c|c|c}
      \hline
      Error Type & Baseline & Dictionary & Combined \\
      \hline 
	  Tagged `ADV'/`PSP' & 4112 & 2561 & 4331 \\
	  Tagged `A SG NOM'/`V PCP2 SG NOM' & 3093 & 885 & 3388 \\
	  \hline
	  Token `on' & 3855 & 0 & 3855 \\
	  Token `ei' & 1170 & 0 & 1170 \\
	  Token `ollut' & 735 & 0 & 735 \\
	  \hline 
    \end{tabular}
  \end{scriptsize}
\end{table}

\subsection{Analyses without Correct Tag}

For analyses where correct analysis was not among the readings, 
In corpus there’s a handful of underspecified analyses, such as (blah A),
which aren’t produced at all by dictionary based analyzer, but assumably the
corpus’s syntactic tagging mechanism has had use for those. Also for some
adverbs and adpositions the dictionary only contains the non-lexicalised
nominal form. There is also some overlap for cases in previous category here
if the alternate reading for ambiguous form is not generated or found for some
year.

\begin{table}[h!]
  \centering
  \caption{Error analysis for cases where correct reading is missing.
  }\label{tab:incorrect1}
  \begin{scriptsize}
    \begin{tabular}{c|c|c|c}
      \hline
      Error Type & Baseline & Dictionary & Combined \\
	  \hline
	  Tagged `A SG NOM' & 24 & 771 & 82 \\
	  Tagged `V PCP2 SG NOM' & 20 & 4212 & 50 \\
	  Tagged `A' & 0 & 3300 & 0 \\
	  \hline 
    \end{tabular}
  \end{scriptsize}
\end{table}

\subsection{No Analysis}

For dictionary based tagger, the tokens which mainly dominate the missing
analyses are proper nouns, abbreviations and numerals, which are known
shortcomings for the analyzer. For other analyzers, such as baseline or
extended, the main problem is proper nouns, many of which may appear only in
one years issues. Also, since the dictionary based analyzer lacks productive
numeral formation, many of the complex numeral expressions (e.g.
\emph{5---15-vuotiaat} `5-to-15-year-olds')  or specific numbers (e.g. 
\emph{4029354})
are missing when using training corpora from one year to test other years
analyses.

\begin{table}[h!]
  \centering
  \caption{Error analysis for cases where no results are given.
  }\label{tab:incorrect1}
  \begin{scriptsize}
    \begin{tabular}{c|c|c|c}
      \hline
      Error Type & Baseline & Dictionary & Combined \\
      \hline
	  Proper nouns & 17795 & 106101 & 17379 \\
	  \hline
	  Token `klo' & 0 & 13242 & 0 \\
	  Token `mk' & 0 & 5432 & 0 \\
	  \hline
	  Tag NUM & 699 & 7830 & 388 \\
	  \hline 
    \end{tabular}
  \end{scriptsize}
\end{table}

\subsection{Other Observations}

The error analysis confirms that the compounds for which the all parts
were known contributed on the average 0.67 \% to the overall error
rate, i.e. correct not in the first position, for words with at least
one correct analysis. For further discussions on the similarities and
differences between Finnish, German and Swedish compounding, see
\cite{linden09nodalida}.

If a disambiguated corpus is not available for calculating the word
analysis probabilities, it is still possible to use only the string
token probabilities to disambiguate the compound structure without
saying anything about the most likely morphological reading. This
segmentation would be similar to the segmentation the Morfessor
software \cite{creutz2005} tries to discover in an unsupervised way
from corpora alone.

The good results for statistical morphological disambiguation of
Finnish with a full morphological tag set using only a unigram model
is most likely the result of the highly inflectional and compounding
morphology of Finnish with free word order. In order for a language to
achieve a free word order, morphological ambiguities have to be
resolvable locally almost without context. 

As the inflected Finnish compounds correspond to noun phrases or
prepositional phrases in English. This also sheds some additional
light on the supposedly free word order in Finnish, which is similar
to the rather free phrase ordering in many other languages,
i.e. similar changes in the topic of a clause occurs in Finnish when
shifting a phrase e.g. to a clause initial position.

\section{Conclusions}
\label{Sect8}

We demonstrated how to build a weighted lexicon for a highly
inflecting and compounding Fenno-Ugric language like Finnish. Similar
methods apply to a number of Germanic languages with productive
morphological compounding. From a practical point of view, we
introduced the open source command line tools of \textsc{HFST} and
used them successfully for compiling a weighted lexicon. We applied
the weighted lexicon as a unigram tagger of running Finnish text
achieving 97~\% precision on words in the vocabulary. The unigram
tagger is a good baseline when tagging morphologically complex
languages like Finnish and for some purposes it may even be sufficient
as such. In addition, it is easy to implement if a full-fledged
morphological analyzer and a training corpus is available. For unknown
foreign words and names, a guesser and an n-gram tagger may still be
necessary.

\section*{Acknowledgments}
This research was funded by the Finnish Academy and the Finnish
Ministry of Education. We are also grateful to the HFST--Helsinki
Finite State Technology research team and to the anonymous reviewers
for various improvements of the manuscript.

\bibliographystyle{acl}

\begin{thebibliography}{14}

\bibitem{openfst}
Cyril Allauzen, Michael Riley, Johan Schalkwyk, Wojciech Skut, and Mehryar
  Mohri.
\newblock 2007.
\newblock Open{F}st: A General and Efficient Weighted Finite-State Transducer
  Library.
\newblock In {\em Proceedings of the Ninth International Conference on
  Implementation and Application of Automata, (CIAA 2007)}, volume 4783 of {\em
  Lecture Notes in Computer Science}, pages 11--23. Springer.
\newblock \url{http://www.openfst.org}.

\bibitem{beesley2004}
Kenneth~R. Beesley and Lauri Karttunen.
\newblock 2003.
\newblock {\em Finite State Morphology}.
\newblock CSLI Publications.
\newblock \url{http://www.fsmbook.com}.

\bibitem{creutz2005}
\newblock Mathias Creutz, Krista Lagus, Krister Lindén, Sami Virpioja.
\newblock 2005.
\newblock Morfessor and Hutmegs: Unsupervised Morpheme Segmentation for Highly-Inflecting and Compounding Languages.
\newblock In {\em Proceedings of the Second Baltic Conference on Human Language Technologies}.

\bibitem{iskwww}
Auli Hakulinen, Maria Vilkuna, Riitta Korhonen, Vesa Koivisto, Tarja~Riitta
  Heinonen, and Irja Alho.
\newblock 2008.
\newblock {\em Iso suomen kielioppi}.
\newblock Suomalaisen Kirjallisuuden Seura.
\newblock referred on 31.12.2008, available from
  \url{http://scripta.kotus.fi/visk}.

\bibitem{karlsson1999}
Fred Karlsson.
\newblock 1999.
\newblock Finns - An Essential Grammar.
\newblock Routledge. London. First published 1983 as Finnish Grammar.

% \bibitem{karlsson1992}
% Fred Karlsson.
% \newblock 1992.
% \newblock Swetwol: A Comprehensive Morphological Analyzer for Swedish.
% \newblock {\em Nordic Journal of Linguistics}, 15(2):1--45.

\bibitem{koskenniemi1983}
Kimmo Koskenniemi.
\newblock 1983.
\newblock Two-Level Morphology: A General Computational Model for Word Form Generation and Recognition.
\newblock {\em Publication No. 11. Publications of the Department of General Linguistics}. University of Helsinki.

\bibitem{linden09nodalida}
Krister Lindén and Tommi Pirinen.
\newblock 2009.
\newblock Weighted Finite-State Morphological Analysis of Finnish Compounding
with \textsc{HFST-LexC}.
\newblock In {\em Proceedings of NoDaLiDa 2009}.

\bibitem{listenmaa2009}
Inari Listenmaa.
\newblock 2009.
\newblock Combining Word Lists: Nykysuomen sanalista, Joukahainen-sanasto 
and Käänteissanakirja (in Finnish). 
\newblock Bachelor’s Thesis. Department of Linguistics. University of Helsinki. 

\bibitem{marek2006}
Torsten Marek.
\newblock 2006.
\newblock Analysis of German Compounds using Weighted Finite State Transducers.
\newblock Technical report, Eberhard-Karls-Universität Tübingen.

\bibitem{pirinen2008}
Tommi Pirinen.
\newblock 2008.
\newblock Suomen kielen äärellistilainen automaattinen morfologinen
  analyysi avoimen lähdekoodin keinoin.
\newblock Master's thesis, Helsingin yliopisto.

\bibitem{schiller2005}
Anne Schiller.
\newblock 2005.
\newblock German Compound Analysis with {\textsc wfsc}.
\newblock In {\em FSMNLP}, pages 239--246.

% \bibitem{sjobergh2004}
% Jonas Sjöbergh and Viggo Kann.
% \newblock 2004.
% \newblock Finding the Correct Interpretation of {Swedish} Compounds -- a
%  Statistical Approach.
% \newblock In {\em Proceedings of {LREC}-2004}, pages 899--902, Lisbon,
%  Portugal.
% \newblock \url{http://dr-hato.se/research/sjobergh_kann_04.ps}.

\end{thebibliography}

\end{document}
