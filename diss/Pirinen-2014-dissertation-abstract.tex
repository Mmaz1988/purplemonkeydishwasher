This dissertation is a large-scale study of spell\-/checking and correction
using finite\-/state technology. Finite-state spell\-/checking is a key method
for handling morphologically complex languages in a computationally
efficient manner. This dissertation discusses the technological and practical
considerations that are required for finite\-/state spell-checkers to be at
the same level as state-of-the-art non-finite\-/state spell-checkers. 

Three aspects of spell\-/checking are considered in the thesis: modelling of
correctly written words and word-forms with finite\-/state language
models, applying statistical information to finite\-/state language models
with a specific focus on morphologically complex languages, and modelling
misspellings and typing errors using finite\-/state automata-based error
models.

The usability of finite\-/state spell-checkers as a viable alternative to
traditional non-finite-state solutions is demonstrated in a large-scale
evaluation of spell-checking speed and the quality of languages with
morphologically different natures. The selected languages display a
full range of typological complexity, from isolating English to 
polysynthetic Greenlandic to Finnish and the Saami languages.
