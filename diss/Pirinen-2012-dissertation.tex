\documentclass[officiallayout]{unihelcompling}
%\documentclass[officiallayout,draft]{unihelcompling}

\usepackage{amssymb}

% hacking citations
\usepackage{natbib}


\usepackage{fontspec}
\usepackage{xunicode}
\usepackage{xltxtra}

\usepackage{url}
\usepackage{hyperref}
\usepackage[obeyDraft]{todonotes}


\usepackage{multirow}

\usepackage[normalem]{ulem}


% hacking the images
\usepackage{graphicx}
\usepackage{geometry}

\newcommand\misspelt{\bgroup\markoverwith
{\textcolor{red}{\lower3.5pt\hbox{\sixly \char58}}}\ULon}

%\setmainfont{Times}

\title{Weighted Finite-State Methods for 
%\misspelt{
Spell-Checking
%}
and Correction\footnote{Rough draft}}
\author{Tommi A Pirinen}


\authorcontact{\url{tommi.pirinen@helsinki.fi}\par
  \url{http://www.helsinki.fi/\%7etapirine}}
\pubtime{Oddmonth}{2012}
\reportno{0}
\isbnpaperback{000-00-0000-0}
\isbnpdf{000-00-0000-0}
\issn{0000-0000}
\printhouse{Unigrafia?}
\pubpages{000} % --- remember to update this!
\supervisorlist{Krister Lindén, University of Helsinki, Finland}
\preexaminera{Kekke Roos, Unseen University, United Kingdom}
\preexaminerb{I.D.K. Kurt, University of Turku, Finland}
\opponent{Advocatus Diaboli, University of Amsterdam, The Netherlands}
\custos{Homer Simpson, University of Springfield, The U.S.A.}
\generalterms{thesis, example, another example, still more examples,
  more and more examples}
\additionalkeywords{example, an example phrase with many words}
\crcshort{A.0, C.0.0}
\crclong{
\item[A.0] Example Category
\item[C.0.0] Another Example
}
\permissionnotice{
  To be presented in \ldots{} text of a long permission notice. Text of
  a long permission notice. Text of a long permission notice. Text of
  a long permission notice. Text of a long permission notice. Text of
  a long permission notice.
}



\date{\today}

\begin{document}

\frontmatter

\maketitle

\begin{abstract}
    This dissertation is a collection of practical results and methods used
    to implement efficient spell-checking systems a selection of typologically
    different languages.
\end{abstract}

\tableofcontents

\listoftodos

\mainmatter

\chapter*{Preface}
\label{chap:preface}

Spell-checkers are a very basic-level, commonly used practical software.
Getting usable spell-checkers for the many morphologically complexer languages
has been, and still is quite a chore on many systems. This even includes trying
to get Finnish spell-checking for open-source systems. The science behind many
of the practical contributions of this thesis has been known for half a
century, yet the inclusion of the findings is still not there.  While I'm
doubtful that the software that is in the core topic of this thesis may never
be integrated to the open source systems as the de facto standard, installed by
default, or available in standard distributions for common use, at least it has
been of good use for me and the possibly a few alpha testers in Greenland, as
well as the ten or so fellow researchers who dare to take the steps to install
it to their systems.  All in all, the common requirement for many communities
for a usable spell-checker has been enough motivation to build these
proofs-of-concept implementations, hopefully they will carry over to future of
spell-checking of these computationally more challenging languages in some form
and/or are usable.

%%\section*{Acknowledgements}
%%\label{sec:acknowledgements}
%%
%%Generally speaking, I am not a person to explicitly thank any one typically. It
%%should be noted though, that if you are reading this text searching for your
%%name, it is the most likely that you deserve it; obviously keeping track of
%%everyone I owe gratitude for is not an easy task for disorganised person like
%%myself, so here is a randomly organised list of things I am thankful for, or
%%have been thankful for during the idle times of writing this thesis, in
%%absolutely no particular order and without doubt full of omissions.
%%
%%The first obvious contribution for the whole thesis project and all comes from
%%the local research group and other colleagues in University of Helsinki. The
%%HFST project for working on the software and the FinnTreeBank and Finnish
%%Wordnet project for providing backup leg power to Finnish language model
%%building. Specifically I would like to thank my original master's thesis 
%%advisor Kimmo Koskenniemi for leading me to this line of work, my PhD thesis
%%advisor and HFST project leader Krister Lindén for keeping the goal in
%%sight throughout the thesis project and helping managing the times and ideas.
%%In HFST team I owe to Sam Hardwick, for doing really the majority of software
%%work in \texttt{hfstospell} branch of project, Miikka Silfverberg for
%%providing the statistical and context-based finite-state models that go beyond
%%my expertise and Erik Axelson for maintaining the whole software behemoth. 
%%
%%I am very grateful to Per Langgård for picking my contribution in LREC 2011 up
%%as a way to get the next version of Greenlandic speller to people and thus
%%introducing me to whole intriguing problem field of poly-synthetic languages'
%%computational modelling. The thesis and its focus would be quite different
%%without it.
%%
%%I thank Jack Rueter for numerous fruitful discussions on all topics related
%%to uralic languages, among others. 
%%
%%On personal level I am thankful to my family, my friends in university of
%%Helsinki and the one true deity who shares the initials with finite-state
%%machines---without them, none of this would have been possible.


\chapter{Introduction}
\label{chap:introduction}

Spell-checking and correction is among the more practical, well-understood
subjects in study of computational linguistics, and earlier, computer science.
The task of detecting spelling mistakes in different texts---written, scanned,
or otherwise---is a very simple concept to grasp. And the computational
handling of the problem has been subject of research for almost as long as
computers have been capable of processing texts, with first influential
academic works published in the 1960's. And the field has developed much since.

The purpose of this thesis is to research one specific approach to the
spell-checking---that of finite-state technology. The finite-state technology
has its roots in mathematical theory of formal \emph{languages}, which I will
refer to as \emph{string sets} throughout this thesis to avoid confusion with
the more common meaning of the term, natural languages. The theoretical beauty
of this approach will be recapped in the later subsections more closely, the
practical considerations it has are following: it gained popularity among the
computational linguists of morphologically complex languages around 1980's and
onwards, and is thought by some of us as the only way to handle more
morphologically complex languages, therefore, this sets the directions of this
thesis as such \emph{implementing spell-checking for languages of varying
morphological complexity} (efficiently). 

Especially traditional the finite-state approaches are to computational
handling of Finnish, which is my native language and a recurring theme in this
thesis, for practical reasons if not more. This thesis---whilst concentrating
on scientific contributions made during the years--is in a way also a book
describing the past years of development the actual spell-checking software
\footnote{available, free and open source, like all good scientific research
products}, one that is used also in making of this very thesis. So, it should
be clear that the contribution of this thesis is meant to be a very concrete,
should I even say a final---albeit beta-quality---system for spell-checking.
There are some practical implications of using native language to evaluate and
develop new and improved methods, the motivation of spending measurable amount
of spare time is there.

So one common theme throughout the thesis is efficiency. As we are working with
a very practical system with potentially everyday use, we need to be sure to
root the theoretical advances so that they are practically usable. For this
reason the systems used in the thesis are also mainly limited to freely
available and open source materials and techniques, indeed, if a spell-checking
sub-system is mentioned within this thesis you should be sure to be able to
download a functional prototype at least, all the software and data is
available in our version management 
system\footnote{\url{http://svn.code.sf.net/p/hfst}}, as should be in all
properly conducted scientific research to fulfill the basic requirement of
reproducibility.\todo{urls here and there}

Another point-of-view in the thesis is quality of the spell-checking for the
morphologically complex languages. The vast majority of research on the topic
that there is has already shown impressive numbers on quality of spell-checking
and corrections, with statistical approaches, gigaword corpora easily available
(for English), and so on, it almost seems like a solved problem. Even the
common intuition, however, does raise suspicion if the morphologically more
complex language may not be so favorable to simple statistical approaches. The
contribution of this thesis is also in showing the limits of these approaches
for morphologically complex, and resource-poor languages, exploring possible
modifications or tricks that can be used to recover the situation and so forth.

The remainder of this first chapter is dedicated for informal introduction to
the thesis topic. It consists of closer definitions of the practical and
scientific concepts central to treating building a practical spell-checking
system a research problem, some very down-to-earth rationale for this approach,
and the background details that motivate the research of finite-state methods
for spell-checking even when spell-checking itself has already been researched
to death otherwise. Finally, as this is an article-based thesis, I give an
overview to the articles of the thesis, and describe how the thesis is
structured.

\section{Practical Components of Spell-Checking System}
\label{sec:practical-components}

The spell-checking and correction, like the title would suggest, can already be
torn up to two sub-parts. Furthermore both of the sub-parts can be divided,
classified and analysed into many separate pieces. In this section I try to
cover different existing and practical conceptualisations, and set the relevant
terminology up in neat and systematic manner.

The primary division of labour that is present throughout this thesis including
the title, is the division to spell-checking and correction. The former is the
task of \emph{detecting} the errors from the texts, whereas the latter is the
task of \emph{suggesting} the most likely correct word forms in place of the
error. It is of great importance to treat these two tasks as separate, as the
implementations made for each other can be quite freely mixed and matched.
While there are systems that use the same methods and approaches for both,
this is not by any means necessity. A system giving suggestions should usually
not need to care which kind of word form it is correcting and by what criteria
it got it in order to work, and this is an important feature in a few of the
approaches I present in the thesis and therefore some concepts I present in
this section.

The methods of detecting errors can be divided practically among the data they
use for evidence of word being \emph{misspelt}. The basic division is whether
the system looks word-form itself, in \emph{isolated} spelling error detection,
or if it actually looks the \emph{context}. Traditionally the isolated error
detection has based on word-form lists from where to lookup if the word-form
to check is valid, the only additional notion we have in terms of this thesis,
as I am talking about finite-state spell-checking, is that our 
\emph{dictionaries}, are ...(systems) capable of representing infinite number
of word-forms, that is, they contain not only the dictionary words, but the
derivations and compounds necessary to reasonably spell-check morphologically
complex language. Sometimes, but not very often, this spell-checking is
backed up with statistical information about words, the words that are in
dictionary, but very rare, may be marked as errors, usually just only if they
are one typing error away from a very common word, such as in case of
\todo{pick examples and few references, jurafsky etc.}
The short-coming of isolated approaches for spell-checking
is that they will only recognise those spelling errors, which result in a
word-form that is not in the dictionary, the so-called \emph{non-word}
spelling errors. There are also a lot of spelling errors that result in a
word-form that is another valid word-form in the language, but not one that
the writer meant. These \emph{real-word} spelling errors will almost always
require the use of context to get some evidence that something is wrong. Here
again, the simplest approach is to use statistics; if the two-three words do not
usually appear around the one we are detecting, it may be an error. Another,
more elaborate context-based method for detecting errors is to use full-fledged
natural language processing system that can parse morphology, syntax or other
features from running text, can usually recognise sentences or phrases that
are unusual, or even directly grammatically wrong; these systems are usually
already called \emph{grammar checkers} and correctors, rather than spelling
checkers, and are not mostly covered in this thesis\footnote{a word of warning
    about this terminology: some practical systems, such as office software,
    will call any spelling checker that uses any context based approach,
    as opposed to isolated approach, a spelling checker. This terminological
    confusion is not carried over to Microsoft Word's recent incarnations, but
    at the time of writing Apache OpenOffice.Org and LibreOffice do still follow
the confused terminology.}.

The error-correction task is divided most naturally by the types of errors that
are made. There have been many different classifications for these with
different namings and all, but at least most will agree that there is a
category of errors that is based on unintentional mistake, such as
mistyping on the keyboard. This is the type of spelling error that is the
main target for correction suggestion systems typically, and the common
research results have claimed that between 80 and 95 \% of all spelling
errors can be attributed by having \emph{one} typing error in a word.\todo{the
error types are in that one article}



\section{Languages' Morphological Complexity}
\label{sec:morphological-complexity}

One theme of the thesis was to bring the effective and high-quality
spell-checking for languages that have different variety of morphological
features---or complexity, though this term is sometimes frowned upon with the
belief that morphologically simple languages like English would be considered
less valuable. To substantiate this claim I will first try to define what I
mean by the morphological complexity. The topic of this thesis is not typology
or morphological complexity, and I will take a rather practical view on the
topic. This is a measure of complexity that affects the building of the correct
word-form recognisers and use and requirements of the resources; there are only
few easily measurable factors in this play-field: The morphological complexity
for many of the purposes of creation of the language model is the complexity of
the word-forms in terms of \emph{morphs}, that is, the average count of morphs
in the word or \emph{morph-to-word} ratio\footnote{In literature the term is
often morpheme to word ratio, as morpheme is an abstraction that cannot really
be counted from actual text, I prefer to use less misleading form of the
term.}. The morphs I use here refers to the smallest divisible unit in the word
that is used so regularly, that it can be practically used in the language
description---that is, a productive morph. For example, the English word
\emph{cat} has one morph, and its plural, \emph{cats} has two morphs.  Another
factor of morphological complexity that is important is the variation, that
creates a set of related morphs, which are encoded in an abstraction called
\emph{morpheme}, for example, the words \emph{runs}, and \emph{catches}, have
suffix morphs \emph{-s}, and \emph{-es}, which could be grouped together. This
variation can be realised in terms of number of \emph{morphemes} per language
or \emph{morphs-per-morpheme}.k when spelling.

The common terminology for complexity of languages talks about \emph{isolating}
and \emph{synthetic} languages. In terms of our morphological complexity
measures isolating are languages that have average morph-to-word ratio around
1---2. The languages with high morph-to-word ratio are often called
\emph{poly-synthetic}. Another scale of terms is \emph{analytic},
\emph{agglutinative}, \emph{fusional}, and again, \emph{poly-synthetic}, to
refer to the kind of variation the morphemes tend to, in analytic cases the
morph-to-word ratio is low and therefore morphs do not have variations either,
the agglutinative languages have morphs that have little variation to extent
that typical language models are concatenated from separate morphs, and
fusional languages have very high rate of variation up to extent that variation
itself is considered a formative of new morph. 

These classifications lend themselves nicely to one nice view of the
spell-checking software, the languages that fall into synthetic and
agglutinative have most of the languages that have their special
implementations in world of open source spell-checkers; Finnish and Voikko,
Hungarian and Hunspell, Turkish and Zemberek, and so forth.  This tendency
suggests that people trying to build spell-checkers with limitations of
(nearly) word-list or statistics based systems will not be able to reach
satisfying results and other approaches are in fact necessary.

One, perhaps the most important form of morphological complexity in terms of
topic of this thesis, and one that is often ignored, is the rate of productive
derivation and compounding that forms recurring structures capable of
generating infinitely long words. For language where these processes are common
the spell-checking needs to support the infinite dictionary made by these
morphological processes at least to some extent to be useful. And surprisingly
many of both existing systems and those described in recent research simply do
not acknowledge the possibility\citep{}, some skimming over the fact as possible
extension to the system, some ignoring it altogether but usually without any
example of implementation and evaluation strictly around English or few other
popular indo-european languages with similar features.  The fact that lexicon
of many languages is infinite and that it has practical effects has been well
researched \citep{kornai2002many}.

To illustrate the practical issues with morphological complex languages we use
the de facto open source spell-checking system, hunspell. Of aforementioned
morphological features, hunspell supports up to 4 affixes per word, or 5
morphs, after that the very least user needs to combine multiple morphs into
one to get working system. The morpheme count in hunspell is realised as affix
sets, of which hunspell supports up to some 65,000 (the exact number supported
and reported to be supported may(?)\todo{check from repo, docs} vary a bit in
versions, it is not, however, 65,535 but slightly less). On more positive
note, the compounding part of infinite word-lists is supported by hunspell.

Of specific implementations for Hunspell.\todo{wordings} For example for
Finnish, there have been at least one failed attempt to build a
spelling-checker using hunspell \citep{pitkanen2006hunspell}, whereas we know
from early days of the finite-state methods in computational linguistics that
Finnish is doable, in fact it has been often suggested that the initial
presentation of automatic morphological analysis of Finnish in
\citep{koskenniemi1983twolevel} is one of the main factors for popularity of
finite-state methods among many of the related languages. Many of the languages
with similar morphological features however have been implemented also under
the hunspell formalism, indeed as even the name suggests, the formalism
originated as set of extensions to older *spell formalisms to support Hungarian
language. For North Sámi, there is a implementation for hunspell, The languages
that fall into poly- side of the morphological complexity are usually noted as
very hard to properly implement with current hunspell formalism and
limitations, though often the results are found tolerable. E.g. for mangdungbu
\citep{} have\todo{recheck name and cite} collected some 50,000 most common
word-forms and used that as a spell-checker, for which they say the coverage is
already 95 \%, so the spell-checker only misses every 20th word. Same does not
seem to apply for Greenlandic, where it was found that collection of 350,000
word-forms will only cover 35 \% of the word-forms in texts from the
newspaper\footnote{\url{http://oqaaserpassualeriffik.org/}}.\todo{Verify URL}

The sizes of word-lists lead to final, perhaps the most practical figure to
measure morphological complexity of the language in context of computational
applications. That is, the size of the dictionary when it is in the memory of
the computer, whatever may be the method of encoding it or compressing it; this
is easily measurable and very important factor of practical applications.
For example for word-list approach, even uncompressed the word-list of million
words with average word length of $N$ is just $N$ megabytes, storing the list in
finite-state automaton, suffix trie or hash considreably less. With derivation
and compounding the memory requirements tend to go up rapidly, so there is
a practical correlation measure between the complexity of the language defined
in theoretical terms earlier; in worst case scenarios this is obviously a 
limiting factor, e.g. some versions of my Greenlandic experiment taking up
to 3 gigabytes of memory, which is clearly unacceptable for a spelling checker.

\section{Finite-State Technology in Natural Language Processing}
\label{sec:finite-state-technology-in-nlp}

Overviewing the limitations and problems of the word-lists and such simple
approaches we come to second part of the thesis topic, the finite-state
methods.  It has been widely recognised for a long time that regular languages,
that finite-state automata encode, are of same expressive power as the natural
language morphologies, and therefore it is also the case that all language's
dictionaries can be without doubt encoded using finite-state automata, this has
been so widely acknowledged that the main proponents of the finite-state
approach have declared computational approaches to morphology ``a solved
problem'' requiring no further scientific research or discussion. And while
there is a certain amount of consensus that finite-state morphology is the way
to go, in real-world applications we commonly hit the limitations of the 
system in any case, and documenting these is a part of good research on topic.

The newest developments in the finite-state approach to natural
language processing has been the concept of weights in finite-state automata,
simply said bringing the expressive power of statistical language models to
well-tested rule-based models. One significant part of the contributions in
this thesis researches the notions of applying statistical approaches in
conjunction with morphologically complex languages using these 
methods.

The concept of modeling typing errors is not so widely accepted part of the
finite state technology, while for example in the speech applications, trying
to map a representation of the spoken audio signal into written language is
quite common, it is relatively new in the mapping of typing errors to correctly
typed text.

\section{Overview of the Articles}
\label{sec:articles}

This dissertation is a collection of articles I have written during the
development of different parts of the finite-state spell-checking system. It
covers a wide range of topics all tied together by the very practical goal of
getting things work nicely; catching up with the quality and speed of the
spell-checkers for languages like English. The articles, in chronological order
are:\todo{the enumerate might be automated as bib stuff}

\defcitealias{pirinen2009weighted}{I}
\defcitealias{pirinen2009weighting}{II}
\defcitealias{pirinen2010finitestate}{III}
\defcitealias{pirinen2010building}{IV}
\defcitealias{pirinen2010creating}{V}
\defcitealias{pirinen2011modularisation}{VI}
\defcitealias{pirinen2012compiling}{VII}
\defcitealias{pirinen2012effects}{VIII}
\defcitealias{pirinen2012improving}{IX}
\defcitealias{pirinen2013quality}{X}

\begin{enumerate}
    \item[\citetalias{pirinen2009weighted}]
        Krister Lindén and Tommi Pirinen.
        \newblock Weighted finite-state morphological analysis of Finnish 
        compounds.
        \newblock In \citep{conf/nodalida/2009}.
    \item[\citetalias{pirinen2009weighting}]
        Krister Lindén and Tommi Pirinen.
        \newblock Weighting finite-state morphological analyzers using HFST
        tools.
        \newblock In \citep{conf/fsmnlp/2009}.
    \item[\citetalias{pirinen2010finitestate}]
        Tommi~A Pirinen and Krister Lindén.
        \newblock Finite-state spell-checking with weighted language and error
        models.
        \newblock In {\em Proceedings of the Seventh SaLTMiL workshop on 
            creation and use of basic lexical resources for less-resourced 
        languagages}, pages 13--18, Valletta, Malta, 2010.
    \item[\citetalias{pirinen2010building}]
        Tommi~A Pirinen and Krister Lindén.
        \newblock Building and using existing hunspell dictionaries and {\TeX }
        hyphenators as finite-state automata.
        \newblock In {\em Proccedings of Computational Linguistics -
            Applications, 2010}, pages 25---32, Wis{\l}a, Poland, 2010.
    \item[\citetalias{pirinen2010creating}]
        Tommi A Pirinen, K.~Lind{\'e}n, et~al.
        \newblock Creating and weighting hunspell dictionaries as finite-state
        automata.
        \newblock {\em Investigationes Linguistic\ae}, 21, 2010.
    \item[\citetalias{pirinen2011modularisation}]
        Tommi~A Pirinen.
        \newblock Modularisation of Finnish finite-state language 
        description—towards wide collaboration in open source development of
        morphological analyser.
        \newblock In {\em Proceedings of Nodalida}, volume~18 of {\em NEALT
        proceedings}, 2011.
    \item[\citetalias{pirinen2012compiling}]
        Tommi A Pirinen and Francis M. Tyers.
        \newblock Compiling apertium morphological dictionaries with hfst and
        using them in hfst applications.
        \newblock {\em Language Technology for Normalisation of Less-Resourced
        Languages}, page~25, 2012.
    \item[\citetalias{pirinen2012effects}]
        Tommi~A Pirinen and Sam Hardwick.
        \newblock Effects of weighted finite-state language and error models on
        speed and efficiency of finite-state spell-checking.
        \newblock In {\em FSMNLP 2012\/} \citep{fsmnlp2012}, pages 6--14.
    \item[\citetalias{pirinen2012improving}]
        Tommi A Pirinen, Miikka Silfverberg, and Krister Lindén.
        \newblock Improving finite-state spell-checker suggestions with part of
        speech n-grams.
        \newblock 2012.
    \item[\citetalias{pirinen2013quality}]
        Tommi A Pirinen.
        \newblock Weighting Schemes for Language and Error Models in
        Finite-State Spell-Checking and Correction
\end{enumerate}

The development shown in the articles is easy to follow in the chronological
order, it almost tells its own story dictated by practical necessities. In the
beginning we have Finnish morphological analyser \citep{pirinen2008suomen},
which cannot quite simply be turned into a statistical dictionary for it
contains too rich morphological productivity, in articles
\citepalias{pirinen2009weighted,pirinen2009weighting} we explore some advanced
weighting options of the Finnish morphology that could give better treatment
for compounds and derived word-forms in morphologically complex languages. In
\citepalias{pirinen2010finitestate} we try to tackle the problem with scarcity
of the corpus resources especially in conjunction with morphologically complex
languages, using crowd-sourcing option provided by Wikipedia and measuring how
smaller text corpora will still benefit the spell-checking results. In
\citepalias{pirinen2010building,pirinen2010creating} and then again
\citepalias{pirinen2012compiling} I've researched the compilation and use of
existing non-finite state language descriptions as finite-state automata
language models of finite-state spell-checking systems. Specifically in
\citepalias{pirinen2010creating} I attempt to recreate much of Hunspell
spell-checking system, including the error correction part, in finite-state
form. In \citepalias{pirinen2011modularisation} I research the topic of
maintenance and upkeep of finite-state language models, and try to prove why
finite-state language models are feasible for long-term maintenance for vast
array of applications with just one language description instead of one per
application.  In \citepalias{pirinen2012improving} I tackled the problematic
issue of context-aware spell-checkers for morphologically complex and
resource-poor languages, showing and proving what practical limitations are and
what can be done to get some benefits of such implementation. In
\citepalias{pirinen2012effects,pirinen2013quality} we have already reached to
point of full-fledged, workable systems for spell-checking, and test the full
power of those systems, first for speed and efficiency in
\citepalias{pirinen2012effects}, then in a larger scale
survey~\citepalias{pirinen2013quality}, the quality of spelling suggestions, as
well as extensions to some of the speed measurements.
%As an epilogue, in I revisit the concept of Finnish as morphologically complex
%language and prove that limitations of at least Finnish for traditional
%non-finite-state approaches to spell-checking simply might not have been there
%after all.

Another way to conceptualise the thesis is under the following research
questions: How to implement reasonable statistical language models for
finite-state spell-checking in, how to implement a finite-state equivalent of
state-of-the-art software spell-checking, how to develop and maintain a
language model for finite-state spell-checking, what is the quality of
finite-state spell-checking compared to software-based approach, what is the
speed of the finite-state spell-checking compared to other software
based-approach, what are the limitations of the finite-state based approaches
compared to software-based spell-checking---and doing this all with
morphologically complex languages, that are minority, lesser resourced ones.

The rest of the thesis is organised by topic in the following chapters: In
\ref{chap:background} I will attempt to summarise the whole prior research of
spell-checking and correction, as well as describe the development of the
actual in-use spell-checkers to best of my knowledge. I will then describe the
previous research specifically on finite-state approaches to spell-checking,
and describe the theory of finite-state spell-checking in terms of our specific
implementation of finite-state spell-checking. The chapters
\ref{chap:language-models}---\ref{chap:efficiency} contain the research papers
of the thesis sorted under four headings that match the original goals of
thesis. This arrangement is summarised in the
figure~\ref{fig:articles-chapters}. In chapter~\ref{chap:language-models} I go
through the existing spell-checking and other language models and their use as
part of the spell-checking system, and introduce the finite-state point of view
to the language models that initially were software based.
In~\ref{chap:statistical-models} I show the weighted finite-state methods to
bring statistics to the language and error models. In~\ref{chap:error-models}
I study the finite-state formulations of error modeling and contrast my
implementations to the others that have been used. Finally, 
in~\ref{chap:efficiency}, I fully evaluate these systems on both speed and
quality to verify their usability in practical and real-world applications.
Most of the articles I've written fall neatly under one of these headings, but I do revisit few of them in more than one of the chapters. In the final 
chapter~\ref{chap:conclusion} I summarise the thesis and lay out the possible
future work in the field.

\begin{figure}
    \includegraphics[width=\textwidth]{diss-structure-uml-ish}
    \caption{Articles and chapters
    \label{fig:articles-chapters}}
\end{figure}

\chapter{Background}
\label{chap:background}

The spell-checking by computer is a topic that is already over half a century
old, and a one that has been researched periodically throughout that time.  In
this chapter I attempt to walk through the long history of spell-checking with
eye on both the scientific study of spell-checking, as well as the practical
end user systems for that. While I attempt to cover as much as possible on the
topic, there are bound to be many omissions and for fuller picture I recommend
reading previous surveys of spell-checking, such as
\citep{kukich1992spelling,mitton2009ordering}.  In~\cite{kukich1992spelling} the
history of spell-checking, both scientific improvements and some real-world
applications is very well summed up until the publication time of early 1990's.
In \citep{mitton2009ordering} there's further\todo{check}.
\citep{kukich1992spelling}\todo{was it her} also refers to spell-checking as a
`\emph{perennial topic}' in computational linguistics, I find this
characterisation quite accurate as it has been prevalent and recurring theme
throughout the history of computational linguistics and earlier.

After dealing with the history of the spell-checking I will proceed to the
current situation, and specifically the finite-state approaches to
spell-checking, and describe my particular system. I will also detail much of
the notation and terminology in these chapters and bleh\todo{needs rewording}

\section{Brief History of Spell-Checking and Correction; Prior Work}

The history of spell-checking and correction by computer can usually be timed
to have begun somewhere around 1960's, with inventions like Levenshtein's and
Damerau's measures of distances between strings
\citep{levenshtein1966binary,damerau1964technique}, or Damerau-Levenshtein
\emph{edit distance}, which even today is the core of practically all spelling
correction algorithms. This measure defines distance between two strings in
terms of editing operations performed to the string to match it to the other
string, the editing operations defined were following:

\begin{itemize}
    \item \emph{deletion} of a letter
    \item \emph{addition} of a letter
    \item \emph{changing} a letter to another, and
    \item \emph{swapping} or transposing adjacent letters
        (This is omitted from some formulations and sometimes ignored in
        spell-checking applications)
\end{itemize}

For example, distance between \emph{cat} and \misspelt{ca} is 1 (deletion of
t), distance between \emph{cat} and \misspelt{catt} is 1 (addition of t), and
so forth.  It is easy to see from these definitions how it is useful for
spelling correction; they are the model of basic typos or slips of the finger
on keyboard. The \citep{damerau1964technique} is often cited to be saying that
80 \% of the errors are covered by distance of 1, i.e. one application of this
edit algorithm would fix 80 \% of misspellings in a text. While this is indeed
the claim made in the article, citing it in context of modern spelling-checker
may neglect the difference between input mechanisms (and data transmission, and
storage) of computers of the time, although even that is dealt with in the
article.

Much of the research on error detection, rather than correction, concentrated
on efficiently looking up words from finite word-lists and building data
structures\citep{}. During this time, also the first statistical approaches, as
in \citep{raviv1967decision}, drawing from basic foundations of mathematical
theory of information from as early as
\citep{shannon1948mathematical}\footnote{I feel I must attribute
\citep{liberman2012noisily} for this tidbit of information}, were devised. In
this research the statistical theories are applied to characters of English
legal text, recognising also names and other terms. The input mode of this
application seems to be more towards OCR than keyboard input.\todo{is this
ocr?} This basic approach took the letter n-grams and assumed that words
containing unlikely letter combinations were not spelled correctly.

The very first spell checking software in common use, there are a few disputes
and claims to it, but commonly it is still attributed to the SPELL
program\citep{gorin1971spell}, which gained some popularity and common use. Some
of the earlier work has been used mainly by one research group for their own
purposes only \citep{earnest2011first,earnest2012first}. The first predecessors
of SPELL according to Les Earnest were systems for recognising hand-written
cursive text as auxiliary parts of larger software. Following this the SPELL
was possibly among first self-standing software for spell-checking, bearing in
mind differences of computational software of that time.

The features of SPELL are interesting, as its legacy has carried over to all
the spell-checkers for Unix systems that came after it up to current ones.
SPELL was already performing the correction using basic one error edit distance
kind of model, it included support of word lists and user dictionaries that it
could generate from missing words itself. And it also included (assumably
English\todo{wherever this can be checked?}) affix stripping. The limitation of
word-length was 40 alphabetical characters including hyphen and apostrophe. It
appears that the hash structure used for correction limits the search space of
the edit distance algorithm, i.e. the corrected word must exist in the same
hash as the misspelled word, which will cut some search space off after 3rd
letter of the word. The first two letters seem to be handled as with regular
edit distance matching. \citep{gorin1971spell}

Much of the following research on error detection consisted of more elaborate
data structures for fast lookup and efficient encoding of large dictionaries,
while interesting as computer science and data structures topic is not mostly
relevant to this thesis, so I will just summarise that it consisted of hashing,
suffix-trees and binary search trees, partitioning of dictionary by frequency
\citep{knuth1973art} and most importantly, finite-state automata
\citep{aho1975efficient}. While Aho's article is about constructing a bit
different finite-state automata than what we work with to match the
efficient Knuth-foo-bar\todo{check and cite} string matching algorithm, it's
a first implementation towards finite-state spell-checking.

The SPELL program's direct descendant is current international ispell
\citep{gorin1971spell}, the i originating from ITS SPELL, is still commonly in
use unix based systems. According to its documentation, the feature of suffix
stripping based on classification was added by Bill Ackerman in 1978, e.g. it
would only attempt to strip plural suffix \emph{-es} for words that were
identified of having such plural suffix.  This concept of affix flags is still
used in all of ispell's successors as well.

In the turn of 1990's there was a lot of new research on improving the error
correction capabilities, possibly partially due to rapid growth in popularity
of home computers. Most popularly the use of statistics from large text corpora
and large dataset of real errors that could be processed to learn probabilities
of word-forms and error types was applied and tested
extensively~\citep{kernighan1990spelling,church1991probability}. These popular
and simple metrics are still considered to be useful for modern spell-checkers.

One of the side-steps was to improve error models for English competence
errors, the initial work for this is the often cited Soundex algorithm,
originally meant for cataloguing names in a manner that lets you find similarly
pronounced names easily \citep{russell1918soundex}. It can also be used to match
common words, since it is a \emph{similarity key}, that maps multiple words
into one code and back, for example \misspelt{squer} and \emph{square} have the
same code \texttt{S140} and can be matched. The soundex ruleset is basically
about saving the first letter and assigning rest of non-adjacent non-vowels a
number, there have been some schemes to elaborate and make this work for
foreign names and more words, most notably metaphones
\citep{philips1990hanging,philips2000double}[third is commercial product without
publicly available documentation].

As the computational power increased it came practical to look again at the
problem of real-word spelling error detection. In \citep{mays1991context} it is
suggested that for English something something with trigrams. It was proven
that something\todo{figures and sutff}. The context-aware models are a
requirement for the discovery of the spelling errors, but independently of
that, they can also be used to improve the corrections of the spelling errors. 

In \citep{al2006learning} it is showed, that it is possible to learn the common
real-word errors directly from the text using correctly spelled reference texts
to calculate context factors for the words, and seeking words deviating from
the factors enough in the other texts, crafting some logic formulas for
replacements of the real-words.\todo{reread art.}

While computation power and storage space has quite raised along foo's
prediction until recently, it is still very much of debate if practical word
n-gram models are light enough to be used interactively in a desktop product
that is virtually on during whole uptime of modern desktop system, for latest
arguments read e.g. \citep{} that was published during the time of writing of
this thesis.

In world of context-aware models, simple word-form n-grams are not the only
form of context that has been used to improve---especially in context of
languages other than English it has often been noted that using sequences of
morphological analyses instead of the surface word-forms is more important
factor in detecting the errors and improving the results~\citep[for
Spanish]{otero2007contextual}. I've performed an experiment with this approach
in one of the articles as well, for Finnish.

The problems of implementing Hungarian with ispell, aspell and the like lead to
hunspell, with multiple affix stripping and compounding added. Similarly for
Turkish, various computational methods have been used. Among
those\citep{oflazer1996errortolerant} demonstrates one of the first finite-state
spell-checking with full-fledged finite-state language model---the earlier by
\citep{aho1975efficient} being more of a keyword search. This method used a
specialised search algorithm for finite-state network. In
\citep{agata2002typographical} the finite-state methods were extended to cover
the error-model as well, showing practical implementation of finite-state edit
distance, and finally in \citep{} it was shown that the trivial extension to
weighted finite-state automata for both language and error models for
spell-checking an correction is plausible for morphologically complex
languages.

% The methods proposed since then use a Bayesian approach (Golding
%(1995)) that may be combined with part-of-speech trigrams (Golding and Sch-
%abes (1996)), transformation-based learning (Mangu and Brill (1997)), latent
%semantic analysis (Jones and Martin (1997)), differential grammars (Powers
%(1997)), lexical chains (St-Onge (1995), Hirst and St-Onge (1995), Budanitsky
%(1999), Budanitsky and Hirst (2001)), and Winnow-based techniques (Gold-
%ing and Roth (1996, 1999), Roth (1998)). The two leading prior methods are
%the statistics-based BaySpell (Golding (1995)) and the Winnow-based WinSpell
%(Golding and Roth (1999))
% [[quoted from al2006learning]]

In the table~\ref{table:history-apps} I have summarised listed the practical
end user applications of spell-checking, the main academic studies of specific
practical and statistical~\citep{al2006learning} advances that are not so widely
used in the abovementioned apps, and the finite-state spell-checking.

\begin{table}
    \centering
    \begin{tiny}
    \begin{tabular}{|l|r|l|l|l|}
        \hline
        \bf Name & Year & Error Models & Language Models & Note \\
(Authors) [cite] & & & & \\
        \hline
        \multicolumn{5}{|c|}{\bf SPELL --- Unix --- FLOSS branch (*spell) }\\
        \hline
             SPELL, & 1971 & Edit-Distance 1* & Word-list, & Edit distance \\
\citep{gorin1971spell} &  & & English affixes & limited to \\
                                  &  & &              & 2 first letters \\
        ITS SPELL, & 1978 &  & Affix rules & full ED \\
     (Bill Ackerman) & & & & \\
        international ispell & 1988 & & & Non-English \\
              (Geoff Kuenning) & & & & \\
        \hline
        kspell, & 1998 & Metaphone 1 & Affix rules & \\
        GNU aspell & 2002 & Rule-weighted ED & Compounding & ED is avgd \\
    (Kevin Atkins) & & & (dropped) & w/ soundslike \\
        \hline
        myspell & 200x & weighted ED & 2 affixes & OpenOffice 1 \\
        \hline
        hunspell & 200x & weighted ED & 2 affixes & Configurable \\
                 &      & Confusables & Compounds & edits \\
        \hline
        \multicolumn{5}{|c|}{\bf Academic projects etc.} \\
        \hline
        correct & 1991 & Prob. ED 1 & Prob. word-list & Reranker \\
        \citep{church1991probability} & & & & \\
        \hline
        bayspell  & 1995 \\
        \citep{golding1995bayesian} \\
        \hline
        WinSpell & 1999 \\
        \citep{golding1999winnow} & \\
        \hline
        \multicolumn{5}{|c|}{\bf Finite-State} \\
        \hline
      \citep{oflazer1996errortolerant} & 1996 & ED & FSA & Fuzzy search\\
\citep{agata2002typographical} & 2002 & ED & FSA & Automata algebra \\
      \citep{schulz2002fast} & 2002 & ED & FSA & Levenshtein automata \\
        \citep{mohri2003edit} & 2003 & ED & WFSA & Automata algebra \\

      \citep{hulden2009fast} & 2009 & FSA Eqv & FSA & A* search, context \\
\citepalias{pirinen2010finitestate} & 2010 & WFSA & WFSA & Automata algebra \\
    \end{tabular}
    \caption{History of spell-checker applications \label{table:history-apps}}
\end{tiny}
\end{table}

\section{Related Subfields and Research Results}

\todo[inline]{Jaro-Winkler distance? Hamming distrance}

There are number of other research questions in field of computational
linguistics, that use same methodology and face the same problems. A large set
of research problems within computational linguistics can be formulated in some
frames that are directly relevant to the issues of spell-checking and
correction, as most of them require the basic component of a language model and
very often another one of an error model. The language model of spell-checking
predicts how correct a word-form is, and in linguistic analyser it predicts the
analysis and its likelihood. The error model of spelling corrector predicts
what is meant based on what is written assuming error in the process of typing
somewhere, part of the error sources, such as cognitive errors, overlap for
speech recognition and information extraction alike. Part of the error models
for other applications like diacritic restoration for information retrieval and
noise cancellation for speech recognition may solve different problems but use
approaches are still applicable. Realising this all at one point of the
research made me persistently require modular design from the spelling
correction---including strict separation of the spelling detection task, the
correction task---to mix and match the approaches found in various places to
the spelling correction. The rest of this subsection I shortly introduce the
specific approaches and algorithms from outside the spelling detection and
correction that I have re-used or tried to re-use in my experiments on
spell-checking and correction.

The relevant studies on finite-state approaches to predictive text entry have
been studied in~\citep{silfverberg2010partofspeech}, and the results of these
applications have been applied verbatim to our language models where
applicable.

The highly relevant sub-field of computer science is string algorithms,
specifically \emph{approximate string matching}. In practice, even this
thesis would be suitable contribution to the field of approximate string
matching, as it strongly builds on the algorithms developed within. The
approximate string matching using fuzzy search algorithms in finite-state
automata is one of the implementations I often contrast my work with, as
I have routinely replaced the fuzzy search algorithm by a composition of
well-specified fuzziness.

\section{Theory of Finite-State Models}

Finite-state models for spell-checking and especially correction are relatively
new---beginning from \citep{oflazer1996errortolerant}(?)---, and definitions
and interpretations have not been standardised yet, so in this chapter I will
try to go through the various definitions and explain my solution and how I
ended up with it. For starters I will use few paragraphs the recap the formal
definitions of finite-state systems and my selected notations, these should be
familiar to anyone who's dabbled in weighted finite-state theory e.g. from
\citep{aho2007compilers,mohri1997finitestate}.

The automata are conventionally marked in computer sciences and matheamtics as
systems, such as n-tuple $(Q, \Sigma, \delta, Q_i, Q_f, W)$, where $Q$ is the
set of the states in the automaton, $\Sigma$ is the alphabet in transitions,
$\delta$ is the transition mapping, and $Q_i, Q_f$ the subsets of states for
initial states, and final states and $W$ is the structure of weights. The
$\Sigma$ set in the automata of spell-checking system is almost always just
some---usually language specific for optimisation reasons---subset of the
Unicode set of symbols for writing natural languages, with the addition of
following special symbols, which have specific meaning in automata theory: the
empty symbol epsilon $\epsilon$ that matches zero-length strings on application
of the automata, the wild card symbol $???$\todo{gotta use some better symbol
here} that matches any one symbol of $\Sigma$ on application of automata.  When
we are talking of two-tape automata\footnote{I will systematically \emph{avoid}
    the term transducer in this thesis, a two-tape automaton is not special
enough to warrant use of special term over any other specialisation of
automaton, use of such will only confuse reader to think there is something to
it}, it merely means that the alphabet is of form $\Sigma^2$. The weighted
automata talked of throughout the thesis are using the \emph{tropical semiring}
weight structure $(\mathbb{R}_+ \cup \infty, min, +)$; this is so-called
penalty weight structure, which practically means that on application and
weight combination the smallest one is used, on combination of the weights they
are added together. The set of final states is extended with final weight
function $\rho$ that specifies additional weight to the path with each end 
state. For the rest of the finite-state algebra I use the standard notations
which to my knowledge do not have any variation that requires documenting in
this introduction (e.g. $\cup$ for union, $\cap$ for intersection and so forth).

A finite-state automaton, with $\Sigma$ set drawn from the set of natural
language alphabets, such as letters A through Z, digits 0 through 9 and the
punctuation marks hyphen and apostrophe, can accurately encode all words of
English language in accepting paths of the automaton. These kind of single-tape
finite-state automata recognising words of a language are used both in process
of detecting spelling errors of non-word type in running text, and matching
the misspelt word-forms to correct word-forms. The use of such automata as
language models is documented very extensively in natural language processing;
in context of morphologically complex languages more relevant to the topics of
this thesis the relevant reference reading would be the Finite-State Morphology~\citep{beesley2003finite,beesley2004morphological}.

The finite-state models are also used in error-correction, in this case the
finite-state automata are two-level automata that encode the relations from
misspellings to corrections in their accepting paths. There are few influential
work in field of finite-state methods for the error modeling, the initial work
may have been laid out by \citep{oflazer1996errortolerant}, which uses just
algorithmic approach on the language model traversal as an limited form of
error modeling; this has been extended and improved e.g. in
\citep{hulden2009fast}. 

An approach on weighted finite-state system was
described by \citep{mohri2003edit}, the definitions by Mohri are slightly more
technical and slightly different than the ones used in my work. \todo{explain
mohri's definitions and differences.}.

In \citep{agata2002typographical} the
finite-state error model as automaton was described. In that paper the research
concentrates on the concept of expanding the language model automata by
Levenshtein rule automaton, creating a language model automaton that can
recognise words containing given number of Levenshtein type errors, i.e. all
word-forms at given Levenshtein-Damerau distance. My definitions diverge here
by considering the error model as separate, arbitrary two-tape automaton; this
allows operating on either the language model or the misspelt string with the
error producing or removing functionality of the model, and gives chance to
test which variation is the most effective. 

Now we can also make the generalisation as to consider the strings of the
language that we are correcting as single-path single-tape automaton consisting
just one word, so we can formally define an unweighted finite-state spelling
correction system as composition $(w \circ M_e \circ M_c)^2$, where $w$ is word
to correct, $M_e$ the automaton encoding error model, and $M_c$ the automaton
encoding the language model. 

The weights in weighted finite-state automata are used to encode the preference
in spelling correction, and sometimes also acceptability in the spell-checking
function. A path that has larger collected weight gets pushed down in the
suggestion list and oens with smaller weights are in top. The weight can be
amended by the error model, there the weight would express the preference on
errors corrected when mapping the correct string to incorrect. One of the big
contributions of the research throughout the thesis is the theory and
methodology for creating, acquiring and combining these weights in a way that
is optimal for speed, and that is usable for morphologically complex languages
with limited resources.

The baseline for acquisition of the weights for language and error models is
simply calculating and encoding probabilities---this is what most of the
comparable products do in spell-checking and correction, and what has been
researched in statistical language models. The key formula for conceiving any
statistical process giving probabilistic distribution $P$ as a part of
finite-state automaton is $-log P$, i.e. the smaller the probability the bigger
the weight. The probability here is not straight-forward, but relatively
simple, for most things we calculate $P(x) = \frac{f(x)}{\sum_{x \in
\mathcal{D}} f(x)}$, where x is the event whose probability to count, $f()$ is
the frequency of event, and $\mathcal{D}$ the collection of all events, so the
probability of $x$ is counted as proportion of it in all $x$'s of
$\mathcal{D}$. For example, for simple language model $x$ could be word-form,
and $\mathcal{D}$ a corpus of running text turned into word-forms, then we'd
expect that $P('is') < P('quantitatively')$, for any reasonable corpus of
English language.

The important factor that morphologically complex languages bring to the
concept of statistical language models is obvious, the amount of different
plausible word-forms is greater, the data is more sparse and essentially
language models turn from finite word-form lists to infinite language models.
There's a good mathematical-lexicographical description of this problem in
\citep{kornai2002many}. This means that for simple statistical training models
the amount of unseen words raises, and unseen words in naive models mean
probability of $0$, which would pose problems for simple language models, e.g.
given above weight formula $-log(0) = \infty$ for any given non-zero-size
corpus, the practical finite-state implementation will regard infinite weight
as non-accepting path even if it were otherwise valid path in automaton. The
traditional approach to deal this is well documented and known documented,
assuming a probability distribution we can estimate the likelihoods of the
tokens that were not seen, discount a bit of the probability mass of the seen
tokens or otherwise increase the probability mass, and distribute it among the
part of the language model that would've been unseen. With language models
generating infinitely many word-forms the models may often not fit to their
probability distributions, in practical applications this does not necessarily
matter as the preference relations still work; a bit of this backgrounds on not
following strict statistical distributions in NLP context has been given by
Google in their statistical machine translation work e.g. in
\citep{brants2007large}.

The basic forms of estimating and distributing the probability mass to account
for unseen events has been researched extensively. The basic logic that I've
used in many of my research papers is the simplest known additive discounting:
here the estimated probability for event $x$ is seen as 
$P(\hat{x}) = \frac{f(x) + \alpha}{\sum_{x \in \mathcal{D}}(f(x) + \alpha)}$,
that is, each frequency is added by one and this mass of additions is added to
divisor to keep distribution in probabilistic bounds---this results that all
unseen tokens are considered to be seen $\alpha$ times and all others $\alpha$
more times than they would've been seen in reality. 



\chapter{Language Models}
\label{chap:language-models}

In this chapter I will go through the various kinds of language models that are
used for the task of spell checking and correction. Language models are used
for these two tasks in the spelling checker, and depending on the system,
these two tasks can either share a common language model, or use very
separate and different language models, or even approaches to apply that
language model, this separation is important one and I will try to make it
clear whenever I refer to language models in various systems and aspects of
the finite-state spelling checking. The primary purpose of language model in
both of these tasks is similar, to tell whether a word-form is suitable, and,
ideally, how suitable it is.

The purpose of this chapter is to show the development through the traditional
simple word-list spell-checking models to complex morphological analysers with
compounding and derivation, and their finite-state formulations. This follows
my initial goal to not only show off new language models and spell-checkers
in my thesis, but recreate the existing results of non-finite-state 
spell-checkers. In later chapters I will show that these formulations are
also as good as, or better than their algorithmically processed counterparts,
and the statistical methods we devise to support morphologically complex
languages apply for these finite-state automat 

The rest of the chapter is organised as follows: First I introduce the
finite-state compilation of Hunspell, the de facto standard of open source
spell-checking in section~\ref{sec:hunspell}.  Then I show the possible use of
rule-based machine translation dictionaries as spell-checkers in
section~\ref{sec:apertium}.  Finally, in section~\ref{sec:maintenance}, I
introduce a brief general treatment on management of linguistic data to have it
apply on multitude of projects in an article where I try to tie together the
different approaches to compiling and using the finite-state dictionaries and
computational linguistic models at large.

\section{Generic Finite-State Formula for Morphology}
\label{sec:generic}

Before delving further to the intricacies of compiling existing and new
formalisms into finite-state automata, I must remind one underlying formula
of computational morphology that is central to all these approaches. The
conceptualisation that all systems are regular combinations of sets of
morphemes, combined with rules of morphotactics and possibly morphophonology.
The basic startpoint is arbitrary combinations of morphs $\mathcal{M}_{morphs}
:= \bigcup ({morph})^\star$. The set of morphotactic rules over classifications
of morphs (e.g. hidden in the automata in the boundaries of morphs) is always
construed over $\mathcal{R}_{morph} \in \Gamma^\star << \Sigma^\star$ where
$\Gamma$ is the boundaries of morphs. The morphophonology likewise 
$\mathcal{R}_{phon} \in \Sigma \times \Sigma$ giving variations possibly bound
to morpheme boundaries for the letters. If you follow this train of thought
that all formulas generalise to this, the chapter will be more straighforward
to follow right up until the conclusion.

When dealing with finite-state language models for other purposes, e.g.
analysis of language, the compilation formula is exactly the same as shown for
the lexc language in e.g.~\citep{linden2009hfst}, keeping in mind that second
tape means cross-products in the formulas above. In the resulting language, the
language model for finite-state spelling correction only needs the surface tape
of the automaton, so it is possible to take a projection, or use composition
instead of intersection in the application phase of the finite-state
spelling correction.

\section{Compiling Hunspell Language Models}
\label{sec:hunspell}

\textbf{Main Article}: \emph{Building and Using Existing Hunspell Dictionaries
and \TeX\ Hyphenators as Finite-State Automata} by myself and Krister Lindén.
In this article we present finite-state formulations of existing spell-checking
language and error models.

\textbf{Motivation} of this piece of engineering is far-grasping both on 
technical and theoretical level. We set out to prove that the existing
methods for spell-checking are truly a subset of the finite-state methods.
Practically we also tried to show that typically, if not always, the 
finite-state version should be faster than software-based, e.g. for affix
stripping for sure, but also for error modelling. The practical motivation
for the work is that there is no chance for any new spell-checking system to
survive, unless it is capable of using the current hunspell dictionaries.

In \textbf{related works}, there has been to my knowledge, following two
attempts to use hunspell data as automata: a master's thesis project partially
guided by hunspell maintainers~\citep{greenfield2010open}, and a generalised
restriction compilation formula by Anssi Yli-Jyrä, for much more efficient form
of compiling the finite-state automata from the hunspell roots and affixes. The
original hunspell has been dealt with in academic context in
\citep{tron2005hunmorph}.

There is a number of meaningful \textbf{Results} in the article. This article
is central part of the thesis in that it provides the empiric proof that
finite-state spell-checking is proper superset of hunspell's algorithmic 
approach in terms of expressive powers, and the empiric results also suggest
improvement in speed over the board. The results on speed are detailed in my
latter articles and also in the chapter~\ref{chap:efficiency} of this thesis,
so here I will first concentrate on enumerating the parts of the first
result---the expressiveness and faithfulness of finite-state formulation of
the hunspell systems.

Hunspell's language model writing formalism clearly shows that it's based on
the main branch of the word-list based ispell spell-checkers. Words are
assigned flags that combine them with affixes, here the concept of affixes is
extended with context restrictions and deletions, but nevertheless the same
affix logic as its predecessors. The additional features brought to Hunspell to
set it apart from previous versions were the possibility to have more than one
affix per root---two for most versions, and then a number of various extra
features based on the same flags as used for affixation, such as compounding,
offensive suggestion pruning, and very limited circumfixation.  Majority of
these features come from the basic formula we use for bag of morphs and filters
style finite-state morphology as it is defined in\citep{linden2009hfst}. The
practical side of the main result shows that all of the language models are
around the same size as finite-state dictionaries and morphologies made with
other methods.

When talking about hunspell it is not possible to avoid the motivation behind
the development of a new system for spell-checking---as my thesis is also about
a new system for spell-checking. Hunspell was made solely on the basis that
existing spell-checking dictionary formalisms are insufficient to write a
Hungarian dictionary efficiently. Similar systems were written for many of the
morphologically complexer languages, however, finite-state approach is known
to defeat the problems here.

As the formula for context restrictions and deletions for the hunspell affix
morphotactics we used twol rules~\citep{karttunen1992two}, on the context
restrictions and its combinatorics it is well within reason, as that is one of
the central rule types in design of twol, but the deletions are not very easy
to follow in that formalism. As an afterthought, better and more efficient
representation would now be to combine morphotactic filters with replace style
rules~\citep{karttunen1995replace}, although neither compilation time nor the
clarity of algorithms are key issue for one-shot format conversion like this.

One of the main tumbling stones of many finite-state systems for full language
models is that---while the running of finite-state automata is known to be
fast---the building of finite-state automata can be lengthy. 


\section{Using Rule-Based Machine Translation Dictionaries}
\label{sec:apertium}

\textbf{Main Article:} \emph{Compiling apertium morphological dictionaries with
hfst and using them in hfst applications} by myself Francis M. Tyers. In this
engineering oriented article we experiment with the existing language models
used for machine translation as a part of the finite-state spell-checking
system.

The main \textbf{Motivation} for doing this experiment was to see how well we
can re-use the vast amounts of existing dictionaries from other projects as
finite-state automata in completely unrelated language technology systems.
This is especially interesting from my computer science background, since
the re-usability of code and data is something that I feel is lacking in
computational linguistics even now in 2010's.

As this experiment was built on existing, partially finite-state based system,
the \textbf{Related works} already contained a specific algorithm on building
the automata from XML dictionaries~\citep{rojas2005construccion} and its
features on fast and accurate tokenising and analysing~\citep{}. On that end,
our experiment does not provide any significant improvements, the final
automata are nearly the same and tokenisation algorithm used to evaluate
closely imitates the one described in the paper.

The compilation formula build in the article was merely a simplification of
the ones used in hunspell and lexc, making it relatively straightforward and
short as a techical contribution. The simple generalisation here is notable
however, as the computational side is based on solid standardised XML
format for dictionary representation and interchange, it makes it optimal
for future experiments.

The main \textbf{results} shown in the paper are the improvement on lexicon
processing time, which is in line with previous
results~\citep{silfverberg2009hfst}, and, more importantly to topic of this
thesis, results showing reasonably fast processing time of the finite-state
spelling checkers built in this manner; this is not totally unexpected as the
formalism does not support compounding or recurring derivation, and thus final
automata are acyclic. 

The one result that we tried to high-light in the paper was, that we could
already provide a rudimentary spell-checker based on lexical data build solely
for the purpose of limited range machine translation, and we could do that for
a range of languages that, at the time of writing lacked the spell-checkers
altogether. And the task of writing the compiler for this existing format
based on other morphology formalisms we had made and turning the compiled
language model to baseline spell-checker was not more than few days work.

A side result that I partially failed to communicate in this paper is, movement
towards generalised compilation formula of the finite-state dictionaries. The
compilation formula in this article is a variation of the earlier formulas I
have presented~\citepalias{linden2009hfst,pirinen2010building}---I suggest this
has two implications: for engineering side, we can use same algorithms to
compile all different dictionaries, which means that automatic improvements
and optimisations that can be applied to the structure of the automata will
improve all applications. More importantly, it means that the different
applications' formalisms for writing dictionaries use same linguistic models
and same kind of (lack of) abstraction is present. This suggests that there is
potential for generalisations such that different applications could converge
using a single, linguistically encoded dictionaries.

\section{Other Possibilities for Generic Morphological Formula}

There is an abundance of formalisms for writing dictionaries, analysers and
language technology tools available, and I have only covered here the few most
prominent for the use of finite-state spell-checking. For example I have shown
compilation formula for hunspell formalism. The aspell and ispell formalisms
are simplifications of this, and the formula can easily be simplified for them
by removing the multiple suffix consideration and compounding loop in the
bag-of-morphs construction: $\mathcal{M}_L := \mathcal{M}_{prefix}
\mathcal{M}_{roots} \mathcal{M}_{suffix}$

\section{Maintenance the Language Models}
\label{sec:maintenance}

\textbf{Main Article:} \emph{Modularisation of Finnish finite-state language 
description—towards wide collaboration in open source development of
morphological analyser} by myself. This short paper is mainly an opinion
piece style article, but it is included here as it ties together the
aspect of maintainable language models for spell-checking application.

One of the main \textbf{motivations} for writing this article was the
realisation after some years of working on the finite-state language model
compilation and harvesting different computational dictionaries and
descriptions of morphologies, that there is a common pattern in all the 
data---as one generalised finite-state algebraic compilation algorithm works
for them all. An opposite but exactly as strong motivation was, that for
lesser resourced language a single computational language model is very
precious resource, that needs to be re-used for all applicable applications,
as there are no resources to rewrite the descriptions separately to all 
applications. This should motivate for computational linguists in general to
work towards re-usable, well-written resources, with proper abstraction levels
and the basic knowledge representation.

Another part of motivation is that the language descriptions, whether for
finite-state morphological analysis or hunspell spell-checking are typically
written by one linguist, and left to rot as no one can read them anymore.
This is very unuseful approach in the crowd-sourcing world of today, and it
could surely be improved by simple improvements to formalisms and practices.
In this branch of my research I argue for building more crowd-sourceable 
systems, in the form of linguistics that are understandable for anyone
fluent in the language whenever it is possible.

Looking into the \textbf{related works} I soon realised that I was not totally,
though admittedly mostly, alone in my feeling, that the finite-state and other
computational language descriptions need a bit of thought to be maintainable
and re-usable. The two notable computational linguists writing on same topic
that I came across immediately when searching into the question were Mike
Maxwell~\citep{maxwell2008joint} and~\cite{wintner2008strengths}. The common
features of all our writings include indeed the lack of abstraction, and
therefore very application specific hacks in language descriptions that could
ideally be very general. This also leads me to suggest that related works that
are very under used in the field of \emph{computational} linguistics are the
basic schoolbook materials on computer science and software engineering;
Knuth's pursuit on literate programming~\citep{knuth1984literate} for
well-structured and documented programs---as that's what computational language
models really are, rather than just data. And the whole object-oriented
programming movement, for encapsulation, abstraction and management of the data
for re-usability and other principled purposes.

The main \textbf{results} of this paper are seen on the continuity of Finnish
language model and its relatively good amount of repurposing over the years.
Beyond that, the common movement towards maintainable and re-usable language
models in computational linguistics and morphology specifically is still very
much work in progress.

\section{Conclusions}



\chapter{Statistical Language Models}
\label{chap:statistical-models}

One of the goals of my thesis was to bring the statistical language models
closer to usability for morphologically complex languages with lesser
resources. For this purpose I sought into trying out the most typical
traditional statistical approaches turning them into finite-state form
using the well-studied algorithms of weighted finite-state automata. As for
most morphologically complex languages the improvement gained by bringing
statistical data to the language models into system is not as optimistic as
for English, I then experimented with various ways of improving the quality
by more efficient use of linguistic data and simply fine-tuning the parameters.

The problem that arises from the combination of morphologically complex 
languages and the lesser resourced languages is two-fold, and I go through two
separate approaches to tackle this. The first problem can be alleviated by
raising other---more linguistically motivated objects---to the primary training
feature rather than plain running word-forms. This in general is of course not
problem-free, as the general solution would then require analysed, 
disambiguated language resources for training, which requires manual labour to
produce, whereas surface word-forms come for free. So instead of requiring
manually annotated data I've studied into salvaging what can be easily used
from running text without the need of deeper analysis---in Finnish this meant
word-forms and roots for compound parts. Other part of the problem I tried to
deal with showing that even a limited resource of lesser quality---that
happens to be available under free and open licence---can be used
in conjunction with well-formed rule-based material to greatly improve results
of spell-checking.

The basic statistical language model is a simple structure that can be trained
using a corpus of texts that contains word-forms. For spell-checking a simple
logic of suggesting a word-form that is more common before one that is less, is
usually the correct approach, and in majority of the cases leads into correct
suggestion. This is formalised into the weighted finite-state form by a simple
probability formula: $W = -\log(\frac{c(wf)}{CS}$, where $W$ is the weight in
tropical semi-ring structure, $c(wf)$ is the count of word-forms  $wf$ in the
corpus and $CS$ is the count of all word-forms in corpus. The morphological
complex languages are estimated to be more difficult to train because the
number of the different word-forms is greatly larger than with morphologically
simple languages, which leads to decreased discriminative capabilities for
the probability calculations, as the number of word-forms with exactly same
amount of word-forms increases. In my research I have studied some approaches
to deal with this situation. The thing that further worsens this situation is,
that morphologically complex languages also have lesser resources. Much of the
contributions I've provided here concentrate on using the scarce resources in
a way that will give nearly the same advantage from the statistical training
as one would expect for morphologically poor languages with more resources.

The more complex statistical language models, such as ones that use
morphological, syntactic or other analyses, require a large amount of
manually verified data, that is not in general available for majority of the
world's languages. In my research I present some methods to make use of the
data that is available, or unverified, to get improvement to the language
models, when possible. In general I try to show that the significance of such
quality analysis data is not essential to typical spell-checking systems, but
more for fine-tuning of the final product and building of the grammar-checking
systems out of spell-checkers.

In this chapter I will describe the statistical methods I needed to introduce
to get good statistical language models for morphologically complex languages,
specifically Finnish, but could be applied for others with reasonable
modifications and generalisations; the main article I discussed this in
was~\citepalias{pirinen2009weighted}. Then I discuss the techniques behind training
morphologically complex languages, the relevant article for this discussion is
\citepalias{pirinen2009weighting}. Finally I talk about the common problem with the
combination of less resources and more complex morphology actually requiring
more resources in~\citepalias{pirinen2010finitestate}, where I propose the use of
Wikipedia as the main free and open source corpus for training of the language
models.

\section{Language Models for Languages with Compounding}

\textbf{Main article}: \emph{Weighted Finite-State Morphological Analysis of
Finnish Inflection and Compounding} \citepalias{pirinen2009weighted} by Krister
Lindén and myself, introduced the statistical training of morphologically
complex language with infinite lexicon using variation of basic unigram
training methods.

The original \textbf{motivation} for this research was to implement the
compound segmentation for Finnish given the ambiguous compounding of previous
implementation \citep{pirinen2008suomen}. In the article we established that
Finnish as morphologically complex language requires special statistical
handling for its language models to be in same ballpark as simpler statistical
models for morphologically simpler languages.

In \textbf{related works}, the compound segmentation problem has been solved
for Swedish language by selecting the compounds with least boundaries
systematically \citep{karlsson1992swetwol}, and for German by using statistics
words \citep{schiller2006german}. In the article we showed a generalisation to
both approaches, where the statistics are learned from the surface word-forms
in running text, instead of pre-analysed and disambiguated corpus and manual
selection routine.  Showing that this method worked built two important
contributions to the language models of morphologically complex languages:
firstly, the learning can be made with running unanalysed texts, which is
crucial, because the lack of good materials for most of the lesser-resourced
languages, and secondly, the application of basic statistical formulas to
morphologically complex languages was an important generalisation.

The \textbf{results}, as stated in the motivation, the experiment framed in
article was built on disambiguation of the compound boundaries, rather than
building language models for general applications. The weighting mechanism
however was applied to the language model of Finnish for all the future
research in wider set of topics. It is only later research that I did in fact
find out that the statistical method of weighting morphologically complex
language models based on the word constituents of found in the text outside
their full word form compound forms, that partially solves the problem of
training statistically the language models of compounding languages.

In the article I study the precision and recall of the compound segmentation
of Finnish compounds using the statistical language modeling scheme. The
results shown in the article are rather promising; nearly all compound
segmentations are correctly found even with the baseline approaches of giving
rule-based weights to morphological complexity. There's a slight improvement
when using the statistically formed system that gives slightly higher
granularity in few cases. There is one caveat in the material used in the
paper, since Finnish lacks gold standard material for any analysed text
corpora really, we used a commercial corpora made by another combination
of automatic analysers. This means that the precision and recall values we
get are measuring how closely we imitate the referent black box system, rather
than real world compound segmentations.

As conceivable \textbf{future research}, there is a generalisation to this
method I have not yet experimented with; the constituents that I used for
Finnish were word-forms, this is based on the fact that Finnish compounds are
of relatively simple form: the initial parts are same as surface forms of
regular singular nominatives or any genitives (productively) or any form
(rarely), this is not the case for all other languages, notably even German and
Swedish have commonly a compositive marker only shown in compound forms and not
in surface forms. There are few of those in Finnish as well, e.g. words ending
in some `-nen' final derivation will compound with `-s' form not seen outside
compounds and some archaic compositives exist for few words. Likewise in
languages like Greenlandic, there is no compounding forming the complexity, but
only recurring inflection and derivation as it is. The generalisation that
might be needed here is to train the language model based on \emph{morphs}
instead of uncompounded word forms. I estimate this would be accurate on
majority of morphologically complex languages and even up the statistical
playing field, since the variation between languages for number of forms is
less\todo{reword}.  There is a previous research on such language models based
on not morphs, but letter n-graphs, or statistically likely affixes
\citep{creutz2005morfessor}.

\section{The Statistical Training of Compounding Language Models}

\textbf{Main Article}: \emph{Weighting Finite-State Morphological Analyzers
using HFST tools} by Krister Lindén and myself. In this article we set out
to provide generic formulation of statistical training of finite-state
language models regardless of complexity.

\textbf{Motivation} of this piece was to streamline and generalise the process
of creation of the statistical finite-state language models, as the experiment
in previous article \citepalias{pirinen2009weighted} the compound disambiguation
scheme was build in very ad hoc manner. The intention of the new method
for training of the language models is to get all existing language models
trainable regardless of how they are built; this makes the process of training
of a language model very much alike it had been in all non-finite-state
software to date.

In \textbf{Related works} there is the whole vast field of literature of
statistical language models per se. The concept of having good language model
and using data from corpora to train it is the very basics of any form of
statistical natural language engineering so it is easily covered in any
elementary text book, e.g. \citep{manning1999foundations}. For related software
see e.g.  SRILM~\citep{stolcke2002srilm},for the more popular ones in open
source statistical language processing.

The existing work on training infinite dictionaries is more rare, as it is not
needed for English. For German this is dealt with in \citep{schiller2006german},
which also uses finite-state technology in its implementation.

As initial \textbf{Results} in the article I replicated the earlier compound
segmentation task from, to show that generalised formula works.
Furthermore I experimented on the traditional POS tagging task to see how the
compound ambiguity actually affects the quality of POS tagging. The result
shows an improvement which proves that statistical lexeme based compound
modeling is actually a useful part of statistically trained morphological
analyser. This has the practical implication that it is indeed more better to
use the compound-based unigram model as baseline statistical trained analyser
of a language, rather than simple word-form training based statistical model.

In general the unigram language model with the addition of word-form based
training for compounds shows a good promise for the method of training
morphologically complex languages from morphologically relevant parts instead
of plain word-forms. And as the morphological sub-parts of the word form in
a way a sub-n-gram structure, it is possible to use the whole learnt knowledge
from word-form n-gram research to apply the methods. For future research it
would be extremely interesting to see a non-compound-heavy morphologically
complex language, such as Greenlandic to be trained from morphs instead of
word-forms and see how it fares.

\section{Weighting Hunspell as Finite-State Automata}

\textbf{Main Article}: \emph{Creating and Weighting Hunspell Dictionaries as Finite-State Automata} by myself and Krister Lindén. In this article we present
weighted versions of finite-state hunspell automata. This is also the main
article introducing the weighted finite-state spell-checking for first time
on larger scale.

The main \textbf{motivation} for this experiment was to work on extending the
previously verified results for weighting finite-state language models to
newly created formulations of hunspell language models as finite-state
automata. This step would be of crucial justification for current users of
traditional software-based hunspell users to move forward, along the already
documented increase in efficiency on most of the cases. 

\textbf{Related works} describing probabilistic or otherwise preferential
language models for spell-checking. This branch of research would've started
as early as \citep{}. The main source for current version of 
context-insensitive spelling correction using statistical models comes
probably from \citep{church1991probability}.

Main \textbf{Results} in the article were measured from large range of hunspell
languages for precision, showing improvement of quality over the board with
simple unigram training methods over the hunspell automata. A methodogical
detail about the evaluation should be noted, the errors were constructed using
automatic methods, in vein of ones researched in
\citep{bigert2003autoeval,bigert2005automatic}. This gives a clear indication
that the statistical approach for language models of spell-checking do improve
the overall quality in its most straightforward setup.

The implication of these results to the large picture of this research is
that it may show that the finite-state formulation of hunspell language models
is a reliable way forward for hunspell spell-checkers, providing the additional
benefit of statistical language modeling to husnpell's well-engineered
rule based models. This would be important for practical development of
the next generation spell-checking systems, since the creation of dictionaries
and language models requires some expert knowledge, which is scarcely available
for most of the lesser-resourced languages.

\section{Lesser Resourced Languages in Statistical Spell-Checking}

\textbf{Main Article}: \emph{Finite-State Spell-Checking with Weighted Language
and Error Models} by myself and Krister Lindén. In this article we first
researched the problem from point of view of less resourced languages. Even
though Finnish does not have significantly more freely usable resources this
is first explicit mention of the problem in my research.

\textbf{Motivation} for this article was to write a publication that
explicitly explores the difficulties that morphologically complex and lesser
resourced languages face when trying to peruse statistical language models and
corpus-based training. The article is well-placed since contemporary research
on natural language engineering often considers corpus-training as a kind of
solution for all problems ignoring the scarcity of resources at one point and
the much higher requirement of unannotated resources for morphologically
complex languages on another.

In \textbf{Related works}, the concept of lesser resourced languages in
computational linguistics was coming to focus in recent years. Particularly I
followed Anssi Yli-Jyrä's recent work on African
languages~\citep{yli2005toward} and of course the workshop organisers' work on
speech and translation technology.

The concept of using Wikipedia as statistical language model, or other
resources was likewise new common theme in the LREC conferences of that
year and the following.

The most central \textbf{Results} of this work show that the Wikipedias for
small language groups, despite being comparably tiny and not very high quality,
give significant gains to spell checking language models when properly
coupled with rule-based language models. I've hypothesised that the underlying
reason for this is that the events where simple spell-checking like edit
distance or confusion sets produce ambiguous results at same length, in
majority of the cases it is beneficial to vote for the more common word-form.

As a side result in this article I show how to use the language model to
automatically generate the baseline error models. This is expanded in
the section~\ref{chap:efficiency}.

In general the concept of using freely available and open sourced---and
crowd-sourced---data should be common focus in the research of computational
linguistics in lesser resourced languages. This is one of the main sources of
large amounts of free data that most communities are willing to put effort for,
and it does not require advanced algorithms or expert proofreader to make the
data usable for many of the applications of computational linguistics.

\section{Conclusions}

In this chapter, I have discussed about different approaches to create language
models for finite-state spell-checking and, in general, about creating and
maintaining the weighted finite-state models that are up to the standards that
are expected of language models overall. 

In general I have shown in this chapter that it is possibly to take arbitrary
language models for morphologically complex languages and train it with same
approaches regardless of the method and theory the original model was built in.


\chapter{Error Models}
\label{chap:error-models}

The error modeling is an important part of spell-checking system and in my
opinion deserves to be treated as a whole separate subtopic of its own. The
finite-state formulation of error models in finite-state spell-checking system
in particular is relatively novel idea, with only some theoretical works
\citep{agata2002typographical,mohri2003edit} and single examples in
books~\citep{beesley2003finite} actually dealing with it. The error modeling
however has been central topic for software based solutions to spell-checking
(cf.~\citep{kukich1992spelling,mitton2009ordering,deorowicz2005correcting}), so
I believe that proper finite-state solution to it is in place.

One of the typical argument for not using finite-state error models is that
special software algorithms are faster and more efficient than building and
especially applying the finite-state models. In the articles where I build
the finite-state models for spell-checking I have ran some evaluations on
resulting system to prove that finite-state system is at usable speed
and memory consumption levels, the thorough evaluations of this however
are in the \ref{chap:efficiency}.

The theoretical and practical motivation for suggesting finite-state models for
error correction is obvious: the expressive power of weighted two-tape 
automaton is an ideal format for specifying the combinations of the
spelling errors that can be made and corrected, and since weighted
finite-state automata are closed under finite-state algebra it is easy
and theoretically nice to build even statistical models of different types
of errors and combine them from building blocks. The fact that traditional
software based spelling correctors use number of different algorithms starting
from all the variants and modifications of the edit distance to arbitrary
context based string rewriting to phonemic folding schemes, provides a good
motivation to have such arbitrarily combinable systems for creating these
error models outside the software level.

The rest of this chapter is laid out along following articles: in~\ref{sec:hunspell-error} I go through the existing error models used in the
popular spelling software \texttt{hunspell}, and show their finite-state
formulations.  Finally I discuss the context-based language models in
finite-state spelling correction for morphologically complex languages with
limited resources, that is, mostly the limitations and necessary modifications,
this was covered in \citepalias{pirinen2012improving}.

\section{The Application of Finite-State Error Models}

There are many approaches to spelling correction with finite-state automata.
It is possible to use various generic graph algorithms to work on fuzzy
traversal of the finite-state graph, like is done in~\citep{hulden2009fast}.  It
is also possible to create a finite-state network that already includes the
potential error mapping, as is shown in~\citep{schulz2002fast}. The approach
that I use is to apply basic finite-state algebra with a two-tape automata
working as the error model. This is a slight variation of model originally
represented by Mohri in~\citep{mohri2003edit}. We apply the errors to misspelled
strings by two compositions, performed serially in single operation:
$\mathcal{M}_{Sugg} = (\mathcal{M}_{input} \circ \mathcal{M}_{error} \circ
\mathcal{M}_{language})_1$.  This specific form of finite-state error
correction was introduced in~\citepalias{pirinen2010finitestate}, but the underlying
technology and optimisations were more widely documented in
\citep{linden2011hfst}.

The implication of using weighted finite-state algebra is two-fold. On the
other hand we have full expressive power of the weighted finite-state systems.
On the other hand, the specialised algorithms are known to be faster, e.g.
the finite-state traversal shown by Huldén in~\citep{hulden2009fast} is both
fast and has possibility to have regular languages as contexts, but no true
weights. So the trade-off to select is not simple, but considering the
popularity of statistical approaches in the field of natural language
engineering the time trade-off may be palatable for most end-users.

The implication of the fully weighted finite-state framework for spell-checking
is that we can e.g. apply statistics for all parts of the process, i.e.,
the aprioric probability of input, the conditional probability of specific
combination of errors and the probability of the resulting word-form in 
context. There are possibly other extensions achievable in this setting, but
this is the extents of the system I am describing in the thesis and referred
articles.

\section{Edit Distance Measures}

The baseline error model for spelling correction since its inception in 1960s
has been Damerau-Levenshtein edit
distance~\citep{damerau1964technique,levenshtein1966binary}. There has been
multiple formulations of the finite-state edit distance, but in this section
we describe the one that is used by our systems. This formulation of
edit distance as two-tape finite-state automaton was first presented in
\citep{schulz2002fast}.

The Finite-state edit distance is rather simple as a concept, each of the
error types except swap is represented by single arc of form $\epsilon:x$
(for insertion), $x:\epsilon$ (for deletion) or $x:y$ (for change). The
swap requires additional final state in the automaton to remember the
symbols to be swapped at price of one edit, that is a path $\pi_{xy:yx}$.

There are numerous possibilities to formalise the edit-distance in weighted
finite-state automata. The simplest form is actually a cyclic automaton that is
capable of measuring arbitrary distances, using the weights as distance 
measure. This form is created by drawing arcs of form $\epsilon:x::w$,
$x:\epsilon:w$, from the initial and final state to itself, 
where $w$ is weight of given operation. If measure includes the swap of 
adjacent characters, the extra states and paths of form $\pi_{xy_yx}$ need to
be defined with the weights encoded to additional end states and returning arc
as needed. Without swaps, the arcs $x:y::w$ can be also drawn from starting
and ending state to itself, making the resulting automaton a single-state
automaton. It is possible to use this automaton to help automatic harvesting
or training of error model. The practical weighted finite-state error models
used in real applications are of same form as the unweighted ones defined
earlier, with limited maximum distance, as application of the infinite edit
measure is expectedly slow.

In~\citep{mohri2003edit}, Mohri describes mathematical backgrounds on
implementing a weighted edit distance measure for finite-state automata. It
differs slightly from the formulation we use, in following aspects: \ldots

\section{Hunspell's Error Modeling}
\label{sec:hunspell-error}

\textbf{Main Article}: \emph{Building and Using Existing Hunspell Dictionaries
and \TeX\ Hyphenators as Finite-State Automata} by myself and Krister Lindén.
In this article I've formulated the hunspell's software based algorithms for
error correction and their specific optimisation tricks in terms of
finite-state automata.

The main \textbf{motivation} for this research was to recreate hunspell's
spell-checking for finite-state form. In terms of error modeling it basically
consists of modification of edit distance for speed gains. The nature of
these modification are based on some popular findings on the spelling errors
that are made. One modification is to take the keyboard layout into account
when considering the replacement type of errors. Another addition is to
sort and limit the alphabet used for other error types. Finally hunspell
allows the writer of the spell-checking dictionary to define specific
string-to-string mutations for common errors.

The finite-state formulation of these errors is a combination of limitations
and modifications to edit distance automaton, disjuncted with simple
string-to-string mapping automata. The key aspect of hunspell's modified
edit distance formulations are optimising the speed of the error lookup using
configurable, language specific, limitations of edit distance algorithm's
search space. The other correctional facilities are mainly meant to cover the
type of errors that cannot be reached with regular edit distance modifications.
The effects of these optimisations are further detailed in the 
chapter~\ref{chap:efficiency}.

The settings that hunspell gives to user are following: \texttt{KEY} to
limit characters that can be \emph{replaced} in edit distance algorithm,
meaning to optimise by limiting the replacements to adjacent keys on a
typical native language keyboard. \texttt{TRY} to limit the possible characters
that are used in the insertion and deletion part of the edit distance
algorithm. \texttt{REP} can be used for arbitrary string-to-string confusion
sets, there is a parallel system to this called \texttt{MAP}, used for
encoding differences, rather than linguistic confusions. Finally, for English
language, there is a \texttt{PHONE} setting, that implements some variation
of the double metaphone algorithm~\citep{philips2000double}.

\section{Phonemic Key Corrections}

The phonemic keys are commonly used for language where orthography does not
have very obvious mapping to pronunciation, such as English. In the phonemic
keying methods the strings of a language are attempted to connect with phonemic
or phonemically motivated description. This technique was originally used for
sorting family names in archives~\citep{russell1918soundex}. These systems have
been successfully adapted to spell-checking and are in use for English for
example in aspell as the double metaphone algorithm~\citep{philips2000double}.
Since these algorithms are a mapping from strings to strings, it is trivially
modifiable into a finite-state spelling correction model. Considering a
finite-state automaton $\mathcal{M}_{phon}$ that maps all strings of a language
into a single string, the spelling correction can be performed with mapping
$\mathcal{M}_{phon} \circ \mathcal{M}_{phon}^{-1}$.

\section{Context-Aware Spelling Correction}

\textbf{Main article:} \emph{Improving finite-state spell-checker suggestions
with part of speech n-grams} by myself, Miikka Silfverberg, and Krister
Lind\'{e}n. In this article I've made an attempt to cover spelling correction
of morphologically complexer languages with finite-state technology.

\textbf{Motivation} for this article was to show that finite-state version of
spell-checking is also usable for context-aware form, and secondarily to
explore the limitations and required changes for the context-based language
models that are needed for morphologically complexer languages to get similar
results as are attained with simple statistical context models for
morphologically poor languages.

The \textbf{related works} include the well-known results on English for same
task implemented in software approaches.  In general, the baseline of the
context-based spelling correction was probably established
by~\citep{mays1991context} and has been
revisited~\citep{wilcox-ohearn2008realword} many times, and usually found to be
beneficial for the languages that have been researched. The examples I have
found are all of morphologically poor languages: English, Spanish. In my
research I have tried to replicate the results for Finnish, but the results
have not been as optimistic as with English or Spanish for example.  The basic
word-form n-grams were not seen as efficient method for Spanish in
e.g.~\citep{otero2007contextual}, instead the method was extended by studying
also the part-of-speech analyses to rank the suggestion lists for
spell-checker. In the article~\citepalias{pirinen2012improving} we reconstructed the
finite-state system very similar to the one used for Spanish, originally
unaware of these parallel results.

\textbf{Results} show that with the regular n-grams that are used for English
to successfully improve the spelling correction are not as effective for 
Finnish. Furthermore the improvement that is gained for Spanish in application
of the POS n-grams in this task does not give equally promising results for
Finnish, but do improve the quality of suggestions, by around 2~\% 
units\todo{check}.

One of the negative results of the paper is that the time and memory
consumption of the POS based or word-form based spelling correction are not yet
justifiable for general consumer products as a good tradeoff. There have been
many recent articles exploring on the optimisatory side of the context-based
spell-checking that should be seen into as future work for context-based
finite-state spelling correction. The optimisation of the n-gram models has
been brought up many times with non-finite-state solutions too,
e.g.~\citep{church2007compressing}. While mainly an engineering stunt, there
is a lot to be deserved from this development before practical end-user
applications for spell-checking are to be considered.

\section{Other Finite-State Error Models}

In~\citep{deorowicz2005correcting} there is an extensive study on error
modeling for spelling correction. Some of the models possible have been
introduced earlier in this chapter. There are further

One of the optimisation tricks used in some practical spell-checking systems,
but not in hunspell, is to avoid modifying the first character of the word in
the error modeling step. The gained improvement in speed is notable, the
rationale for this is the assumption that it is rarer to make mistakes at the
first character of the word---or perhaps easier to notice and fix such
mistakes. There are some measurements on this phenomenon
in~\citep{bhagat2007spelling}, but the probabilities of having the first-letter
error in referred papers vary from 1~\% up to 15~\%\todo{check}, so it is not
exactly decisive. In measurements I made for morphologically complex languages
in~\citepalias{pirinen2012improving}, I also got slight tendency towards the effect
that the errors are more likely in non-initial parts of the words, accounting
word length effects.

The concept of creating error model from (possibly annotated) error corpus is
doable for finite-state spelling-correction as well. The modification of the
training logic as presented by \citep{church1991probability} into finite-state
form is mostly trivial.

\section{Conclusions}

In chapter \ref{chap:error-models}, I discussed the formulation of error
correction in finite-state spell-checking as basic finite-state automaton. 

\chapter{Efficiency of Finite-state Spell-checking}
\label{chap:efficiency}

One of the motivations for finite-state technology is that the time complexity
of application of the finite-state automata is known to be linear with regards
to its size, and finite-state automata generally can be optimised by operations
like minimisation. In practice, however, the speed and size requirements of
finite-state automata required for language models, error models and their
combinations are not easy to predict, and the worst case scenarios of these
combinatorics specifically are still exponential. This is especially important
for error modelling in spell-checking, since it easily goes towards the worst
case scenario, i.e. the theoretically ideal naive error model that can rewrite
any string into any string with some probabilities attached is not easily
computable.

Towards the latter parts of my research on finite-state spell-checking I set to
write few larger scale evaluation articles to prove the efficiency and
usability of finite-state spell-checking. For larger scale testing I also
sought to show how the linguistically common but technologically almost
undealt with concept of morphological complexity practically relates to the
finite-state language models.

The evaluation of the finite-state spelling correction needs to be contrasted
to existing comparisons and evaluations of spell-checking systems. Luckily, the
spell-checking has been a popular topic enough to have number of reasonable
markable evaluations done, ``a perennial topic in computer science'', to cite
one of such survey~\citep{kukich1992spelling}. Also the very informal
introduction to spell-checking by~\citep{norvig2010howto} has been of valuable
source of practical framework to test setup of the evaluation of the spelling
checkers as well as the baseline comparison for English language
spell-checking.

The efficiency evaluation is done in two articles, \citepalias{pirinen2012effects} I
wrote together with Sam Hardwick who did the most legwork on the finite-state
spell-checking implementation in e.g. \citep{linden2011hfst}---in this article
we measured the effects of different language and error model combinations to
the speed. Then in \citepalias{pirinen2013quality} I extended the same evaluation to
make more formal evaluation of the quality with same methods, using real
spelling errors made by Wikipedia contributors instead of traditional error
archives that are not a good representative set of what errors nowadays are, or
automatically made ones.

\section{Speed of Finite-State Spell-Checking}

\textbf{Main article:} \emph{Effect of Language and Error Models on Efficiency
of Finite-State Spell-Checking and Correction} by myself and Sam Hardwick. In
this article we set out to prove finite-state spell-checking is an efficient
solution and document how builders of the spellers can tune their parameters.

The \textbf{motivation} for the article was practical, in discussions with
people building alternative spelling checkers, as well as with alpha testers of
our more complex spell-checkers it was commonly noted that the speed might be
an issue for finite-state spell-checking systems to gain ground from the
traditional solutions. 

When discussing of such practical measurements as speed of spell-checker, a
care must be taken in evaluation metrics. The most common use of spell-checker
is still inter-active systems where a misspelt word is underlined when user
makes a mistake, and the system is capable of suggesting corrections when user
asks for them. This sets the limits that system should be capable of checking
text as it is typed, and responding with suggestion as they are 
demanded. It is well-known that finite-state language models that are single
tape automata, language recognisers, can answer the question of whether the 
string is in given language in linear time~\citep{aho2007compilers}, in
practical measurements this amounts to spell-checking times up to
100,000~strings per second~\citep{silfverberg2009hfst}. The central evaluation
and optimisation was therefore performed on the error correction part which
varies both with the language model and the error model selected.

In the article we study different \textbf{related} optimisation schemes from
non-finite-state spell-checking and evaluate the how different error models and
parameters affect the speed of finite-state the spell-checking. We also ran the
tests on scale of morphologically different languages to see if the concept of
morphological complexity reflects on the speed of spell-checking as it does to
the base features (node, edge counts etc.) of the language models.

The optimisation tricks of speed in spell-checking software has been a central
topic in the research since the beginning. There are numerous solutions that we
ported here to finite-state world to test their efficiency, and few methods
related to problems with the finite-state formulation of the error models (i.e.
the generated duplicate correction paths in the edit distance models larger
than 1 edits long, and limitation of result set size).

One interesting optimisation that we studied was the ignoring of the errors
at first-position of the world, this is known from many practical systems, but
scarcely documented or especially explained. Some of the research, such
as~\citep{bhagat2007spelling}, give some statistics on plausibility of such 
method, that it is more likely to make mistake in other parts of the words than
the very first letter, but do not offer hypothesis as to why this happens.
Practically, we replicated these results for our set of morphologically correct
languages as well, giving further evidence that it may be used as a possible
optimisation scheme.

It is likewise interesting, that even though we tried to devise methods to
prevent the larger error models of doing unnecessary things like removing and
adding the precisely same characters in succession, this did not provide
significant addition in the search speed of the corrections.

The comparative evaluation of different language models is relatively rarely
found in the literature. This could be due to the fact that scientific writing
in spell-checking is still more oriented towards algorithmics research, where
the language models of related systems may not have significant influence on
the asymptotic speed of the process and are thus considered uninteresting. Most
of the studies are centered towards measuring overall systems of single
languages, such as Arabic, Indian languages~\citep{chaudhuri2002towards},
Hungarian~\citep{tron2005hunmorph} or Spanish~\cite{otero2007contextual}, on top
of the default treatments on English~\citep{mitton1987spelling}.

\section{The Precision of Finite-State Spell-Checking}

\textbf{Main article:} \emph{Speed and Quality Trade-Offs in Weighted
Finite-State Spell-Checking}. In this article I take a broad look on evaluation
of all the language and error models and testing approaches I've developed in
the past years of research and compare them to existing results on real
systems.

\textbf{Motivation} for this larger article was to perform a larger survey
for all the finite-state spelling methods that have been created in the
project. In the paper I aim to show that the current state-of-the-art
finite-state spell-checking is starting to be viable option for many
common end-applications. The article is practical continuation to the speed
optimisation evaluation, where I studied the concept of speeding up the
finite-state spell-checking systems. In order for optimisations to be usable
for end-users I needed to show that the quality degradation caused by the
optimisation schemes that cut off part of a search space is not 
disproportionately big for the error models that are effective enough to be
used in the real world systems.

When measuring the quality of spelling error correction, the basic finding is
that the simple edit-distance measure at length of one will usually cover
majority of errors on typical typing error situations, for English the 
original statistics were at 80~\%~\citep{damerau1964technique}, with various
newer studies giving similar findings between 79~\% and 94~\% depending on
the type of corpora (hand-written and dictated versus normal typing).
The numbers for morphologically complex languages are not in the same range,
rather the edit distance of one is around 50 --- 70~\% of coverage. One
reason for such may be that the length of a word is a factor in the typing
error process; e.g. such that it is more likely to type two or more typos
and leave them uncorrected with a language where average word length is 16
as compared to one where it is 6. Intuitively this would make sense, and the
figures obtained from my tests in the article provide some evidence of this.

\section{Conclusions}

In this chapter I have shown the aspects of finite-state spell-checking that
make it a plausible end-user spell-checker as a practical application. I have
shown that the decrease in speed when moving from optimised specific
algorithmic methods to correct strings or traverse finite-state network to
actual finite-state algebra is negligible. The speed of finite-state solutions
for correcting morphologically complex languages is typically faster than
that of current software solutions such as hunspell.

I have tested extensive set of morphologically different languages and
large corpora to show the usability of finite-state language and error-models
for practical spell-checking applications and shown that it is possible to tune
the systems for interactive spell-checking with little effort using the
finite-state algebra.

\chapter{Conclusion}
\label{chap:conclusion}

In this thesis I have researched the plausibility of making finite-state
language and error models for real-world, finite-state spell-checking and
correction applications, and presented approaches to patch traditional
spell-checking solutions for finite-state use. The contributions of this work
is presented in two separate sections: section
\ref{sec:scholarly-contributions} summarises the scientific contributions of
the work that went to this thesis, and section
\ref{sec:practical-contributions} recounts the results of this thesis in terms
of real-world spelling checkers that can be used in end-user systems. Finally,
in \ref{sec:future-work} I go through the related topics that were restricted
outside the topic of this thesis, but are to be expected from results of this
thesis.

\section{Scholarly Contributions}
\label{sec:scholarly-contributions}

One of the main contributions, I believe, is the improvement and verification
of the statistical language modeling methods in context of weighted 
finite-state automata. I believe that throughout the thesis, by continously
using the initial finding of the word-form based compound word training
approach introduced in~\citepalias{pirinen2009weighted}, I have shown that the
morphologically complex languages can be statistically trained as weighted
finite-state automata and I maintain that this approach should be carried on
to majority of future finite-state language models.

In the spelling correction mechanics, I have stuck to the two-tape finite-state
automaton as ubiquitous error model for finite-state spelling correction, and
shown that it can be reasonably used in interactive spelling correction
software in place of any traditional spelling correction software.

As a contribution that is almost concerning to practical, software engineering
side of language modeling, I have shown that a generalised finite-state formula
based on knowledge of morphology can be used to compile all existing language
models into equivalent automata that the traditional, morphologically complex
languages use with their finites-state systems.

Finally, to tie in the scientific evaluation of finite-state spell-checking as
viable alternative, I have performed a large scale testing on wide range
morphologically different languages, with different resources, and shown that
all are plausibly usable for typical spelling correction situations.


\section{Practical Contributions}
\label{sec:practical-contributions}

The spell-checking software, that is the topic of this thesis, is very
practical real world application, that, despite this being an academic
dissertation, cannot be ignored, when talking about the contributions of this
thesis. One of the important contributions, I believe, is the alpha testing
version of Greenlandic spelling correction for OpenOffice, providing a kind of
a proof of concept that it is implementable.

The real world tool chain, consisting of libvoikko, enchant, and the GNOME
desktop, as well as the OpenOffice.org / Libreoffice and Firefox browser, has
been a good testing ground for other morphologically complex languages,
including Finnish and North Saami, but also the slew of forthcoming Uralic
languages currently being implemented in the University of Helsinki, most of
which have not ever had a spell-checking system at all. 

\section{Future Work}
\label{sec:future-work}

In terms of error correcting, the systems are deeply rooted in the basic models
of typign errors, such as Levenshtein-Damerau edit distance. It would be
interesting to try to implement more accurate models of errors based on
findings of cognitive-psychological studies on the kinds of errors that are
actually made---this path of research I believe is only scratched on the
surface with the optimisations like \emph{avoid making typing mistakes at the
beginning of the word}. Furthermore, it should be possible to make adaptive
error models using simple feedback system for the finite-state error-models.

During the time I spent writing this thesis, the world of spell-checking has
considerably changed---with the mass market popularity of a variety of input
methods in new touch screen user interfaces. When I started the by far the most
common methods to input text were regular 100+ key PC keyboards and T9 key pads
on mobile phones. This is no longer the case, as input methods like Swype, XT9,
and so on, gaining a market share. A revision of the methods introduced in this
thesis is required, especially so since the input methods of these systems are
\emph{more} heavily reliant on the high quality, automatic spelling correction
than before\footnote{case in point, there are collections of humorous automatic
    spelling corrections made by these systems floating around the internet,
such as \url{http://www.damnyouautocorrect.com/}}.

On the engineering and software management side, one very visible, practical,
perhaps even foreseeable future work, there would not be anything stopping from
these methods getting into use in real world spell-checkers. The actual
implementation has been done with help of spell-checking systems like voikko,
and enchant. The morphologically complex languages that are one selling point
of this thesis are lacking of the basic language technology support partially
due to the dominance of too limited string based forms of language modeling.

The concept of context-aware spell-checking with finite-state systems was
merely scratched on the surface by the article \citepalias{pirinen2012improving},
the results showed some promise and some problems, but with work it may be
possible to get as sizable improvements for morphologically complex languages
as for morphologically simple ones. One possible solution to be drawn from
the actual results of the other articles in the thesis is to approach the
morph based statistics for all the languages to get the possibly less scarce
statistics and more similar results. With regards to getting the speed and
memory efficiency to the level that is usable in everyday applications, I
suspect there is space for much improvement by basic engineering decisions and
optimisations. This has also been a topic in much of recent spell-checking
research for non-finite-state systems, e.g. in~\citep{carlson2001scaling}.

The concept of context-based error-detection, the real-word error detection,
and other forms of grammar correction has not been dealt with very large scale
in this thesis, that is an exclusion I have done partly on purpose, but in
fact the basic context-aware real-word error detection task does not really
have any practical differences to non-finite-state versions of it, and the
conversion is even more trivial than in the case of the error correction part
shown in the article mentioned \citepalias{pirinen2012improving}.

\bibliographystyle{apalike}
\bibliography{diss,arts}

\end{document}
% vim: set spell:
