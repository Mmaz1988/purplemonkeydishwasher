Tässä väitöskirjassa tutkin äärellistilaisten menetelmien käyttöä
oikaisuluvussa. Äärellistilaiset menetelmät mahdollistavat sananmuodostukseltaan
monimutkaisempien kielten, kuten suomen tai grönlannin, sanaston sujuvan
käsittelyn oikaisulukusovelluksissa. Käsittelen tutkielmassani tieteellisiä
ja käytännöllisiä toteutuksia, jotka ovat tarpeen, jotta tällaisia
sananmuodostukseltaan monimutkallisempia kieliä voisi käsitellä oikaisuluvussa
yhtä tehokkaasti kuin yksinkertaisempia kieliä, kuten englantia tai
muita indo-eurooppalaisia kieliä nyt käsitellään.

Tutkielmassa esitellään kolme keskeistä tutkimusongelmaa, jotka koskevat
oikaisuluvun toteuttamista sanarakenteeltaan monimutkaisemmille kielille:
miten mallintaa oikeinkirjoitetut sanamuodot äärellistilaisin mallein,
miten soveltaa tilastollista mallinnusta monimutkaisiin sanarakenteisiin kuten
yhdyssanoihin, ja miten mallintaa kirjoitusvirheitä äärellistilaisin mentelmin.

Tutkielman tuloksena esitän äärellistilaisia oikaisulukumenetelmiä soveltuvana
vaihtoehtona nykyisille oikaisulukimille, tämän todisteena esitän
mittaustuloksia, jotka näyttävät, että käyttämäni menetelmät toimivat niin
rakenteellisesti yksinkertaisille kielille kuten englannille yhtä hyvin kuin
nykyiset menetelmät että rakenteellisesti monimutkaisemmille kielille kuten
suomelle, saamelle ja jopa grönlannille riittävän hyvin tullakseen käytetyksi
tyypillisissä oikaisulukimissa.

