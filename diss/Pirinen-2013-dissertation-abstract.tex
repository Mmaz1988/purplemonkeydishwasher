This dissertation is a large-scale study of spell\-/checking and correction
using finite\-/state technology. Finite-state spell\-/checking is a key method
for handling morphologically complex languages in a computationally
efficient manner. This dissertation goes through technological and practical
considerations that are required for finite\-/state spell-checkers to be at
the same level as state-of-the-art non-finite\-/state spell-checkers. 

Three aspects of spell\-/checking are considered in the thesis: modelling of
correctly written words and word-forms with finite\-/state language
models, applying statistical information to finite\-/state language models
with a specific view to morphologically complex languages, and modelling
misspellings and typing errors using finite\-/state automata-based error
models.

The usability of finite\-/state spell-checkers as a viable alternative to
traditional string-algorithm solutions is demonstrated in a large-scale
evaluation of speed and quality of spell\-/checking with languages of
a morphologically different nature. The selected languages display a
full range of typological complexity from isolating English to 
poly-agglutinative Greenlandic via Finnish and the Saami languages.

