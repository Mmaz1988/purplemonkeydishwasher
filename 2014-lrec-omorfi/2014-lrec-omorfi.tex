\documentclass[a5paper]{article}

\usepackage{amssymb}
\usepackage{polyglossia}

\usepackage{fontspec}
\usepackage{xunicode}
\usepackage{xltxtra}

\usepackage{url}
\usepackage{hyperref}
\usepackage[obeyDraft]{todonotes}

\usepackage{graphicx}
\usepackage{geometry}

\setmainfont[Mapping=tex-text]{Liberation Serif}

\title{Collecting, Extending, Verifying and Regression Testing Lexical Data through
Use of Large Corpora---A Case Study in Management of Finnish Lexicon in Omorfi}

\author{Tommi A Pirinen \and Juha Kuokkala \and \ldots}

\date{\today}

\begin{document}

\maketitle

\begin{abstract} 
    
    In this article we demonstrate the use of large coverage varied real text
    corpora for four different purposes: first to collect new lexical data,
    e.g., words, classifications and other features, second to extend existing
    lexical data, third to verify lexical data collected from various sources
    and finally to regression test the final system using that data.  In our
    experiment we take a free and open source morphological analyser and freely
    available corpora to verify some existing features that are morphological
    (i.e., visible in the word-form) as well as to extend and to verify some
    features that are syntactic-semantic (i.e., visible in distribution and
    context). We also demonstrate that the resulting morphological and
    surface-syntactic statistics extracted in the process can be readily used
    as naive statistical language models. The experiment is performed using a
    mature morphological computational linguistic description of Finnish
    language that has been adapted for multiple applications and uses a variety
    of different sources for acquiring lexical data, to show the general
    applicability and usefulness of the method.

\end{abstract}

\section{Introduction}
\label{sec:introduction}

For a mature computational language description, harvesting and maintaining the
masses of lexical data is one of the main tasks done all the time. In a modern
environment, this task is no longer carried on by one computational-linguist
expert. Often the data are harvested from various crowd-sourced dictionary
development environments, such as
Wiktionary\footnote{\url{http://fi.wiktionary.org}} or
O\-mega\-wi\-ki\footnote{\url{http://www.omegawiki.org}}, and while they
ideally need to be verified by competent language speaker before integrating
into system, often the task is time consuming and prone to mistakes, so
providing rigorous test sets by use of extensible corpora is crucial to help
the expert to perform the task in a timely manner. Another aspect of lexical
data harvesting that often comes into focus is, when extending the use of a
computational linguistic description into new applications, it is invariably
necessary to create new classifications for words. For example the Finnish
morphology that we use has been adapted to Finnish-English machine
translation\footnote{\url{http://sourceforge.net/p/apertium/svn/HEAD/tree/nursery/apertium-fin-eng/}},
which would have needed, among other classifications, the gender of human
referent nominals in order to get personal pronouns right\footnote{Finnish uses
gender neutral third person singular whereas English has one for each gender
leaving a slot of missing information to be filled}. Another project adapted
the analyser to named entity recognition task, which in turn needed great
quantities of new proper nouns added to the dictionary and classified by
semantic means. Third aspect of lexical data maintenance is verifying the
integrity and definitions, because even if data comes from trustworthy sources
it may often be, that the definitions used by lexicographers have been
different when it comes to the classification, or there may be simple human
mistakes. In order to ensure that all lexical data from different sources works
with same computational linguistic description, large scale tests can provide
very helpful evidence of the uniformity of behaviour within a class.

The concept of using corpora as means of harvesting new data is not new in
itself, in fact probably most linguistic projects that make use of
lexicographical data will have at some point used corpora to extract new words
and features from, e.g., for extraction part alone, LREC held in 2012 had at
least 5 papers dealing with automatic extraction of lexical features.  What we
provide here is a software engineering oriented view on using corpora as
properly integrated test unit in all phases of the lexicon development for a
continuous and semi-automated maintenance of the lexical data instead of a
single update of a partial data.

The big picture behind this paper is that computational linguistics and
linguistics often have too big of a gap between them, which ends up in
linguistic grammars defining classifications too loosely and too full of
exceptions to be implementable. This in turn leads then computational linguists
to define classes that do not exist in linguistic grammars, which will appear
in the specific software show as one-off hacks, which only the users of these
specific systems will understand. If a classification in a grammar has a
criteria, it must be observable and thus testable, if it is observable it will
be usable in computational linguistics too, thus, I suggest combining the two
disciplines by making the criteria of linguistic classifications the first
class citizens of both the linguistic and the computational descriptions by use
of corpora and statistics.  This in part provides for better long-term
maintainability for the (computational) linguistic descriptions, an aspect of
grammar development that has been often called for in recent years~\cite{maxwell2013system,maxwell2008joint,moshagen2013building,pirinen2011modularisation}.

\section{Methods}

All of the methods from verification to regression testing are based on simple
statistic distributions applied over word-forms, tokens and analyses. The form
is always the same, proportion of features $f$ in context $C$ of feature $g$ 
from all features in same category in context $C$. The features can be 
single analyses of single like parts-of-speech matching $P$, the context can
be either the same analysis, or any of the analyses in contextual positions
left or right of this word in text. E.g., given analysis with part-of-speech
\emph{adjective}, what is the distribution of different derivation
\emph{comparative} analyses, or in context, given the part-of-speech
\emph{adjective}, what is the distribution of agreeing analyses with
part-of-speech \emph{noun} in the next word to the right. If a classification
requires adjectives to have comparative forms, we expect to see distribution of
comparative forms that shows non-zero amount of some variants of comparatives.

To implement these we use the python\footnote{\url{http://python.org}}
scripting language to gather the basic statistics from the output of a
morphological analyser in terms of distributions of analyses per input token
and analyses of specific context\footnote{the implementation is freely
available in \url{http://code.google.com/p/omorfi/}, we consider providing free and open data and
implementations crucial for scientific reproduction and replication purposes}.
In our case we have an underlying python interface to FST-based corpus
processing provided by free open source project HFST~\cite{linden2013hfst},
this is used to apply the morphological analyser to raw corpora to get the
ambiguous analyses for the statistics. From the analyses we gather the
statistics based on linguistic rules according to categories, such as
parts-of-speech, and their defined classifications.

The classification used
are mainly retrieved from the official grammar~\cite{visk}, and should
therefore form an ideal test for collecting, verifying and testing the
linguistic classifications for use of a computational system. In the grammar
classifications, e.g., (certain kinds of) \emph{adjectives}\footnote{Word
    `adjective' is somewhat problematically defined loosely in terms of
    morphology, syntax and semantics together, which makes classification often
    troublesome for computational methods; in future it might be sensible to
separate these tasks under different class names. The current grammar actually
literally uses the expression \emph{certain kinds of} when defining adjectives'
different classes.} are recognised by the existence of comparative forms, and
the lack of those forms will be a good clue of misclassification in the lexical
data we currently use. On the other hand (certain kinds of) \emph{adjectives}
are defined by having an agreeing nominal on right context or predicative verb
in the left context. To test these classifications we gather the adjectives
comparative form statistics, and their agreeing nominal statistics. Another
feature that is interesting to gather for Finnish is the position and form of
the complements for e.g., adpositions, which generally can take nominal
complements in one of six cases and in either direction; traditionally
following models from other languages the Finnish grammars have often defined
adposition to be one of preposition, postposition or both, but with proper
distributions from corpora we can easily improve this feature. This can be
captured by collecting distributions of left and right nominals and their cases
for the words that are analysed as adpositions. Also, some of the seemingly
semantic features can also be guessed and verified using merely morphological
distributions like this, e.g., whether a nominal is a title correlates with
humanness of its right nominal complement.

\section{Data}

For experimentation we have used a mature language description of
Finn\-ish~\cite{pirinen2008suomen} that consists some 400,000 lexemes and a few
existing classifications, such as part-of-speech classes, inflectional
paradigms, and proper noun classes, for full view of all, it is advised to see
the project's home site\footnote{\url{http://code.google.com/p/omorfi/}}. The
lexical data comes from number of projects varying from institute of languages
in Finland's official
dictionary\footnote{\url{http://kaino.kotus.fi/nykysuomi/sanat}} and academic
projects to fully crowd-sourced projects like Wiktionary. To test out that the
classifications match the expectations we have used three large and
well-established free and open source corpora:
Wikipedia\footnote{\url{http://download.wikimedia.org/}}, Project
Gutenberg\footnote{\url{http://www.gutenberg.org}} and texts of the European
commission in JRC Acquis
corpus\footnote{\url{http://ipsc.jrc.ec.europa.eu/index.php?id=198}}~\cite{steinberger2006jrc}.
The fetching and preprocessing of the data has been done using tools available
freely from a web
repository\footnote{\url{https://github.com/flammie/bash-corpora}}.

\section{Evaluation}

To evaluate the scheme we have randomly sampled the statistics that give the
strongest evidence (multiple matches with over 95~\% of agreement or less than 
5~\% in case of negative matches) of either
disproving the current class or extending the classification with new
information, and evaluated them using linguistic expertise and the specific
contexts. The results are given in table~\ref{table:results}.

\section{Discussion}

In previous work the corpora have been used for single time extension of the
lexical data or coverage based maintenance of computational linguistic
description. We have presented a sustainable way of integrating corpora as part
of the development work-flow to extend, verify and maintain the lexical data
and to keep the language description clean and free of errors.

\subsection{Error Analysis}

When looking at some of the tests that fail to give good results, we notice
that even with as large corpora as we use, the lack of data may leave existing
linguistic phenomena non-existing. For example both the data in our corpora and
official grammar classify `\emph{kielitieteellinen}' (linguistic) as an
non-comparing adjective, yet a simple Google search yields a number of real
world results, that are grammatical, such as: ``\emph{Torstain haastavampaan
    settiin lukeutuu sit se avoin \textbf{kielitieteellisempi} kurssi, jossa
valtaosa opiskelijoista on
japanilaisia}''\footnote{\url{http://7830km.blogspot.fi/2013/04/mulla-kauas-paluulippu-on.html}}
(\ldots open \textbf{more linguistic} course \ldots) and ``\emph{Tosin aihe on
\textbf{kielitieteellisempi} kuin muistaakseni mikään kielitiederyhmään viime
kuukausina tulleista artikkelista,}''
\footnote{\url{news:oiiup4p6hj.fsf@beta.hut.fi}} (\ldots is \textbf{more
linguistic} than any other \ldots). For this reason there needs to be methods
for an expert to override and blacklist / whitelist the results as necessary.

\subsection{Future Work}

In this experiment we worked with ambiguous morphological analyses and simple
positional rules only capable of scanning specific relative
positions. For many of the real linguistic definitions, it would be interesting
to have better analyses and word-to-word relations available, e.g., such that
dependency tree banks like~\cite{haverinen2013building} have.

\section{Conclusion}

In this article we applied large corpora-based extraction, verification and
testing for Finnish language description. Results show that even with quite
tight thresholds we are able to extract good new data and clean up existing
data with accuracy of nnn~\%.

\bibliographystyle{unsrt}
\bibliography{lrec2014}

\end{document}
% vim: set spell:
