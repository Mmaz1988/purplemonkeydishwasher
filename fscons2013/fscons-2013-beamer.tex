\documentclass[t,12pt]{beamer}
\usepackage{helvet}
\usepackage{times}
\usepackage{courier}

\usepackage[T1]{fontenc}
\usepackage[english]{babel}

\usepackage{amssymb}
\usepackage{amsmath}
\usepackage{amsfonts}
\usepackage{graphicx}
\usepackage{color}
\usepackage{url}
\usepackage{textpos}
\usepackage{xspace}
\usepackage{array}

\graphicspath{{./fig/}}


% theme options: hy/ml/hum, rovio/sinetti, hiit
% default: hy,rovio

%\usetheme[hy]{HY}
%\usetheme[hy,sinetti]{HY}
\usetheme[hum,rovio]{HY}
%\usetheme[ml,rovio]{HY}
%\usetheme[ml,rovio,hiit]{HY}


\title[Open- and Crowd-sourced Lexicography]{Open-Source Morphologies and Crowd-Sourcing Lexicography}
\author[Tommi A Pirinen]{Tommi A Pirinen \scriptsize \guilsinglleft{}tommi.pirinen@helsinki.fi\guilsinglright{}}
\institute[University of Helsinki]{Department of Speech Sciences\\University of Helsinki}
\date{\today (draft)}

\subject{Research}



\AtBeginSubsection[]
{
  \begin{frame}<beamer>{Outline}
    \tableofcontents[currentsection,currentsubsection]
  \end{frame}
}


\begin{document}

\selectlanguage{english}

\HyTitle
%\maketitle

\begin{frame}{outline}
    \frametitle{Outline}
    \tableofcontents
\end{frame}

\section{Part 1: Crowd-Sourcing and Lexical Data Concepts and Experiences}

\subsection{Introduction: Concepts}

\begin{frame}{Morphology}
    \begin{itemize}
        \item inflect words
        \item in a broad sense: classifying words, inflectional suffixes, etc.
        \item E.g., \emph{hundarnas} = hund + ar + n + as = dog, common gender,
            needs ar as plural suffix, possessive form
        \item to reach a system dealing with this we need data about
            words, leading to...
    \end{itemize}
\end{frame}

\begin{frame}{Lexicography}
    \begin{itemize}
        \item ``Dictionary writing'', in this context more like data harvesting
        \item Collect all words
        \item How do they inflect (i.e., which are the valid forms of the word)
        \item How do they operate with other words in sentence (syntax)
        \item What do words mean, how do you translate them (semantics)
    \end{itemize}
\end{frame}

\begin{frame}{Example of trad. dictionary}
\end{frame}

\begin{frame}{One example of Digital Dictionary}
    \texttt{}
\end{frame}

\begin{frame}{Crowd-sourcing}
    \begin{itemize}
        \item Getting lots of people to work on same project
        \item Wikipedia is the best success story here
        \item Ideal for lexicography: no special skills needed, all native
            speakers know words of their language
        \item There are projects for dictionary building as well:
            Wiktionary, Omegawiki, \ldots (not as huge success stories, yet)
    \end{itemize}
\end{frame}

\subsection{Crowd-sourcing: uses and issues}

\begin{frame}{Levels of Crowd-Sourcing}
    \begin{enumerate}
        \item Using data (long) after it has been built by harvesting, scraping,
            etc., this is where we are mostly at with Wiktionary
        \item Requesting specific data 
        \item Designing system to collect data in good manner, like Omegawiki
    \end{enumerate}
\end{frame}

\begin{frame}{Example of Wiktionary Scraping}

\end{frame}

\section{Part 2: Productising Research Results}

\subsection{Introduction}


\end{document}
% vim: set spell:
