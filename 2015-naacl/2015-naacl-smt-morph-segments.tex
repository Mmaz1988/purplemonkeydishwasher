\documentclass[11pt,letterpaper]{article}
\usepackage{naaclhlt2015}
\usepackage{times}
\usepackage{latexsym}
\setlength\titlebox{6.5cm}    % Expanding the titlebox

\title{Morphological segmentation and de-segmentation for statistical machine
    translation\Thanks{This is an unpublished
authors draft}}}

\author{Tommi A Pirinen\\
    Ollscoil Chathair Bhaile Átha Cliath\\
    CNGL---School of Computing\\
    Dublin City University, Glasnevin, D9\\
    Dublin, Ireland\\
    {\tt tommi.pirinen@computing.dcu.ie}
}

\begin{document}
\maketitle
\begin{abstract}
    Statistical machine translation works well when working between languages
    with poor or moderately poor morphology, but typically fail if one
    of the languages is much richer in morphology than other. In this article
    we present an experiment on Finnish-English statistical machine translation
    using segmentation and de-segmentation of morphs as pre- and post-processing
    step to the standard statistical phrase based machine translation scheme
    as presented by moses. We compare the use of statistical and rule-based
    morphological segmenters and different approaches to pick up the
    ideal segmentations at different phases of translation process as well
    as different methods of de-segmentation. We note that the best system
    improves the BLEU score by 0.001 points.
\end{abstract}

\section{Introduction}
\label{sec:introduction}



\section{Methods}

\section{Data}

\section{Evaluation}

\section{Discussion}

\section{Conclusion}

\section*{Acknowledgments}


\bibliographystyle{naaclhlt2015}
\bibliography{naacl2015}

\end{document}


