\documentclass[t]{beamer}

\usepackage{fontspec}
\usepackage{polyglossia}

\usepackage{graphicx}
\usepackage{color}
\usepackage{url}


\title{ACL SIGUR, UD, CLARIN etc., how-to harmonising standards and progress,
\scriptsize{in eerbmt, Alicante, 2016.\\
\url{http://gtweb.uit.no/sigur}}}
\author{T.~A.~Pirinen 
\scriptsize \guilsinglleft tommi.antero.pirinen@uni-hamburg.de \guilsinglright }
\institute{Universität Hamburg, ACL SIGUR, CLARIN-D and various others}
\date{\today}

\begin{document}

\selectlanguage{english}

\AtBeginSection[]{\begin{frame}{Outline}
    \tableofcontents[currentsection]
\end{frame}}

\maketitle

\begin{frame}
    \frametitle{Structure of the Presentation}
    \tableofcontents
\end{frame}

\section{Intro}

\begin{frame}
    \frametitle{Background ``before I start''}
    How does ACL SIGUR fit into RBMT and interoperability?
    \begin{itemize}
        \item getting people to agree on analyses / tags / features
            etc. one transfers in RBMT is hard
        \item UD is a good project towards this goal...,
        \item at version 1.3 three languages from same family mark some same
            things in up to three (four?) different ways
        \item Can we (ACL SIGUR) bring order into this chaos?
    \end{itemize}
\end{frame}

\begin{frame}
    \frametitle{Obligatory triangle}
    \only<1>{\includegraphics[width=\textwidth]{triforce}}
    \only<2>{\includegraphics[width=\textwidth]{vauquois}
    \textit{CC BY-SA 3.0: Francis Tyers / wikimedia commons}
    \url{https://commons.wikimedia.org/wiki/File:Direct_translation_and_transfer_translation_pyramind.svg}
    }
\end{frame}

\section{ACL SIGUR}

\begin{frame}
    \frametitle{ACL}
    Association for computational linguistics:
    \begin{itemize}
        \item \url{http://aclweb.org}
        \item Members are academics, professionals etc. interested in CL
        \item Responsible for many high-impact conferences, publications
            etc. in CL
    \end{itemize}
\end{frame}

\begin{frame}
    \frametitle{SIG}
    Special Interest Group:
    \begin{itemize}
        \item A (loosely organised) group of people interested in a
            sub-topic of Computational linguistics (e.g. a language
            group, a theory or sub-field)
        \item organises events, publications, usually also resources,
            standards and best common practices
        \item 
    \end{itemize}
\end{frame}

\begin{frame}
    \frametitle{UR}
    as in SIGUR, Special Interest Group for Uralic Ranguages
    \url{http://gtweb.uit.no/sigur}
    \begin{itemize}
        \item Newly founded ACL SIG
        \item chaired by me
        \item organising yearly workshops, journal publications
        \item first group effort: harmonising and expanding Uralic UD
    \end{itemize}
\end{frame}


\section{Case: Uralic Universal Dependencies}

\begin{frame}
    \frametitle{Uralic Universals}
    \includegraphics[width=0.7\textwidth]{uralic-universals}
\end{frame}

\begin{frame}
    \frametitle{Problems}
    Languages that are quite known to be related and even mutually intelligible
    should probably share the same analyses for diachronically and synchronically
    same morphs and structures
    \begin{itemize}
        \item For longer presentation on the Uralic tagging issues,
            see Tyers \& Pirinen (2016) at
            \url{http://www.computing.dcu.ie/~tpirinen/Pirinen-2016-iwclul-rbmt-representations.pdf}
        \item E.g. The past participle in Finnish is 
            \texttt{VerbForm=Part|PartForm=Pres}, Estonian 
            \texttt{VerbForm=Part|Tense=Pres} and Hungarian \texttt{VerbForm=PartPres}
        \item if you used UD to UD transfer, this could've been a free ride but
            isn't!
    \end{itemize}
\end{frame}

\begin{frame}
    \frametitle{Solutions??}
    We just starting this work:
    \begin{itemize}
        \item Read and write UD documentations for intermediat level between
            Universal and language specific \url{http://universaldependencies.org/urj/overview/introduction.html}
        \item Write a lot of bugs \url{https://github.com/UniversalDependencies/docs/labels/Uralic}
        \item Work with authors of current UD treebanks (help / bribe / coerce)
            to harmonise and, work with new languages from beginning on as
            guide
        \item ...
    \end{itemize}
\end{frame}

\section{Other Standardisations and efforts}

\begin{frame}
    \frametitle{Standards...}
    \includegraphics[width=\textwidth]{xkcd-standards}
    \textit{CC BY-NC 2.5: by Randall Munroe / XKCD} 
    \url{http://imgs.xkcd.com/comics/standards.png}
\end{frame}

\begin{frame}
    \frametitle{CLARIN}
    Common Language Resource Infrastructure Network
    \url{http://clarin.eu}
    \begin{itemize}
        \item Big EU / International effort
        \item Good track record with organsing infrastructures, resources,
            peoples
        \item Standards group not hugely succesful with standards relevant to
            RBMT...
        \item Standards coordination is traditionally slow-moving process
        \item CLARIN standards can go ISO etc.
    \end{itemize}
\end{frame}

\begin{frame}
    \frametitle{CLARIN's managed / endorsed standards relevant for RBMT}
    \begin{itemize}
        \item ISOcat (now defunct) $\rightarrow$ CLARIN Category Registry
        \item Lexical Markup Framework
        \item Virtual Language Observatory / Metashare etc. (repositories of
            things)
        \item no Universal xxx yet (UPOS is part of some CLARIN standards)
    \end{itemize}
    Let's find them all, \url{http://clarin.eu} (if we have time)
\end{frame}

\section{Conclusion}

\begin{frame}
    \frametitle{Inconclusion}
    \begin{itemize}
        \item is hard to get compatible tagsets even with universal 
            guidelines
        \item working with 3+ contributors and resources with known comparable
            langs may turn out to be easier than with dozens
        \item we have a cool \textit{bazaar} spot at UD and a nice 
            \textit{cathedral} with CLARIN
    \end{itemize}
\end{frame}

\section{Acknowledgments}

\begin{frame}
    \frametitle{Acknowledgements}
    This presentation was brought to you by: CLARIN-D\\
    \vspace{2em}
    Questions and Answers?
\end{frame}

\end{document}
